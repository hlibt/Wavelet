\documentclass[10.5pt]{article}

% Packages:
\usepackage[utf8]{inputenc}
\usepackage[english]{babel}
\usepackage{float}
\usepackage{graphicx}
\usepackage{amssymb}
\usepackage{pgfplots}
\usepackage{bm}
\usepackage{mathtools}          %loads amsmath as well
\usepackage{titling}
\usepackage{biblatex}
\usepackage{indentfirst}
\addbibresource{bibliography.bib}
\DeclareGraphicsRule{.tif}{png}{.png}{`convert #1 `dirname #1`/`basename #1 .tif`.png}

% Move title up:
\setlength{\droptitle}{-10em}

\begin{document}

% Title information:
\title{Study of Multiresolution Concepts for Numerically Solving PDEs}
\author{Brandon Gusto \\}
\maketitle

% Introductory section:
\section*{Motivation for Multiresolution Methods}
Computer simulation of physical phenomenon has become a centerpiece of modern engineering design. As more accuracy is demanded,
and more complex physics need to be resolved, the demand on computing power increases dramatically. In recent years, processing power for a single
chip has stopped making significant year-over-year gains. In light of this, computational scientists around the world are coming up with
new ways to handle the tremendous amount of computations necessitated by numerical simulations. 
Computer simulations are designed to numerically solve the (usually simplified) system of partial differential equations which govern
the behavior of the physics of interest. Real-world simulations typically run with a computational grid that contains tens or hundreds 
of millions of unknowns. Since the 1980s, computational scientists have devised methods which try to reduce the number of degrees of 
freedom (unknowns) in a system. Adaptive mesh refinement (AMR) \cite{berger} was the first method of adaptively refining the solution where 
more accuracy is needed. AMR is still the standard in most production and research codes today. AMR typically refines the solution
by running the simulation on a coarse and fine grid simultaneously, and estimating via Richardson extrapolation the truncation error.

An alternative to AMR methods for providing a reliable measure of error is multiresolution analysis (MRA). MRA provides a hierarchy 
of nested approximation spaces, with a coarsest and finest approximation. Errors between levels of resolution are stored in the detail
wavelet spaces. In the last few decades, multiresolution methods have gained considerable attention and have been the focus
of research in a variety of fields, most notably signal processing and computer graphics.
Numerically solving differential equations is a more recent application of multiresolution properties, and has shown
to be an attractive alternative to traditional methods for particular problems. Some methods simply use the compression properties
of wavelets to provide spatial adaptivity to a numerical method, while other methods such as wavelet galerkin or wavelet collocation
use the wavelet basis explicitly to reconstruct a field. Some important qualities of the method which make it so capable 
for spatial adaptivity are that
\begin{itemize}
    \item wavelets are localized in space and scale
    \item the wavelet basis forms a multiresolution analysis
    \item the existence of a fast discrete wavelet transform makes the method $\mathcal{O}(\mathcal{N})$
\end{itemize} 
One of the most attractive qualities of using wavelets for solving partial differential equations (PDEs) 
or systems of PDEs is for their ability to naturally compress the solution, providing high resolution where 
it is needed, and fewer grid points where the solution is smooth. In a study between the compression properties and 
time-to-solution \cite{amr_vs_mra}, multiresolution methods showed a greated ability for compression of solution when
large disparity was present between length scales in the problem.

% Review of wavelets:
\section*{Brief Review of Wavelets}
A wavelet is a mathematical function which can be used to generate a basis for representing either continuous functions or 
discrete signals. There exist many other bases which have been utilized, most notably the Fourier
basis composed of sine and cosine functions.
Unlike the Fourier basis functions however, wavelets are functions with finite length. The most simple example of a wavelet
is the Haar wavelet, which has a scaling function, $\phi(x)$, and a mother wavelet $\psi(x)$ as shown in 
figure (\ref{fig:haar_scaling}) and (\ref{fig:haar_wavelet}).
\begin{figure}[H]
	\centering
	% This file was created by matlab2tikz.
%
%The latest updates can be retrieved from
%  http://www.mathworks.com/matlabcentral/fileexchange/22022-matlab2tikz-matlab2tikz
%where you can also make suggestions and rate matlab2tikz.
%
\begin{tikzpicture}

\begin{axis}[%
width=5.738in,
height=4.625in,
at={(1.29in,0.763in)},
scale only axis,
every outer x axis line/.append style={black},
every x tick label/.append style={font=\color{black}},
every x tick/.append style={black},
xmin=-0.2,
xmax=1.2,
xlabel={$x$},
every outer y axis line/.append style={black},
every y tick label/.append style={font=\color{black}},
every y tick/.append style={black},
ymin=-1,
ymax=1.2,
ylabel={$\phi(x)$},
axis background/.style={fill=white},
axis x line*=bottom,
axis y line*=left
]
\addplot [color=black, line width=4.0pt, forget plot]
  table[row sep=crcr]{%
-0.2	0\\
-0.192964824120603	0\\
-0.185929648241206	0\\
-0.178894472361809	0\\
-0.171859296482412	0\\
-0.164824120603015	0\\
-0.157788944723618	0\\
-0.150753768844221	0\\
-0.143718592964824	0\\
-0.136683417085427	0\\
-0.12964824120603	0\\
-0.122613065326633	0\\
-0.115577889447236	0\\
-0.108542713567839	0\\
-0.101507537688442	0\\
-0.0944723618090452	0\\
-0.0874371859296483	0\\
-0.0804020100502513	0\\
-0.0733668341708543	0\\
-0.0663316582914573	0\\
-0.0592964824120603	0\\
-0.0522613065326633	0\\
-0.0452261306532664	0\\
-0.0381909547738694	0\\
-0.0311557788944724	0\\
-0.0241206030150754	0\\
-0.0170854271356784	0\\
-0.0100502512562814	0\\
-0.00301507537688445	0\\
0.00402010050251253	1\\
0.0110552763819095	1\\
0.0180904522613065	1\\
0.0251256281407035	1\\
0.0321608040201005	1\\
0.0391959798994974	1\\
0.0462311557788945	1\\
0.0532663316582914	1\\
0.0603015075376884	1\\
0.0673366834170854	1\\
0.0743718592964824	1\\
0.0814070351758794	1\\
0.0884422110552764	1\\
0.0954773869346733	1\\
0.10251256281407	1\\
0.109547738693467	1\\
0.116582914572864	1\\
0.123618090452261	1\\
0.130653266331658	1\\
0.137688442211055	1\\
0.144723618090452	1\\
0.151758793969849	1\\
0.158793969849246	1\\
0.165829145728643	1\\
0.17286432160804	1\\
0.179899497487437	1\\
0.186934673366834	1\\
0.193969849246231	1\\
0.201005025125628	1\\
0.208040201005025	1\\
0.215075376884422	1\\
0.222110552763819	1\\
0.229145728643216	1\\
0.236180904522613	1\\
0.24321608040201	1\\
0.250251256281407	1\\
0.257286432160804	1\\
0.264321608040201	1\\
0.271356783919598	1\\
0.278391959798995	1\\
0.285427135678392	1\\
0.292462311557789	1\\
0.299497487437186	1\\
0.306532663316583	1\\
0.31356783919598	1\\
0.320603015075377	1\\
0.327638190954774	1\\
0.334673366834171	1\\
0.341708542713568	1\\
0.348743718592965	1\\
0.355778894472362	1\\
0.362814070351759	1\\
0.369849246231156	1\\
0.376884422110553	1\\
0.38391959798995	1\\
0.390954773869347	1\\
0.397989949748744	1\\
0.405025125628141	1\\
0.412060301507538	1\\
0.419095477386935	1\\
0.426130653266332	1\\
0.433165829145729	1\\
0.440201005025126	1\\
0.447236180904522	1\\
0.45427135678392	1\\
0.461306532663317	1\\
0.468341708542714	1\\
0.47537688442211	1\\
0.482412060301507	1\\
0.489447236180904	1\\
0.496482412060301	1\\
0.503517587939698	1\\
0.510552763819095	1\\
0.517587939698492	1\\
0.524623115577889	1\\
0.531658291457286	1\\
0.538693467336683	1\\
0.54572864321608	1\\
0.552763819095477	1\\
0.559798994974874	1\\
0.566834170854271	1\\
0.573869346733668	1\\
0.580904522613065	1\\
0.587939698492462	1\\
0.594974874371859	1\\
0.602010050251256	1\\
0.609045226130653	1\\
0.61608040201005	1\\
0.623115577889447	1\\
0.630150753768844	1\\
0.637185929648241	1\\
0.644221105527638	1\\
0.651256281407035	1\\
0.658291457286432	1\\
0.665326633165829	1\\
0.672361809045226	1\\
0.679396984924623	1\\
0.68643216080402	1\\
0.693467336683417	1\\
0.700502512562814	1\\
0.707537688442211	1\\
0.714572864321608	1\\
0.721608040201005	1\\
0.728643216080402	1\\
0.735678391959799	1\\
0.742713567839196	1\\
0.749748743718593	1\\
0.75678391959799	1\\
0.763819095477387	1\\
0.770854271356784	1\\
0.777889447236181	1\\
0.784924623115578	1\\
0.791959798994975	1\\
0.798994974874372	1\\
0.806030150753769	1\\
0.813065326633166	1\\
0.820100502512563	1\\
0.82713567839196	1\\
0.834170854271357	1\\
0.841206030150754	1\\
0.848241206030151	1\\
0.855276381909548	1\\
0.862311557788945	1\\
0.869346733668342	1\\
0.876381909547739	1\\
0.883417085427136	1\\
0.890452261306533	1\\
0.89748743718593	1\\
0.904522613065327	1\\
0.911557788944724	1\\
0.918592964824121	1\\
0.925628140703518	1\\
0.932663316582915	1\\
0.939698492462312	1\\
0.946733668341708	1\\
0.953768844221106	1\\
0.960804020100502	1\\
0.9678391959799	1\\
0.974874371859296	1\\
0.981909547738693	1\\
0.98894472361809	1\\
0.995979899497487	1\\
1.00301507537688	0\\
1.01005025125628	0\\
1.01708542713568	0\\
1.02412060301508	0\\
1.03115577889447	0\\
1.03819095477387	0\\
1.04522613065327	0\\
1.05226130653266	0\\
1.05929648241206	0\\
1.06633165829146	0\\
1.07336683417085	0\\
1.08040201005025	0\\
1.08743718592965	0\\
1.09447236180905	0\\
1.10150753768844	0\\
1.10854271356784	0\\
1.11557788944724	0\\
1.12261306532663	0\\
1.12964824120603	0\\
1.13668341708543	0\\
1.14371859296482	0\\
1.15075376884422	0\\
1.15778894472362	0\\
1.16482412060301	0\\
1.17185929648241	0\\
1.17889447236181	0\\
1.18592964824121	0\\
1.1929648241206	0\\
1.2	0\\
};
\end{axis}
\end{tikzpicture}%
	\caption{Haar scaling function}
	\label{fig:haar_scaling}
\end{figure}
\begin{figure}[H]
	\centering
	% GNUPLOT: LaTeX picture
\setlength{\unitlength}{0.240900pt}
\ifx\plotpoint\undefined\newsavebox{\plotpoint}\fi
\sbox{\plotpoint}{\rule[-0.200pt]{0.400pt}{0.400pt}}%
\begin{picture}(1500,900)(0,0)
\sbox{\plotpoint}{\rule[-0.200pt]{0.400pt}{0.400pt}}%
\put(171.0,192.0){\rule[-0.200pt]{4.818pt}{0.400pt}}
\put(151,192){\makebox(0,0)[r]{$-1$}}
\put(1419.0,192.0){\rule[-0.200pt]{4.818pt}{0.400pt}}
\put(171.0,343.0){\rule[-0.200pt]{4.818pt}{0.400pt}}
\put(151,343){\makebox(0,0)[r]{$-0.5$}}
\put(1419.0,343.0){\rule[-0.200pt]{4.818pt}{0.400pt}}
\put(171.0,495.0){\rule[-0.200pt]{4.818pt}{0.400pt}}
\put(151,495){\makebox(0,0)[r]{$0$}}
\put(1419.0,495.0){\rule[-0.200pt]{4.818pt}{0.400pt}}
\put(171.0,647.0){\rule[-0.200pt]{4.818pt}{0.400pt}}
\put(151,647){\makebox(0,0)[r]{$0.5$}}
\put(1419.0,647.0){\rule[-0.200pt]{4.818pt}{0.400pt}}
\put(171.0,798.0){\rule[-0.200pt]{4.818pt}{0.400pt}}
\put(151,798){\makebox(0,0)[r]{$1$}}
\put(1419.0,798.0){\rule[-0.200pt]{4.818pt}{0.400pt}}
\put(171.0,131.0){\rule[-0.200pt]{0.400pt}{4.818pt}}
\put(171,90){\makebox(0,0){$-0.2$}}
\put(171.0,839.0){\rule[-0.200pt]{0.400pt}{4.818pt}}
\put(352.0,131.0){\rule[-0.200pt]{0.400pt}{4.818pt}}
\put(352,90){\makebox(0,0){$0$}}
\put(352.0,839.0){\rule[-0.200pt]{0.400pt}{4.818pt}}
\put(533.0,131.0){\rule[-0.200pt]{0.400pt}{4.818pt}}
\put(533,90){\makebox(0,0){$0.2$}}
\put(533.0,839.0){\rule[-0.200pt]{0.400pt}{4.818pt}}
\put(714.0,131.0){\rule[-0.200pt]{0.400pt}{4.818pt}}
\put(714,90){\makebox(0,0){$0.4$}}
\put(714.0,839.0){\rule[-0.200pt]{0.400pt}{4.818pt}}
\put(896.0,131.0){\rule[-0.200pt]{0.400pt}{4.818pt}}
\put(896,90){\makebox(0,0){$0.6$}}
\put(896.0,839.0){\rule[-0.200pt]{0.400pt}{4.818pt}}
\put(1077.0,131.0){\rule[-0.200pt]{0.400pt}{4.818pt}}
\put(1077,90){\makebox(0,0){$0.8$}}
\put(1077.0,839.0){\rule[-0.200pt]{0.400pt}{4.818pt}}
\put(1258.0,131.0){\rule[-0.200pt]{0.400pt}{4.818pt}}
\put(1258,90){\makebox(0,0){$1$}}
\put(1258.0,839.0){\rule[-0.200pt]{0.400pt}{4.818pt}}
\put(1439.0,131.0){\rule[-0.200pt]{0.400pt}{4.818pt}}
\put(1439,90){\makebox(0,0){$1.2$}}
\put(1439.0,839.0){\rule[-0.200pt]{0.400pt}{4.818pt}}
\put(171.0,131.0){\rule[-0.200pt]{0.400pt}{175.375pt}}
\put(171.0,131.0){\rule[-0.200pt]{305.461pt}{0.400pt}}
\put(1439.0,131.0){\rule[-0.200pt]{0.400pt}{175.375pt}}
\put(171.0,859.0){\rule[-0.200pt]{305.461pt}{0.400pt}}
\put(30,495){\makebox(0,0){$\psi(x)$}}
\put(805,29){\makebox(0,0){$x$}}
\put(171,495){\usebox{\plotpoint}}
\multiput(349.59,495.00)(0.485,23.071){11}{\rule{0.117pt}{17.414pt}}
\multiput(348.17,495.00)(7.000,266.856){2}{\rule{0.400pt}{8.707pt}}
\put(171.0,495.0){\rule[-0.200pt]{42.880pt}{0.400pt}}
\multiput(802.59,629.88)(0.482,-54.735){9}{\rule{0.116pt}{40.500pt}}
\multiput(801.17,713.94)(6.000,-521.940){2}{\rule{0.400pt}{20.250pt}}
\put(356.0,798.0){\rule[-0.200pt]{107.441pt}{0.400pt}}
\multiput(1254.59,192.00)(0.485,23.071){11}{\rule{0.117pt}{17.414pt}}
\multiput(1253.17,192.00)(7.000,266.856){2}{\rule{0.400pt}{8.707pt}}
\put(808.0,192.0){\rule[-0.200pt]{107.441pt}{0.400pt}}
\put(1261.0,495.0){\rule[-0.200pt]{42.880pt}{0.400pt}}
\put(171.0,131.0){\rule[-0.200pt]{0.400pt}{175.375pt}}
\put(171.0,131.0){\rule[-0.200pt]{305.461pt}{0.400pt}}
\put(1439.0,131.0){\rule[-0.200pt]{0.400pt}{175.375pt}}
\put(171.0,859.0){\rule[-0.200pt]{305.461pt}{0.400pt}}
\end{picture}

	\caption{Haar wavelet function}
	\label{fig:haar_wavelet}
\end{figure}
The scaling and wavelet functions have compact support, thus to approximate functions it is necessary to shift them. 
This action is \textit{translation}. Furthermore, since functions may exhibit sharp changes over an
interval which is smaller than the piecewise constant wavelet, a wavelet of higher frequency is necessary
to fit the function. A \textit{dilation} of the original wavelet decreases its support, allowing it to 
approximate more complex functions. These basic operations allow for the formation of a basis.
The basis is constructed from the scaling and wavelet functions as
\begin{align}
\phi_{j,k}(x) & = 2^{j/2} \phi(2^j x - k) \\
\psi_{j,k}(x) & = 2^{j/2} \psi(2^j x - k).
\label{dilation equation}
\end{align}
Only in the limit as $j$ tends to infinity does this basis fully represent a continous function.
The three main qualities of wavelets which make them such a useful tool for 
approximation are \textbf{compact support}, \textbf{orthogonality}, and \textbf{multiresolution analysis}.

\subsection*{Compact Support}
Wavelets are uniformly zero outside of a specified interval, and this useful property is known as compact support.
This property is advantageous for approximating sharp signals, where one would like to represent the spike using only a small 
handful of basis functions. The Fourier basis is not adequate for this purpose, since the support for the basis is global; 
the sine and cosine functions have infinite length. 

\subsection*{Orthogonality}
The basis of wavelet functions constitute an orthonormal set. All translations and dilations of the 
wavelet $\psi(x)$ are orthogonal to each other,
\begin{equation}
\int_{-\infty}^{\infty} \psi_{j,k}(x) \psi_{j',k'}(x) dx = 0,
\end{equation}
if $j \neq j'$ or $k \neq k'$. Furthermore, all combinations of translates and dilates of the wavelet are orthogonal
to all tranlates and dilates of the scaling function,
\begin{equation}
\int_{-\infty}^{\infty} \phi_{j,k}(x) \psi_{j',k'}(x) dx = 0,
\end{equation}
for all integers $j,j',k$, and $k'$.

\subsection*{Multiresolution Analysis}
The nestedness of the subspaces spanned by the different wavelet scales is a most useful property, allowing functions
to be approximated using as few basis functions as possible. Let $V_0$ be the space which is spanned by the basis of 
scaling functions $\phi_{0,k}(x)$ at the coarsest level of resolution. The space spanned by the wavelets at the same
level $j=0$ is $W_0$. Due to the aforementioned orthogonality between scaling functions and wavelets, the space
$W_0$ is orthogonal to $V_0$. The refined space $V_1$ can be described by
\begin{equation}
V_1 = V_0 \oplus W_0,
\label{V1}
\end{equation}
which represents all piecewise functions defined on half-intervals. Equation (\ref{V1}) indicates that the space spanned by 
$\phi(2x)$ can be described by the `summation' of the spaces spanned by
$V_0$ and $W_0$. One finer level, $V_2$, is spanned by translates of $\phi(4x)$. It can be written as
\begin{equation}
V_2 = V_1 \oplus W_1 = V_0 \oplus W_0 \oplus W_1.
\label{V2}
\end{equation}
Considering successively finer approximation spaces yields the spaces
\begin{equation}
V_j = V_0 \oplus W_0 \oplus W_1 \oplus \dots \oplus W_j.
\end{equation}
Thus the spaces $V_j$ are nested, since the function $\phi(x)$ is a combination of $\phi(2x)$ and $\phi(2x-1)$, and so on. 
In mathematical notation, $V_0 \subset V_1 \subset \dots \subset V_j$.

% Second-generation wavelets:
\section*{Second-Generation Wavelet Methods}
The wavelets previously described are only ever translates and dilates of a single mother wavelet function.
This family of wavelets is classified as first-generation, and they are tied to the use of regular grids. 
The lifting scheme developed by \cite{sweldens} abandoned the restriction to regular grids and the concept of translates
and dilates. These so-called second-generation wavelets are adaptive wavelets, allowing wavelets to be built with points
`missing' from the grid (an irregular grid).

% Present methods:
\subsection*{Adaptive Wavelet Collocation Method}
The adaptive wavelet collocation method \cite{vasilyev} has been successfully applied to parabolic, elliptic, and
hyperbolic PDEs in applications such as aeroacoustics, turbulence, flame interactions, and others.
The algorithm makes use of adaptive, second-generation wavelets to compress the solution.

\subsubsection*{Dyadic Grid}
The multi-resolution properties of wavelets previously described warrant a multilevel, or dyadic grid.
This grid should have as many levels as there are scales in the given problem. It can consist of 
either uniformly or non-uniformly spaced points. In the case of equally spaced grid points, 
let each grid level $j = 0, \dots, J$ be computed by 
\begin{align}
x^{j}_{k} &= 2^{-(j+\delta)} k,\text{ } \text{ } \text{ }  \text{ for $k=0,\dots,2^{j+\delta}$ },
\end{align}
where $\delta$ is some integer shifting parameter, allowing one to dictate that the coarsest level of resolution have 
smaller spacing than $1$ as in the case where $\delta=0$. From here forth we keep omit the parameter $\delta$, but keep
in mind that the number of grid points at each level may be shifted at will based on a fixed choice of $\delta$. 
The grid levels are formally defined by 
\begin{equation}
    \mathcal{G}^j= \{ x_{k}^{j} \in \Omega : k \in \mathcal{K}^j \}, \text{ } j \in \mathcal{Z},
\end{equation}
where $\mathcal{K}^{j}$ is the integer set representing the spatial locations in the grid at level $j$. The grids are 
nested, implying that $\mathcal{G}^{j} \subset \mathcal{G}^{j+1}$. In other words, the points $x^{j}$ are a perfect 
subset of the points $x^{j+1}$. This can be demonstrated by the relation that 
$x_{k}^{j}=x_{2k}^{j+1}$ for $k \in \mathcal{K}^{j}$.

\subsubsection*{Interpolating Subdivision Algorithm}
The interpolating subdivision scheme is central to the second-generation wavelet collocation approach. The scheme is used to
approximate values at odd points $x_{2k+1}^{j+1}$ by using $2N$ nearest points to construct interpolating polynomials of 
order $2N-1$. Lagrange polynomials are used, and the method can be used with a uniform grid or with 
nonuniform points such as Chebyshev points. The interpolating scheme is 
\begin{equation}
    f(x_{2k+1}^{j+1})=\sum_{l=-N+1}^{N} L_{k+l} f(x_{k+l}^{j}), \label{interp}
\end{equation}
where the interpolating weights are Lagrange polynomial coefficients given by 
\begin{equation}
    L_{k+l}(x)=\prod_{ \substack{ i=k-N+1 \\ i\neq k+l } }^{k+N} \frac{x-x_i}{x_{k+l}-x_i}.
\end{equation}
Given a uniform grid spacing at level $j$ of $h^j$, the accuracy of such an interpolation is of $\mathcal{O}((h^{j})^{2N})$.  

\subsubsection*{Wavelet Transform}
The existence of a fast wavelet transform is one of the attractive qualities of the method. The transform makes use of the 
interpolating subdivision algorithm (\ref{interp}) for the calculation of the scaling and detail wavelet coefficients. 
The forward wavelet transform is given by
\begin{equation}
	\begin{split}
		d_{k}^{j} &= \frac{1}{2} \left( c_{2k+1}^{j+1}-\sum_{l} w_{k,l}^{j} c_{2k+2l}^{j+1} \right), \\
		c_{k}^{j} &= c_{2k}^{j+1},
	\end{split}
\end{equation}
and the inverse transform is given by 
\begin{equation}
	\begin{split}
		c_{2k+1}^{j+1} &= 2 d_{k}^{j}  + \sum_{l} w_{k,l}^{j} c_{k+l}^{j}, \\
		c_{2k}^{j+1} &= c_{k}^{j}.
	\end{split}
\end{equation}

\subsubsection*{Scaling Functions}
The construction of second-generation interpolating wavelets makes use of the interpolating subdivision algorithm. 
The scheme is used to interpolate functional values defined at points on level $j$, to odd points (i.e. $x_{2k+1}^{j+1}$) 
at the next higher level of resolution. This scheme is used to construct the scaling and detail wavelet functions. 
Examples of the scaling and detail functions are shown in Figure 1 and Figure 2 respectively.
To obtain the scaling function $\phi_{m}^{j}(x)$, from (5) set $c_{k}^{j}=\delta_{k,m}, \forall k \in \mathcal{K}^j$, where $\delta_{k,m}$ is the Kronecker delta function defined by
\[ \delta_{k,m} = \begin{cases} 
      1 & k=m \\
      0 & k \neq m.
   \end{cases}
\]
Then let all $d_{l}^{j'}=0, \forall l \in \mathcal{L}^{j'}, \forall j' \geq j$ and perform the inverse transform 
up to an arbitrarily high level of resolution $J$. 

\subsubsection*{Wavelet Functions}
The wavelet $\psi_{l}^{j}$ is computed by setting $d_{m}^{j'} = \delta_{j',j} \delta_{l,m}, 
\forall l \in \mathcal{L}^{j}, \forall j \geq j$, 
and also $c_{k}^{j}, \forall k \in \mathcal{K}^j$. Then perform the inverse wavelet transform up to an 
arbitrarily high level of resolution $J$.
\begin{figure}
	\center
	% GNUPLOT: LaTeX picture
\setlength{\unitlength}{0.240900pt}
\ifx\plotpoint\undefined\newsavebox{\plotpoint}\fi
\sbox{\plotpoint}{\rule[-0.200pt]{0.400pt}{0.400pt}}%
\begin{picture}(1500,900)(0,0)
\sbox{\plotpoint}{\rule[-0.200pt]{0.400pt}{0.400pt}}%
\put(171.0,131.0){\rule[-0.200pt]{4.818pt}{0.400pt}}
\put(151,131){\makebox(0,0)[r]{$-0.2$}}
\put(1419.0,131.0){\rule[-0.200pt]{4.818pt}{0.400pt}}
\put(171.0,252.0){\rule[-0.200pt]{4.818pt}{0.400pt}}
\put(151,252){\makebox(0,0)[r]{$0$}}
\put(1419.0,252.0){\rule[-0.200pt]{4.818pt}{0.400pt}}
\put(171.0,374.0){\rule[-0.200pt]{4.818pt}{0.400pt}}
\put(151,374){\makebox(0,0)[r]{$0.2$}}
\put(1419.0,374.0){\rule[-0.200pt]{4.818pt}{0.400pt}}
\put(171.0,495.0){\rule[-0.200pt]{4.818pt}{0.400pt}}
\put(151,495){\makebox(0,0)[r]{$0.4$}}
\put(1419.0,495.0){\rule[-0.200pt]{4.818pt}{0.400pt}}
\put(171.0,616.0){\rule[-0.200pt]{4.818pt}{0.400pt}}
\put(151,616){\makebox(0,0)[r]{$0.6$}}
\put(1419.0,616.0){\rule[-0.200pt]{4.818pt}{0.400pt}}
\put(171.0,738.0){\rule[-0.200pt]{4.818pt}{0.400pt}}
\put(151,738){\makebox(0,0)[r]{$0.8$}}
\put(1419.0,738.0){\rule[-0.200pt]{4.818pt}{0.400pt}}
\put(171.0,859.0){\rule[-0.200pt]{4.818pt}{0.400pt}}
\put(151,859){\makebox(0,0)[r]{$1$}}
\put(1419.0,859.0){\rule[-0.200pt]{4.818pt}{0.400pt}}
\put(171.0,131.0){\rule[-0.200pt]{0.400pt}{4.818pt}}
\put(171,90){\makebox(0,0){$-4$}}
\put(171.0,839.0){\rule[-0.200pt]{0.400pt}{4.818pt}}
\put(330.0,131.0){\rule[-0.200pt]{0.400pt}{4.818pt}}
\put(330,90){\makebox(0,0){$-3$}}
\put(330.0,839.0){\rule[-0.200pt]{0.400pt}{4.818pt}}
\put(488.0,131.0){\rule[-0.200pt]{0.400pt}{4.818pt}}
\put(488,90){\makebox(0,0){$-2$}}
\put(488.0,839.0){\rule[-0.200pt]{0.400pt}{4.818pt}}
\put(647.0,131.0){\rule[-0.200pt]{0.400pt}{4.818pt}}
\put(647,90){\makebox(0,0){$-1$}}
\put(647.0,839.0){\rule[-0.200pt]{0.400pt}{4.818pt}}
\put(805.0,131.0){\rule[-0.200pt]{0.400pt}{4.818pt}}
\put(805,90){\makebox(0,0){$0$}}
\put(805.0,839.0){\rule[-0.200pt]{0.400pt}{4.818pt}}
\put(964.0,131.0){\rule[-0.200pt]{0.400pt}{4.818pt}}
\put(964,90){\makebox(0,0){$1$}}
\put(964.0,839.0){\rule[-0.200pt]{0.400pt}{4.818pt}}
\put(1122.0,131.0){\rule[-0.200pt]{0.400pt}{4.818pt}}
\put(1122,90){\makebox(0,0){$2$}}
\put(1122.0,839.0){\rule[-0.200pt]{0.400pt}{4.818pt}}
\put(1281.0,131.0){\rule[-0.200pt]{0.400pt}{4.818pt}}
\put(1281,90){\makebox(0,0){$3$}}
\put(1281.0,839.0){\rule[-0.200pt]{0.400pt}{4.818pt}}
\put(1439.0,131.0){\rule[-0.200pt]{0.400pt}{4.818pt}}
\put(1439,90){\makebox(0,0){$4$}}
\put(1439.0,839.0){\rule[-0.200pt]{0.400pt}{4.818pt}}
\put(171.0,131.0){\rule[-0.200pt]{0.400pt}{175.375pt}}
\put(171.0,131.0){\rule[-0.200pt]{305.461pt}{0.400pt}}
\put(1439.0,131.0){\rule[-0.200pt]{0.400pt}{175.375pt}}
\put(171.0,859.0){\rule[-0.200pt]{305.461pt}{0.400pt}}
\put(30,495){\makebox(0,0){$\phi(x)$}}
\put(805,29){\makebox(0,0){$x$}}
\put(171,252){\usebox{\plotpoint}}
\put(567,251.67){\rule{1.204pt}{0.400pt}}
\multiput(567.00,251.17)(2.500,1.000){2}{\rule{0.602pt}{0.400pt}}
\put(572,252.67){\rule{1.204pt}{0.400pt}}
\multiput(572.00,252.17)(2.500,1.000){2}{\rule{0.602pt}{0.400pt}}
\put(577,253.67){\rule{1.204pt}{0.400pt}}
\multiput(577.00,253.17)(2.500,1.000){2}{\rule{0.602pt}{0.400pt}}
\put(582,254.67){\rule{1.204pt}{0.400pt}}
\multiput(582.00,254.17)(2.500,1.000){2}{\rule{0.602pt}{0.400pt}}
\put(587,255.67){\rule{1.204pt}{0.400pt}}
\multiput(587.00,255.17)(2.500,1.000){2}{\rule{0.602pt}{0.400pt}}
\put(592,256.67){\rule{1.204pt}{0.400pt}}
\multiput(592.00,256.17)(2.500,1.000){2}{\rule{0.602pt}{0.400pt}}
\put(171.0,252.0){\rule[-0.200pt]{95.396pt}{0.400pt}}
\put(602,257.67){\rule{1.204pt}{0.400pt}}
\multiput(602.00,257.17)(2.500,1.000){2}{\rule{0.602pt}{0.400pt}}
\put(607,259.17){\rule{1.100pt}{0.400pt}}
\multiput(607.00,258.17)(2.717,2.000){2}{\rule{0.550pt}{0.400pt}}
\put(612,260.67){\rule{1.204pt}{0.400pt}}
\multiput(612.00,260.17)(2.500,1.000){2}{\rule{0.602pt}{0.400pt}}
\put(597.0,258.0){\rule[-0.200pt]{1.204pt}{0.400pt}}
\put(632,260.17){\rule{1.100pt}{0.400pt}}
\multiput(632.00,261.17)(2.717,-2.000){2}{\rule{0.550pt}{0.400pt}}
\multiput(637.00,258.95)(0.909,-0.447){3}{\rule{0.767pt}{0.108pt}}
\multiput(637.00,259.17)(3.409,-3.000){2}{\rule{0.383pt}{0.400pt}}
\multiput(642.00,255.93)(0.487,-0.477){7}{\rule{0.500pt}{0.115pt}}
\multiput(642.00,256.17)(3.962,-5.000){2}{\rule{0.250pt}{0.400pt}}
\multiput(647.60,249.09)(0.468,-0.774){5}{\rule{0.113pt}{0.700pt}}
\multiput(646.17,250.55)(4.000,-4.547){2}{\rule{0.400pt}{0.350pt}}
\multiput(651.59,242.93)(0.477,-0.821){7}{\rule{0.115pt}{0.740pt}}
\multiput(650.17,244.46)(5.000,-6.464){2}{\rule{0.400pt}{0.370pt}}
\multiput(656.59,234.93)(0.477,-0.821){7}{\rule{0.115pt}{0.740pt}}
\multiput(655.17,236.46)(5.000,-6.464){2}{\rule{0.400pt}{0.370pt}}
\multiput(661.59,226.60)(0.477,-0.933){7}{\rule{0.115pt}{0.820pt}}
\multiput(660.17,228.30)(5.000,-7.298){2}{\rule{0.400pt}{0.410pt}}
\multiput(666.59,217.93)(0.477,-0.821){7}{\rule{0.115pt}{0.740pt}}
\multiput(665.17,219.46)(5.000,-6.464){2}{\rule{0.400pt}{0.370pt}}
\multiput(671.59,210.26)(0.477,-0.710){7}{\rule{0.115pt}{0.660pt}}
\multiput(670.17,211.63)(5.000,-5.630){2}{\rule{0.400pt}{0.330pt}}
\multiput(676.59,203.26)(0.477,-0.710){7}{\rule{0.115pt}{0.660pt}}
\multiput(675.17,204.63)(5.000,-5.630){2}{\rule{0.400pt}{0.330pt}}
\multiput(681.59,196.59)(0.477,-0.599){7}{\rule{0.115pt}{0.580pt}}
\multiput(680.17,197.80)(5.000,-4.796){2}{\rule{0.400pt}{0.290pt}}
\multiput(686.00,191.94)(0.627,-0.468){5}{\rule{0.600pt}{0.113pt}}
\multiput(686.00,192.17)(3.755,-4.000){2}{\rule{0.300pt}{0.400pt}}
\multiput(691.00,187.95)(0.909,-0.447){3}{\rule{0.767pt}{0.108pt}}
\multiput(691.00,188.17)(3.409,-3.000){2}{\rule{0.383pt}{0.400pt}}
\put(617.0,262.0){\rule[-0.200pt]{3.613pt}{0.400pt}}
\multiput(701.00,186.60)(0.627,0.468){5}{\rule{0.600pt}{0.113pt}}
\multiput(701.00,185.17)(3.755,4.000){2}{\rule{0.300pt}{0.400pt}}
\multiput(706.59,190.00)(0.477,0.821){7}{\rule{0.115pt}{0.740pt}}
\multiput(705.17,190.00)(5.000,6.464){2}{\rule{0.400pt}{0.370pt}}
\multiput(711.59,198.00)(0.477,1.267){7}{\rule{0.115pt}{1.060pt}}
\multiput(710.17,198.00)(5.000,9.800){2}{\rule{0.400pt}{0.530pt}}
\multiput(716.59,210.00)(0.477,1.935){7}{\rule{0.115pt}{1.540pt}}
\multiput(715.17,210.00)(5.000,14.804){2}{\rule{0.400pt}{0.770pt}}
\multiput(721.59,228.00)(0.477,2.602){7}{\rule{0.115pt}{2.020pt}}
\multiput(720.17,228.00)(5.000,19.807){2}{\rule{0.400pt}{1.010pt}}
\multiput(726.59,252.00)(0.477,3.493){7}{\rule{0.115pt}{2.660pt}}
\multiput(725.17,252.00)(5.000,26.479){2}{\rule{0.400pt}{1.330pt}}
\multiput(731.59,284.00)(0.477,4.161){7}{\rule{0.115pt}{3.140pt}}
\multiput(730.17,284.00)(5.000,31.483){2}{\rule{0.400pt}{1.570pt}}
\multiput(736.59,322.00)(0.477,4.718){7}{\rule{0.115pt}{3.540pt}}
\multiput(735.17,322.00)(5.000,35.653){2}{\rule{0.400pt}{1.770pt}}
\multiput(741.59,365.00)(0.477,5.051){7}{\rule{0.115pt}{3.780pt}}
\multiput(740.17,365.00)(5.000,38.154){2}{\rule{0.400pt}{1.890pt}}
\multiput(746.59,411.00)(0.477,5.274){7}{\rule{0.115pt}{3.940pt}}
\multiput(745.17,411.00)(5.000,39.822){2}{\rule{0.400pt}{1.970pt}}
\multiput(751.60,459.00)(0.468,7.207){5}{\rule{0.113pt}{5.100pt}}
\multiput(750.17,459.00)(4.000,39.415){2}{\rule{0.400pt}{2.550pt}}
\multiput(755.59,509.00)(0.477,5.385){7}{\rule{0.115pt}{4.020pt}}
\multiput(754.17,509.00)(5.000,40.656){2}{\rule{0.400pt}{2.010pt}}
\multiput(760.59,558.00)(0.477,5.497){7}{\rule{0.115pt}{4.100pt}}
\multiput(759.17,558.00)(5.000,41.490){2}{\rule{0.400pt}{2.050pt}}
\multiput(765.59,608.00)(0.477,5.274){7}{\rule{0.115pt}{3.940pt}}
\multiput(764.17,608.00)(5.000,39.822){2}{\rule{0.400pt}{1.970pt}}
\multiput(770.59,656.00)(0.477,5.051){7}{\rule{0.115pt}{3.780pt}}
\multiput(769.17,656.00)(5.000,38.154){2}{\rule{0.400pt}{1.890pt}}
\multiput(775.59,702.00)(0.477,4.495){7}{\rule{0.115pt}{3.380pt}}
\multiput(774.17,702.00)(5.000,33.985){2}{\rule{0.400pt}{1.690pt}}
\multiput(780.59,743.00)(0.477,4.161){7}{\rule{0.115pt}{3.140pt}}
\multiput(779.17,743.00)(5.000,31.483){2}{\rule{0.400pt}{1.570pt}}
\multiput(785.59,781.00)(0.477,3.382){7}{\rule{0.115pt}{2.580pt}}
\multiput(784.17,781.00)(5.000,25.645){2}{\rule{0.400pt}{1.290pt}}
\multiput(790.59,812.00)(0.477,2.714){7}{\rule{0.115pt}{2.100pt}}
\multiput(789.17,812.00)(5.000,20.641){2}{\rule{0.400pt}{1.050pt}}
\multiput(795.59,837.00)(0.477,1.712){7}{\rule{0.115pt}{1.380pt}}
\multiput(794.17,837.00)(5.000,13.136){2}{\rule{0.400pt}{0.690pt}}
\multiput(800.59,853.00)(0.477,0.599){7}{\rule{0.115pt}{0.580pt}}
\multiput(799.17,853.00)(5.000,4.796){2}{\rule{0.400pt}{0.290pt}}
\multiput(805.59,856.59)(0.477,-0.599){7}{\rule{0.115pt}{0.580pt}}
\multiput(804.17,857.80)(5.000,-4.796){2}{\rule{0.400pt}{0.290pt}}
\multiput(810.59,847.27)(0.477,-1.712){7}{\rule{0.115pt}{1.380pt}}
\multiput(809.17,850.14)(5.000,-13.136){2}{\rule{0.400pt}{0.690pt}}
\multiput(815.59,828.28)(0.477,-2.714){7}{\rule{0.115pt}{2.100pt}}
\multiput(814.17,832.64)(5.000,-20.641){2}{\rule{0.400pt}{1.050pt}}
\multiput(820.59,801.29)(0.477,-3.382){7}{\rule{0.115pt}{2.580pt}}
\multiput(819.17,806.65)(5.000,-25.645){2}{\rule{0.400pt}{1.290pt}}
\multiput(825.59,767.97)(0.477,-4.161){7}{\rule{0.115pt}{3.140pt}}
\multiput(824.17,774.48)(5.000,-31.483){2}{\rule{0.400pt}{1.570pt}}
\multiput(830.59,728.97)(0.477,-4.495){7}{\rule{0.115pt}{3.380pt}}
\multiput(829.17,735.98)(5.000,-33.985){2}{\rule{0.400pt}{1.690pt}}
\multiput(835.59,686.31)(0.477,-5.051){7}{\rule{0.115pt}{3.780pt}}
\multiput(834.17,694.15)(5.000,-38.154){2}{\rule{0.400pt}{1.890pt}}
\multiput(840.59,639.64)(0.477,-5.274){7}{\rule{0.115pt}{3.940pt}}
\multiput(839.17,647.82)(5.000,-39.822){2}{\rule{0.400pt}{1.970pt}}
\multiput(845.59,590.98)(0.477,-5.497){7}{\rule{0.115pt}{4.100pt}}
\multiput(844.17,599.49)(5.000,-41.490){2}{\rule{0.400pt}{2.050pt}}
\multiput(850.59,541.31)(0.477,-5.385){7}{\rule{0.115pt}{4.020pt}}
\multiput(849.17,549.66)(5.000,-40.656){2}{\rule{0.400pt}{2.010pt}}
\multiput(855.60,487.83)(0.468,-7.207){5}{\rule{0.113pt}{5.100pt}}
\multiput(854.17,498.41)(4.000,-39.415){2}{\rule{0.400pt}{2.550pt}}
\multiput(859.59,442.64)(0.477,-5.274){7}{\rule{0.115pt}{3.940pt}}
\multiput(858.17,450.82)(5.000,-39.822){2}{\rule{0.400pt}{1.970pt}}
\multiput(864.59,395.31)(0.477,-5.051){7}{\rule{0.115pt}{3.780pt}}
\multiput(863.17,403.15)(5.000,-38.154){2}{\rule{0.400pt}{1.890pt}}
\multiput(869.59,350.31)(0.477,-4.718){7}{\rule{0.115pt}{3.540pt}}
\multiput(868.17,357.65)(5.000,-35.653){2}{\rule{0.400pt}{1.770pt}}
\multiput(874.59,308.97)(0.477,-4.161){7}{\rule{0.115pt}{3.140pt}}
\multiput(873.17,315.48)(5.000,-31.483){2}{\rule{0.400pt}{1.570pt}}
\multiput(879.59,272.96)(0.477,-3.493){7}{\rule{0.115pt}{2.660pt}}
\multiput(878.17,278.48)(5.000,-26.479){2}{\rule{0.400pt}{1.330pt}}
\multiput(884.59,243.61)(0.477,-2.602){7}{\rule{0.115pt}{2.020pt}}
\multiput(883.17,247.81)(5.000,-19.807){2}{\rule{0.400pt}{1.010pt}}
\multiput(889.59,221.61)(0.477,-1.935){7}{\rule{0.115pt}{1.540pt}}
\multiput(888.17,224.80)(5.000,-14.804){2}{\rule{0.400pt}{0.770pt}}
\multiput(894.59,205.60)(0.477,-1.267){7}{\rule{0.115pt}{1.060pt}}
\multiput(893.17,207.80)(5.000,-9.800){2}{\rule{0.400pt}{0.530pt}}
\multiput(899.59,194.93)(0.477,-0.821){7}{\rule{0.115pt}{0.740pt}}
\multiput(898.17,196.46)(5.000,-6.464){2}{\rule{0.400pt}{0.370pt}}
\multiput(904.00,188.94)(0.627,-0.468){5}{\rule{0.600pt}{0.113pt}}
\multiput(904.00,189.17)(3.755,-4.000){2}{\rule{0.300pt}{0.400pt}}
\put(696.0,186.0){\rule[-0.200pt]{1.204pt}{0.400pt}}
\multiput(914.00,186.61)(0.909,0.447){3}{\rule{0.767pt}{0.108pt}}
\multiput(914.00,185.17)(3.409,3.000){2}{\rule{0.383pt}{0.400pt}}
\multiput(919.00,189.60)(0.627,0.468){5}{\rule{0.600pt}{0.113pt}}
\multiput(919.00,188.17)(3.755,4.000){2}{\rule{0.300pt}{0.400pt}}
\multiput(924.59,193.00)(0.477,0.599){7}{\rule{0.115pt}{0.580pt}}
\multiput(923.17,193.00)(5.000,4.796){2}{\rule{0.400pt}{0.290pt}}
\multiput(929.59,199.00)(0.477,0.710){7}{\rule{0.115pt}{0.660pt}}
\multiput(928.17,199.00)(5.000,5.630){2}{\rule{0.400pt}{0.330pt}}
\multiput(934.59,206.00)(0.477,0.710){7}{\rule{0.115pt}{0.660pt}}
\multiput(933.17,206.00)(5.000,5.630){2}{\rule{0.400pt}{0.330pt}}
\multiput(939.59,213.00)(0.477,0.821){7}{\rule{0.115pt}{0.740pt}}
\multiput(938.17,213.00)(5.000,6.464){2}{\rule{0.400pt}{0.370pt}}
\multiput(944.59,221.00)(0.477,0.933){7}{\rule{0.115pt}{0.820pt}}
\multiput(943.17,221.00)(5.000,7.298){2}{\rule{0.400pt}{0.410pt}}
\multiput(949.59,230.00)(0.477,0.821){7}{\rule{0.115pt}{0.740pt}}
\multiput(948.17,230.00)(5.000,6.464){2}{\rule{0.400pt}{0.370pt}}
\multiput(954.59,238.00)(0.477,0.821){7}{\rule{0.115pt}{0.740pt}}
\multiput(953.17,238.00)(5.000,6.464){2}{\rule{0.400pt}{0.370pt}}
\multiput(959.59,246.00)(0.477,0.599){7}{\rule{0.115pt}{0.580pt}}
\multiput(958.17,246.00)(5.000,4.796){2}{\rule{0.400pt}{0.290pt}}
\multiput(964.60,252.00)(0.468,0.627){5}{\rule{0.113pt}{0.600pt}}
\multiput(963.17,252.00)(4.000,3.755){2}{\rule{0.400pt}{0.300pt}}
\multiput(968.00,257.61)(0.909,0.447){3}{\rule{0.767pt}{0.108pt}}
\multiput(968.00,256.17)(3.409,3.000){2}{\rule{0.383pt}{0.400pt}}
\put(973,260.17){\rule{1.100pt}{0.400pt}}
\multiput(973.00,259.17)(2.717,2.000){2}{\rule{0.550pt}{0.400pt}}
\put(909.0,186.0){\rule[-0.200pt]{1.204pt}{0.400pt}}
\put(993,260.67){\rule{1.204pt}{0.400pt}}
\multiput(993.00,261.17)(2.500,-1.000){2}{\rule{0.602pt}{0.400pt}}
\put(998,259.17){\rule{1.100pt}{0.400pt}}
\multiput(998.00,260.17)(2.717,-2.000){2}{\rule{0.550pt}{0.400pt}}
\put(1003,257.67){\rule{1.204pt}{0.400pt}}
\multiput(1003.00,258.17)(2.500,-1.000){2}{\rule{0.602pt}{0.400pt}}
\put(978.0,262.0){\rule[-0.200pt]{3.613pt}{0.400pt}}
\put(1013,256.67){\rule{1.204pt}{0.400pt}}
\multiput(1013.00,257.17)(2.500,-1.000){2}{\rule{0.602pt}{0.400pt}}
\put(1018,255.67){\rule{1.204pt}{0.400pt}}
\multiput(1018.00,256.17)(2.500,-1.000){2}{\rule{0.602pt}{0.400pt}}
\put(1023,254.67){\rule{1.204pt}{0.400pt}}
\multiput(1023.00,255.17)(2.500,-1.000){2}{\rule{0.602pt}{0.400pt}}
\put(1028,253.67){\rule{1.204pt}{0.400pt}}
\multiput(1028.00,254.17)(2.500,-1.000){2}{\rule{0.602pt}{0.400pt}}
\put(1033,252.67){\rule{1.204pt}{0.400pt}}
\multiput(1033.00,253.17)(2.500,-1.000){2}{\rule{0.602pt}{0.400pt}}
\put(1038,251.67){\rule{1.204pt}{0.400pt}}
\multiput(1038.00,252.17)(2.500,-1.000){2}{\rule{0.602pt}{0.400pt}}
\put(1008.0,258.0){\rule[-0.200pt]{1.204pt}{0.400pt}}
\put(1043.0,252.0){\rule[-0.200pt]{95.396pt}{0.400pt}}
\put(171.0,131.0){\rule[-0.200pt]{0.400pt}{175.375pt}}
\put(171.0,131.0){\rule[-0.200pt]{305.461pt}{0.400pt}}
\put(1439.0,131.0){\rule[-0.200pt]{0.400pt}{175.375pt}}
\put(171.0,859.0){\rule[-0.200pt]{305.461pt}{0.400pt}}
\end{picture}

	\caption{An example of a scaling function, $\phi(x)$, for $N=3$.}
\end{figure}
\begin{figure}
	\center
	% GNUPLOT: LaTeX picture
\setlength{\unitlength}{0.240900pt}
\ifx\plotpoint\undefined\newsavebox{\plotpoint}\fi
\sbox{\plotpoint}{\rule[-0.200pt]{0.400pt}{0.400pt}}%
\begin{picture}(1500,900)(0,0)
\sbox{\plotpoint}{\rule[-0.200pt]{0.400pt}{0.400pt}}%
\put(171.0,131.0){\rule[-0.200pt]{4.818pt}{0.400pt}}
\put(151,131){\makebox(0,0)[r]{$-0.5$}}
\put(1419.0,131.0){\rule[-0.200pt]{4.818pt}{0.400pt}}
\put(171.0,277.0){\rule[-0.200pt]{4.818pt}{0.400pt}}
\put(151,277){\makebox(0,0)[r]{$0$}}
\put(1419.0,277.0){\rule[-0.200pt]{4.818pt}{0.400pt}}
\put(171.0,422.0){\rule[-0.200pt]{4.818pt}{0.400pt}}
\put(151,422){\makebox(0,0)[r]{$0.5$}}
\put(1419.0,422.0){\rule[-0.200pt]{4.818pt}{0.400pt}}
\put(171.0,568.0){\rule[-0.200pt]{4.818pt}{0.400pt}}
\put(151,568){\makebox(0,0)[r]{$1$}}
\put(1419.0,568.0){\rule[-0.200pt]{4.818pt}{0.400pt}}
\put(171.0,713.0){\rule[-0.200pt]{4.818pt}{0.400pt}}
\put(151,713){\makebox(0,0)[r]{$1.5$}}
\put(1419.0,713.0){\rule[-0.200pt]{4.818pt}{0.400pt}}
\put(171.0,859.0){\rule[-0.200pt]{4.818pt}{0.400pt}}
\put(151,859){\makebox(0,0)[r]{$2$}}
\put(1419.0,859.0){\rule[-0.200pt]{4.818pt}{0.400pt}}
\put(171.0,131.0){\rule[-0.200pt]{0.400pt}{4.818pt}}
\put(171,90){\makebox(0,0){$-4$}}
\put(171.0,839.0){\rule[-0.200pt]{0.400pt}{4.818pt}}
\put(330.0,131.0){\rule[-0.200pt]{0.400pt}{4.818pt}}
\put(330,90){\makebox(0,0){$-3$}}
\put(330.0,839.0){\rule[-0.200pt]{0.400pt}{4.818pt}}
\put(488.0,131.0){\rule[-0.200pt]{0.400pt}{4.818pt}}
\put(488,90){\makebox(0,0){$-2$}}
\put(488.0,839.0){\rule[-0.200pt]{0.400pt}{4.818pt}}
\put(647.0,131.0){\rule[-0.200pt]{0.400pt}{4.818pt}}
\put(647,90){\makebox(0,0){$-1$}}
\put(647.0,839.0){\rule[-0.200pt]{0.400pt}{4.818pt}}
\put(805.0,131.0){\rule[-0.200pt]{0.400pt}{4.818pt}}
\put(805,90){\makebox(0,0){$0$}}
\put(805.0,839.0){\rule[-0.200pt]{0.400pt}{4.818pt}}
\put(964.0,131.0){\rule[-0.200pt]{0.400pt}{4.818pt}}
\put(964,90){\makebox(0,0){$1$}}
\put(964.0,839.0){\rule[-0.200pt]{0.400pt}{4.818pt}}
\put(1122.0,131.0){\rule[-0.200pt]{0.400pt}{4.818pt}}
\put(1122,90){\makebox(0,0){$2$}}
\put(1122.0,839.0){\rule[-0.200pt]{0.400pt}{4.818pt}}
\put(1281.0,131.0){\rule[-0.200pt]{0.400pt}{4.818pt}}
\put(1281,90){\makebox(0,0){$3$}}
\put(1281.0,839.0){\rule[-0.200pt]{0.400pt}{4.818pt}}
\put(1439.0,131.0){\rule[-0.200pt]{0.400pt}{4.818pt}}
\put(1439,90){\makebox(0,0){$4$}}
\put(1439.0,839.0){\rule[-0.200pt]{0.400pt}{4.818pt}}
\put(171.0,131.0){\rule[-0.200pt]{0.400pt}{175.375pt}}
\put(171.0,131.0){\rule[-0.200pt]{305.461pt}{0.400pt}}
\put(1439.0,131.0){\rule[-0.200pt]{0.400pt}{175.375pt}}
\put(171.0,859.0){\rule[-0.200pt]{305.461pt}{0.400pt}}
\put(30,495){\makebox(0,0){$\psi(x)$}}
\put(805,29){\makebox(0,0){$x$}}
\put(171,277){\usebox{\plotpoint}}
\put(706,275.67){\rule{1.204pt}{0.400pt}}
\multiput(706.00,276.17)(2.500,-1.000){2}{\rule{0.602pt}{0.400pt}}
\put(171.0,277.0){\rule[-0.200pt]{128.881pt}{0.400pt}}
\put(721,275.67){\rule{1.204pt}{0.400pt}}
\multiput(721.00,275.17)(2.500,1.000){2}{\rule{0.602pt}{0.400pt}}
\put(726,276.67){\rule{1.204pt}{0.400pt}}
\multiput(726.00,276.17)(2.500,1.000){2}{\rule{0.602pt}{0.400pt}}
\put(731,278.17){\rule{1.100pt}{0.400pt}}
\multiput(731.00,277.17)(2.717,2.000){2}{\rule{0.550pt}{0.400pt}}
\put(736,280.17){\rule{1.100pt}{0.400pt}}
\multiput(736.00,279.17)(2.717,2.000){2}{\rule{0.550pt}{0.400pt}}
\put(741,281.67){\rule{1.204pt}{0.400pt}}
\multiput(741.00,281.17)(2.500,1.000){2}{\rule{0.602pt}{0.400pt}}
\multiput(746.00,283.61)(0.909,0.447){3}{\rule{0.767pt}{0.108pt}}
\multiput(746.00,282.17)(3.409,3.000){2}{\rule{0.383pt}{0.400pt}}
\put(711.0,276.0){\rule[-0.200pt]{2.409pt}{0.400pt}}
\put(755,284.17){\rule{1.100pt}{0.400pt}}
\multiput(755.00,285.17)(2.717,-2.000){2}{\rule{0.550pt}{0.400pt}}
\multiput(760.59,281.26)(0.477,-0.710){7}{\rule{0.115pt}{0.660pt}}
\multiput(759.17,282.63)(5.000,-5.630){2}{\rule{0.400pt}{0.330pt}}
\multiput(765.59,271.94)(0.477,-1.489){7}{\rule{0.115pt}{1.220pt}}
\multiput(764.17,274.47)(5.000,-11.468){2}{\rule{0.400pt}{0.610pt}}
\multiput(770.59,257.27)(0.477,-1.712){7}{\rule{0.115pt}{1.380pt}}
\multiput(769.17,260.14)(5.000,-13.136){2}{\rule{0.400pt}{0.690pt}}
\multiput(775.59,241.60)(0.477,-1.601){7}{\rule{0.115pt}{1.300pt}}
\multiput(774.17,244.30)(5.000,-12.302){2}{\rule{0.400pt}{0.650pt}}
\multiput(780.59,227.60)(0.477,-1.267){7}{\rule{0.115pt}{1.060pt}}
\multiput(779.17,229.80)(5.000,-9.800){2}{\rule{0.400pt}{0.530pt}}
\multiput(785.59,217.26)(0.477,-0.710){7}{\rule{0.115pt}{0.660pt}}
\multiput(784.17,218.63)(5.000,-5.630){2}{\rule{0.400pt}{0.330pt}}
\multiput(790.00,213.60)(0.627,0.468){5}{\rule{0.600pt}{0.113pt}}
\multiput(790.00,212.17)(3.755,4.000){2}{\rule{0.300pt}{0.400pt}}
\multiput(795.59,217.00)(0.477,2.046){7}{\rule{0.115pt}{1.620pt}}
\multiput(794.17,217.00)(5.000,15.638){2}{\rule{0.400pt}{0.810pt}}
\multiput(800.59,236.00)(0.477,4.495){7}{\rule{0.115pt}{3.380pt}}
\multiput(799.17,236.00)(5.000,33.985){2}{\rule{0.400pt}{1.690pt}}
\multiput(805.59,277.00)(0.477,7.389){7}{\rule{0.115pt}{5.460pt}}
\multiput(804.17,277.00)(5.000,55.667){2}{\rule{0.400pt}{2.730pt}}
\multiput(810.59,344.00)(0.477,9.393){7}{\rule{0.115pt}{6.900pt}}
\multiput(809.17,344.00)(5.000,70.679){2}{\rule{0.400pt}{3.450pt}}
\multiput(815.59,429.00)(0.477,10.395){7}{\rule{0.115pt}{7.620pt}}
\multiput(814.17,429.00)(5.000,78.184){2}{\rule{0.400pt}{3.810pt}}
\multiput(820.59,523.00)(0.477,10.506){7}{\rule{0.115pt}{7.700pt}}
\multiput(819.17,523.00)(5.000,79.018){2}{\rule{0.400pt}{3.850pt}}
\multiput(825.59,618.00)(0.477,9.949){7}{\rule{0.115pt}{7.300pt}}
\multiput(824.17,618.00)(5.000,74.848){2}{\rule{0.400pt}{3.650pt}}
\multiput(830.59,708.00)(0.477,8.391){7}{\rule{0.115pt}{6.180pt}}
\multiput(829.17,708.00)(5.000,63.173){2}{\rule{0.400pt}{3.090pt}}
\multiput(835.59,784.00)(0.477,5.942){7}{\rule{0.115pt}{4.420pt}}
\multiput(834.17,784.00)(5.000,44.826){2}{\rule{0.400pt}{2.210pt}}
\multiput(840.59,838.00)(0.477,2.269){7}{\rule{0.115pt}{1.780pt}}
\multiput(839.17,838.00)(5.000,17.306){2}{\rule{0.400pt}{0.890pt}}
\multiput(845.59,851.61)(0.477,-2.269){7}{\rule{0.115pt}{1.780pt}}
\multiput(844.17,855.31)(5.000,-17.306){2}{\rule{0.400pt}{0.890pt}}
\multiput(850.59,819.65)(0.477,-5.942){7}{\rule{0.115pt}{4.420pt}}
\multiput(849.17,828.83)(5.000,-44.826){2}{\rule{0.400pt}{2.210pt}}
\multiput(855.60,752.04)(0.468,-11.009){5}{\rule{0.113pt}{7.700pt}}
\multiput(854.17,768.02)(4.000,-60.018){2}{\rule{0.400pt}{3.850pt}}
\multiput(859.59,677.70)(0.477,-9.949){7}{\rule{0.115pt}{7.300pt}}
\multiput(858.17,692.85)(5.000,-74.848){2}{\rule{0.400pt}{3.650pt}}
\multiput(864.59,586.04)(0.477,-10.506){7}{\rule{0.115pt}{7.700pt}}
\multiput(863.17,602.02)(5.000,-79.018){2}{\rule{0.400pt}{3.850pt}}
\multiput(869.59,491.37)(0.477,-10.395){7}{\rule{0.115pt}{7.620pt}}
\multiput(868.17,507.18)(5.000,-78.184){2}{\rule{0.400pt}{3.810pt}}
\multiput(874.59,400.36)(0.477,-9.393){7}{\rule{0.115pt}{6.900pt}}
\multiput(873.17,414.68)(5.000,-70.679){2}{\rule{0.400pt}{3.450pt}}
\multiput(879.59,321.33)(0.477,-7.389){7}{\rule{0.115pt}{5.460pt}}
\multiput(878.17,332.67)(5.000,-55.667){2}{\rule{0.400pt}{2.730pt}}
\multiput(884.59,262.97)(0.477,-4.495){7}{\rule{0.115pt}{3.380pt}}
\multiput(883.17,269.98)(5.000,-33.985){2}{\rule{0.400pt}{1.690pt}}
\multiput(889.59,229.28)(0.477,-2.046){7}{\rule{0.115pt}{1.620pt}}
\multiput(888.17,232.64)(5.000,-15.638){2}{\rule{0.400pt}{0.810pt}}
\multiput(894.00,215.94)(0.627,-0.468){5}{\rule{0.600pt}{0.113pt}}
\multiput(894.00,216.17)(3.755,-4.000){2}{\rule{0.300pt}{0.400pt}}
\multiput(899.59,213.00)(0.477,0.710){7}{\rule{0.115pt}{0.660pt}}
\multiput(898.17,213.00)(5.000,5.630){2}{\rule{0.400pt}{0.330pt}}
\multiput(904.59,220.00)(0.477,1.267){7}{\rule{0.115pt}{1.060pt}}
\multiput(903.17,220.00)(5.000,9.800){2}{\rule{0.400pt}{0.530pt}}
\multiput(909.59,232.00)(0.477,1.601){7}{\rule{0.115pt}{1.300pt}}
\multiput(908.17,232.00)(5.000,12.302){2}{\rule{0.400pt}{0.650pt}}
\multiput(914.59,247.00)(0.477,1.712){7}{\rule{0.115pt}{1.380pt}}
\multiput(913.17,247.00)(5.000,13.136){2}{\rule{0.400pt}{0.690pt}}
\multiput(919.59,263.00)(0.477,1.489){7}{\rule{0.115pt}{1.220pt}}
\multiput(918.17,263.00)(5.000,11.468){2}{\rule{0.400pt}{0.610pt}}
\multiput(924.59,277.00)(0.477,0.710){7}{\rule{0.115pt}{0.660pt}}
\multiput(923.17,277.00)(5.000,5.630){2}{\rule{0.400pt}{0.330pt}}
\put(929,284.17){\rule{1.100pt}{0.400pt}}
\multiput(929.00,283.17)(2.717,2.000){2}{\rule{0.550pt}{0.400pt}}
\put(751.0,286.0){\rule[-0.200pt]{0.964pt}{0.400pt}}
\multiput(939.00,284.95)(0.909,-0.447){3}{\rule{0.767pt}{0.108pt}}
\multiput(939.00,285.17)(3.409,-3.000){2}{\rule{0.383pt}{0.400pt}}
\put(944,281.67){\rule{1.204pt}{0.400pt}}
\multiput(944.00,282.17)(2.500,-1.000){2}{\rule{0.602pt}{0.400pt}}
\put(949,280.17){\rule{1.100pt}{0.400pt}}
\multiput(949.00,281.17)(2.717,-2.000){2}{\rule{0.550pt}{0.400pt}}
\put(954,278.17){\rule{1.100pt}{0.400pt}}
\multiput(954.00,279.17)(2.717,-2.000){2}{\rule{0.550pt}{0.400pt}}
\put(959,276.67){\rule{1.204pt}{0.400pt}}
\multiput(959.00,277.17)(2.500,-1.000){2}{\rule{0.602pt}{0.400pt}}
\put(964,275.67){\rule{0.964pt}{0.400pt}}
\multiput(964.00,276.17)(2.000,-1.000){2}{\rule{0.482pt}{0.400pt}}
\put(934.0,286.0){\rule[-0.200pt]{1.204pt}{0.400pt}}
\put(978,275.67){\rule{1.204pt}{0.400pt}}
\multiput(978.00,275.17)(2.500,1.000){2}{\rule{0.602pt}{0.400pt}}
\put(968.0,276.0){\rule[-0.200pt]{2.409pt}{0.400pt}}
\put(983.0,277.0){\rule[-0.200pt]{109.850pt}{0.400pt}}
\put(171.0,131.0){\rule[-0.200pt]{0.400pt}{175.375pt}}
\put(171.0,131.0){\rule[-0.200pt]{305.461pt}{0.400pt}}
\put(1439.0,131.0){\rule[-0.200pt]{0.400pt}{175.375pt}}
\put(171.0,859.0){\rule[-0.200pt]{305.461pt}{0.400pt}}
\end{picture}

	\caption{An example of a wavelet, $\psi(x)$, for $N=3$.}
\end{figure}

\subsubsection*{Approximation of Functions}
The approximation of a function $f(x)$ is done by setting the scaling coefficients at the arbitrary maximum level of 
resolution $J$ to the function itself. Once the function is sampled this way for all $c_{k}^{J}$, the forward wavelet 
transform is performed down to the coarsest level of resolution. The function is then represented by 
\begin{equation}
        f^J(x)=\sum_{k \in \mathcal{K}^0} c_{k}^{0} \phi_{k}^{0}(x) + \sum_{j=0}^{J-1} \sum_{l \in \mathcal{L}^j}
                d_{l}^{j} \psi_{l}^{j}(x).
\end{equation}
Often, a large number of wavelet coefficients can be discarded, and the approximation (6) is still adequate. Define 
some threshold $\epsilon$ for the coefficients, then keep only those coefficients which satisfy 
$|d_{l}^{j}| \geq \epsilon$. The approximation (6) becomes 
\begin{equation}
        f_{\geq}^{J}(x)=\sum_{k \in \mathcal{K}^0} c_{k}^{0} \phi_{k}^{0}(x) + \sum_{j=0}^{J-1} 
        \sum_{ \substack{ l \in \mathcal{L}^j \\ |d_{l}^{j}| \geq \epsilon} } d_{l}^{j} \psi_{l}^{j}(x).
\end{equation}
In the figures below are approximations to the function $f(x)=\cos{(80 \pi x)} \exp{(-64 x^2)}$ with an increasing threshold 
$\epsilon$.
\begin{figure}
	\center
	% GNUPLOT: LaTeX picture
\setlength{\unitlength}{0.240900pt}
\ifx\plotpoint\undefined\newsavebox{\plotpoint}\fi
\sbox{\plotpoint}{\rule[-0.200pt]{0.400pt}{0.400pt}}%
\begin{picture}(1500,900)(0,0)
\sbox{\plotpoint}{\rule[-0.200pt]{0.400pt}{0.400pt}}%
\put(171.0,131.0){\rule[-0.200pt]{4.818pt}{0.400pt}}
\put(151,131){\makebox(0,0)[r]{$-1$}}
\put(1419.0,131.0){\rule[-0.200pt]{4.818pt}{0.400pt}}
\put(171.0,204.0){\rule[-0.200pt]{4.818pt}{0.400pt}}
\put(151,204){\makebox(0,0)[r]{$-0.8$}}
\put(1419.0,204.0){\rule[-0.200pt]{4.818pt}{0.400pt}}
\put(171.0,277.0){\rule[-0.200pt]{4.818pt}{0.400pt}}
\put(151,277){\makebox(0,0)[r]{$-0.6$}}
\put(1419.0,277.0){\rule[-0.200pt]{4.818pt}{0.400pt}}
\put(171.0,349.0){\rule[-0.200pt]{4.818pt}{0.400pt}}
\put(151,349){\makebox(0,0)[r]{$-0.4$}}
\put(1419.0,349.0){\rule[-0.200pt]{4.818pt}{0.400pt}}
\put(171.0,422.0){\rule[-0.200pt]{4.818pt}{0.400pt}}
\put(151,422){\makebox(0,0)[r]{$-0.2$}}
\put(1419.0,422.0){\rule[-0.200pt]{4.818pt}{0.400pt}}
\put(171.0,495.0){\rule[-0.200pt]{4.818pt}{0.400pt}}
\put(151,495){\makebox(0,0)[r]{$0$}}
\put(1419.0,495.0){\rule[-0.200pt]{4.818pt}{0.400pt}}
\put(171.0,568.0){\rule[-0.200pt]{4.818pt}{0.400pt}}
\put(151,568){\makebox(0,0)[r]{$0.2$}}
\put(1419.0,568.0){\rule[-0.200pt]{4.818pt}{0.400pt}}
\put(171.0,641.0){\rule[-0.200pt]{4.818pt}{0.400pt}}
\put(151,641){\makebox(0,0)[r]{$0.4$}}
\put(1419.0,641.0){\rule[-0.200pt]{4.818pt}{0.400pt}}
\put(171.0,713.0){\rule[-0.200pt]{4.818pt}{0.400pt}}
\put(151,713){\makebox(0,0)[r]{$0.6$}}
\put(1419.0,713.0){\rule[-0.200pt]{4.818pt}{0.400pt}}
\put(171.0,786.0){\rule[-0.200pt]{4.818pt}{0.400pt}}
\put(151,786){\makebox(0,0)[r]{$0.8$}}
\put(1419.0,786.0){\rule[-0.200pt]{4.818pt}{0.400pt}}
\put(171.0,859.0){\rule[-0.200pt]{4.818pt}{0.400pt}}
\put(151,859){\makebox(0,0)[r]{$1$}}
\put(1419.0,859.0){\rule[-0.200pt]{4.818pt}{0.400pt}}
\put(171.0,131.0){\rule[-0.200pt]{0.400pt}{4.818pt}}
\put(171,90){\makebox(0,0){$-0.6$}}
\put(171.0,839.0){\rule[-0.200pt]{0.400pt}{4.818pt}}
\put(382.0,131.0){\rule[-0.200pt]{0.400pt}{4.818pt}}
\put(382,90){\makebox(0,0){$-0.4$}}
\put(382.0,839.0){\rule[-0.200pt]{0.400pt}{4.818pt}}
\put(594.0,131.0){\rule[-0.200pt]{0.400pt}{4.818pt}}
\put(594,90){\makebox(0,0){$-0.2$}}
\put(594.0,839.0){\rule[-0.200pt]{0.400pt}{4.818pt}}
\put(805.0,131.0){\rule[-0.200pt]{0.400pt}{4.818pt}}
\put(805,90){\makebox(0,0){$0$}}
\put(805.0,839.0){\rule[-0.200pt]{0.400pt}{4.818pt}}
\put(1016.0,131.0){\rule[-0.200pt]{0.400pt}{4.818pt}}
\put(1016,90){\makebox(0,0){$0.2$}}
\put(1016.0,839.0){\rule[-0.200pt]{0.400pt}{4.818pt}}
\put(1228.0,131.0){\rule[-0.200pt]{0.400pt}{4.818pt}}
\put(1228,90){\makebox(0,0){$0.4$}}
\put(1228.0,839.0){\rule[-0.200pt]{0.400pt}{4.818pt}}
\put(1439.0,131.0){\rule[-0.200pt]{0.400pt}{4.818pt}}
\put(1439,90){\makebox(0,0){$0.6$}}
\put(1439.0,839.0){\rule[-0.200pt]{0.400pt}{4.818pt}}
\put(171.0,131.0){\rule[-0.200pt]{0.400pt}{175.375pt}}
\put(171.0,131.0){\rule[-0.200pt]{305.461pt}{0.400pt}}
\put(1439.0,131.0){\rule[-0.200pt]{0.400pt}{175.375pt}}
\put(171.0,859.0){\rule[-0.200pt]{305.461pt}{0.400pt}}
\put(30,495){\makebox(0,0){$f(x)$}}
\put(805,29){\makebox(0,0){$x$}}
\put(277,495){\usebox{\plotpoint}}
\put(471,493.67){\rule{0.482pt}{0.400pt}}
\multiput(471.00,494.17)(1.000,-1.000){2}{\rule{0.241pt}{0.400pt}}
\put(277.0,495.0){\rule[-0.200pt]{46.735pt}{0.400pt}}
\put(477,493.67){\rule{0.482pt}{0.400pt}}
\multiput(477.00,493.17)(1.000,1.000){2}{\rule{0.241pt}{0.400pt}}
\put(473.0,494.0){\rule[-0.200pt]{0.964pt}{0.400pt}}
\put(483,494.67){\rule{0.482pt}{0.400pt}}
\multiput(483.00,494.17)(1.000,1.000){2}{\rule{0.241pt}{0.400pt}}
\put(479.0,495.0){\rule[-0.200pt]{0.964pt}{0.400pt}}
\put(491,494.67){\rule{0.482pt}{0.400pt}}
\multiput(491.00,495.17)(1.000,-1.000){2}{\rule{0.241pt}{0.400pt}}
\put(485.0,496.0){\rule[-0.200pt]{1.445pt}{0.400pt}}
\put(495,493.67){\rule{0.482pt}{0.400pt}}
\multiput(495.00,494.17)(1.000,-1.000){2}{\rule{0.241pt}{0.400pt}}
\put(497,492.67){\rule{0.723pt}{0.400pt}}
\multiput(497.00,493.17)(1.500,-1.000){2}{\rule{0.361pt}{0.400pt}}
\put(493.0,495.0){\rule[-0.200pt]{0.482pt}{0.400pt}}
\put(504,492.67){\rule{0.482pt}{0.400pt}}
\multiput(504.00,492.17)(1.000,1.000){2}{\rule{0.241pt}{0.400pt}}
\put(506,493.67){\rule{0.482pt}{0.400pt}}
\multiput(506.00,493.17)(1.000,1.000){2}{\rule{0.241pt}{0.400pt}}
\put(508,494.67){\rule{0.482pt}{0.400pt}}
\multiput(508.00,494.17)(1.000,1.000){2}{\rule{0.241pt}{0.400pt}}
\put(510,495.67){\rule{0.482pt}{0.400pt}}
\multiput(510.00,495.17)(1.000,1.000){2}{\rule{0.241pt}{0.400pt}}
\put(512,496.67){\rule{0.482pt}{0.400pt}}
\multiput(512.00,496.17)(1.000,1.000){2}{\rule{0.241pt}{0.400pt}}
\put(500.0,493.0){\rule[-0.200pt]{0.964pt}{0.400pt}}
\put(516,496.67){\rule{0.482pt}{0.400pt}}
\multiput(516.00,497.17)(1.000,-1.000){2}{\rule{0.241pt}{0.400pt}}
\put(518,495.67){\rule{0.482pt}{0.400pt}}
\multiput(518.00,496.17)(1.000,-1.000){2}{\rule{0.241pt}{0.400pt}}
\put(520,494.17){\rule{0.482pt}{0.400pt}}
\multiput(520.00,495.17)(1.000,-2.000){2}{\rule{0.241pt}{0.400pt}}
\put(522,492.17){\rule{0.482pt}{0.400pt}}
\multiput(522.00,493.17)(1.000,-2.000){2}{\rule{0.241pt}{0.400pt}}
\put(524,490.67){\rule{0.482pt}{0.400pt}}
\multiput(524.00,491.17)(1.000,-1.000){2}{\rule{0.241pt}{0.400pt}}
\put(514.0,498.0){\rule[-0.200pt]{0.482pt}{0.400pt}}
\put(531,491.17){\rule{0.482pt}{0.400pt}}
\multiput(531.00,490.17)(1.000,2.000){2}{\rule{0.241pt}{0.400pt}}
\put(533.17,493){\rule{0.400pt}{0.700pt}}
\multiput(532.17,493.00)(2.000,1.547){2}{\rule{0.400pt}{0.350pt}}
\put(535,496.17){\rule{0.482pt}{0.400pt}}
\multiput(535.00,495.17)(1.000,2.000){2}{\rule{0.241pt}{0.400pt}}
\put(537.17,498){\rule{0.400pt}{0.700pt}}
\multiput(536.17,498.00)(2.000,1.547){2}{\rule{0.400pt}{0.350pt}}
\put(539,500.67){\rule{0.482pt}{0.400pt}}
\multiput(539.00,500.17)(1.000,1.000){2}{\rule{0.241pt}{0.400pt}}
\put(541,500.67){\rule{0.482pt}{0.400pt}}
\multiput(541.00,501.17)(1.000,-1.000){2}{\rule{0.241pt}{0.400pt}}
\put(543,499.17){\rule{0.482pt}{0.400pt}}
\multiput(543.00,500.17)(1.000,-2.000){2}{\rule{0.241pt}{0.400pt}}
\put(545.17,496){\rule{0.400pt}{0.700pt}}
\multiput(544.17,497.55)(2.000,-1.547){2}{\rule{0.400pt}{0.350pt}}
\put(547.17,492){\rule{0.400pt}{0.900pt}}
\multiput(546.17,494.13)(2.000,-2.132){2}{\rule{0.400pt}{0.450pt}}
\put(549.17,488){\rule{0.400pt}{0.900pt}}
\multiput(548.17,490.13)(2.000,-2.132){2}{\rule{0.400pt}{0.450pt}}
\put(551,486.17){\rule{0.482pt}{0.400pt}}
\multiput(551.00,487.17)(1.000,-2.000){2}{\rule{0.241pt}{0.400pt}}
\put(553,484.67){\rule{0.482pt}{0.400pt}}
\multiput(553.00,485.17)(1.000,-1.000){2}{\rule{0.241pt}{0.400pt}}
\put(555,485.17){\rule{0.482pt}{0.400pt}}
\multiput(555.00,484.17)(1.000,2.000){2}{\rule{0.241pt}{0.400pt}}
\put(557.17,487){\rule{0.400pt}{1.100pt}}
\multiput(556.17,487.00)(2.000,2.717){2}{\rule{0.400pt}{0.550pt}}
\put(559.17,492){\rule{0.400pt}{1.100pt}}
\multiput(558.17,492.00)(2.000,2.717){2}{\rule{0.400pt}{0.550pt}}
\multiput(561.61,497.00)(0.447,1.132){3}{\rule{0.108pt}{0.900pt}}
\multiput(560.17,497.00)(3.000,4.132){2}{\rule{0.400pt}{0.450pt}}
\put(564.17,503){\rule{0.400pt}{1.100pt}}
\multiput(563.17,503.00)(2.000,2.717){2}{\rule{0.400pt}{0.550pt}}
\put(566,507.67){\rule{0.482pt}{0.400pt}}
\multiput(566.00,507.17)(1.000,1.000){2}{\rule{0.241pt}{0.400pt}}
\put(568,507.67){\rule{0.482pt}{0.400pt}}
\multiput(568.00,508.17)(1.000,-1.000){2}{\rule{0.241pt}{0.400pt}}
\put(570.17,503){\rule{0.400pt}{1.100pt}}
\multiput(569.17,505.72)(2.000,-2.717){2}{\rule{0.400pt}{0.550pt}}
\put(572.17,495){\rule{0.400pt}{1.700pt}}
\multiput(571.17,499.47)(2.000,-4.472){2}{\rule{0.400pt}{0.850pt}}
\put(574.17,487){\rule{0.400pt}{1.700pt}}
\multiput(573.17,491.47)(2.000,-4.472){2}{\rule{0.400pt}{0.850pt}}
\put(576.17,479){\rule{0.400pt}{1.700pt}}
\multiput(575.17,483.47)(2.000,-4.472){2}{\rule{0.400pt}{0.850pt}}
\put(578.17,475){\rule{0.400pt}{0.900pt}}
\multiput(577.17,477.13)(2.000,-2.132){2}{\rule{0.400pt}{0.450pt}}
\put(580,474.67){\rule{0.482pt}{0.400pt}}
\multiput(580.00,474.17)(1.000,1.000){2}{\rule{0.241pt}{0.400pt}}
\put(582.17,476){\rule{0.400pt}{1.100pt}}
\multiput(581.17,476.00)(2.000,2.717){2}{\rule{0.400pt}{0.550pt}}
\put(584.17,481){\rule{0.400pt}{1.900pt}}
\multiput(583.17,481.00)(2.000,5.056){2}{\rule{0.400pt}{0.950pt}}
\put(586.17,490){\rule{0.400pt}{2.500pt}}
\multiput(585.17,490.00)(2.000,6.811){2}{\rule{0.400pt}{1.250pt}}
\put(588.17,502){\rule{0.400pt}{2.300pt}}
\multiput(587.17,502.00)(2.000,6.226){2}{\rule{0.400pt}{1.150pt}}
\put(590.17,513){\rule{0.400pt}{1.700pt}}
\multiput(589.17,513.00)(2.000,4.472){2}{\rule{0.400pt}{0.850pt}}
\put(592,521.17){\rule{0.482pt}{0.400pt}}
\multiput(592.00,520.17)(1.000,2.000){2}{\rule{0.241pt}{0.400pt}}
\multiput(594.61,519.82)(0.447,-0.909){3}{\rule{0.108pt}{0.767pt}}
\multiput(593.17,521.41)(3.000,-3.409){2}{\rule{0.400pt}{0.383pt}}
\put(597.17,507){\rule{0.400pt}{2.300pt}}
\multiput(596.17,513.23)(2.000,-6.226){2}{\rule{0.400pt}{1.150pt}}
\put(599.17,492){\rule{0.400pt}{3.100pt}}
\multiput(598.17,500.57)(2.000,-8.566){2}{\rule{0.400pt}{1.550pt}}
\put(601.17,476){\rule{0.400pt}{3.300pt}}
\multiput(600.17,485.15)(2.000,-9.151){2}{\rule{0.400pt}{1.650pt}}
\put(603.17,463){\rule{0.400pt}{2.700pt}}
\multiput(602.17,470.40)(2.000,-7.396){2}{\rule{0.400pt}{1.350pt}}
\put(605.17,457){\rule{0.400pt}{1.300pt}}
\multiput(604.17,460.30)(2.000,-3.302){2}{\rule{0.400pt}{0.650pt}}
\put(607.17,457){\rule{0.400pt}{0.700pt}}
\multiput(606.17,457.00)(2.000,1.547){2}{\rule{0.400pt}{0.350pt}}
\put(609.17,460){\rule{0.400pt}{2.500pt}}
\multiput(608.17,460.00)(2.000,6.811){2}{\rule{0.400pt}{1.250pt}}
\put(611.17,472){\rule{0.400pt}{3.900pt}}
\multiput(610.17,472.00)(2.000,10.905){2}{\rule{0.400pt}{1.950pt}}
\put(613.17,491){\rule{0.400pt}{4.500pt}}
\multiput(612.17,491.00)(2.000,12.660){2}{\rule{0.400pt}{2.250pt}}
\put(615.17,513){\rule{0.400pt}{3.900pt}}
\multiput(614.17,513.00)(2.000,10.905){2}{\rule{0.400pt}{1.950pt}}
\put(617.17,532){\rule{0.400pt}{2.500pt}}
\multiput(616.17,532.00)(2.000,6.811){2}{\rule{0.400pt}{1.250pt}}
\put(619,543.67){\rule{0.482pt}{0.400pt}}
\multiput(619.00,543.17)(1.000,1.000){2}{\rule{0.241pt}{0.400pt}}
\put(621.17,534){\rule{0.400pt}{2.300pt}}
\multiput(620.17,540.23)(2.000,-6.226){2}{\rule{0.400pt}{1.150pt}}
\put(623.17,512){\rule{0.400pt}{4.500pt}}
\multiput(622.17,524.66)(2.000,-12.660){2}{\rule{0.400pt}{2.250pt}}
\multiput(625.61,495.53)(0.447,-6.267){3}{\rule{0.108pt}{3.967pt}}
\multiput(624.17,503.77)(3.000,-20.767){2}{\rule{0.400pt}{1.983pt}}
\put(628.17,455){\rule{0.400pt}{5.700pt}}
\multiput(627.17,471.17)(2.000,-16.169){2}{\rule{0.400pt}{2.850pt}}
\put(630.17,435){\rule{0.400pt}{4.100pt}}
\multiput(629.17,446.49)(2.000,-11.490){2}{\rule{0.400pt}{2.050pt}}
\put(632.17,428){\rule{0.400pt}{1.500pt}}
\multiput(631.17,431.89)(2.000,-3.887){2}{\rule{0.400pt}{0.750pt}}
\put(634.17,428){\rule{0.400pt}{1.700pt}}
\multiput(633.17,428.00)(2.000,4.472){2}{\rule{0.400pt}{0.850pt}}
\put(636.17,436){\rule{0.400pt}{4.900pt}}
\multiput(635.17,436.00)(2.000,13.830){2}{\rule{0.400pt}{2.450pt}}
\put(638.17,460){\rule{0.400pt}{7.100pt}}
\multiput(637.17,460.00)(2.000,20.264){2}{\rule{0.400pt}{3.550pt}}
\put(640.17,495){\rule{0.400pt}{7.500pt}}
\multiput(639.17,495.00)(2.000,21.433){2}{\rule{0.400pt}{3.750pt}}
\put(642.17,532){\rule{0.400pt}{6.500pt}}
\multiput(641.17,532.00)(2.000,18.509){2}{\rule{0.400pt}{3.250pt}}
\put(644.17,564){\rule{0.400pt}{3.300pt}}
\multiput(643.17,564.00)(2.000,9.151){2}{\rule{0.400pt}{1.650pt}}
\put(646.17,577){\rule{0.400pt}{0.700pt}}
\multiput(645.17,578.55)(2.000,-1.547){2}{\rule{0.400pt}{0.350pt}}
\put(648.17,553){\rule{0.400pt}{4.900pt}}
\multiput(647.17,566.83)(2.000,-13.830){2}{\rule{0.400pt}{2.450pt}}
\put(650.17,514){\rule{0.400pt}{7.900pt}}
\multiput(649.17,536.60)(2.000,-22.603){2}{\rule{0.400pt}{3.950pt}}
\put(652.17,466){\rule{0.400pt}{9.700pt}}
\multiput(651.17,493.87)(2.000,-27.867){2}{\rule{0.400pt}{4.850pt}}
\put(654.17,422){\rule{0.400pt}{8.900pt}}
\multiput(653.17,447.53)(2.000,-25.528){2}{\rule{0.400pt}{4.450pt}}
\put(656.17,393){\rule{0.400pt}{5.900pt}}
\multiput(655.17,409.75)(2.000,-16.754){2}{\rule{0.400pt}{2.950pt}}
\multiput(658.61,389.26)(0.447,-1.132){3}{\rule{0.108pt}{0.900pt}}
\multiput(657.17,391.13)(3.000,-4.132){2}{\rule{0.400pt}{0.450pt}}
\put(661.17,387){\rule{0.400pt}{4.100pt}}
\multiput(660.17,387.00)(2.000,11.490){2}{\rule{0.400pt}{2.050pt}}
\put(663.17,407){\rule{0.400pt}{8.700pt}}
\multiput(662.17,407.00)(2.000,24.943){2}{\rule{0.400pt}{4.350pt}}
\put(665.17,450){\rule{0.400pt}{11.500pt}}
\multiput(664.17,450.00)(2.000,33.131){2}{\rule{0.400pt}{5.750pt}}
\put(667.17,507){\rule{0.400pt}{11.700pt}}
\multiput(666.17,507.00)(2.000,33.716){2}{\rule{0.400pt}{5.850pt}}
\put(669.17,565){\rule{0.400pt}{8.900pt}}
\multiput(668.17,565.00)(2.000,25.528){2}{\rule{0.400pt}{4.450pt}}
\put(671.17,609){\rule{0.400pt}{4.100pt}}
\multiput(670.17,609.00)(2.000,11.490){2}{\rule{0.400pt}{2.050pt}}
\put(673.17,617){\rule{0.400pt}{2.500pt}}
\multiput(672.17,623.81)(2.000,-6.811){2}{\rule{0.400pt}{1.250pt}}
\put(675.17,574){\rule{0.400pt}{8.700pt}}
\multiput(674.17,598.94)(2.000,-24.943){2}{\rule{0.400pt}{4.350pt}}
\put(677.17,509){\rule{0.400pt}{13.100pt}}
\multiput(676.17,546.81)(2.000,-37.810){2}{\rule{0.400pt}{6.550pt}}
\put(679.17,437){\rule{0.400pt}{14.500pt}}
\multiput(678.17,478.90)(2.000,-41.905){2}{\rule{0.400pt}{7.250pt}}
\put(681.17,375){\rule{0.400pt}{12.500pt}}
\multiput(680.17,411.06)(2.000,-36.056){2}{\rule{0.400pt}{6.250pt}}
\put(683.17,338){\rule{0.400pt}{7.500pt}}
\multiput(682.17,359.43)(2.000,-21.433){2}{\rule{0.400pt}{3.750pt}}
\put(685,336.67){\rule{0.482pt}{0.400pt}}
\multiput(685.00,337.17)(1.000,-1.000){2}{\rule{0.241pt}{0.400pt}}
\put(687.17,337){\rule{0.400pt}{7.700pt}}
\multiput(686.17,337.00)(2.000,22.018){2}{\rule{0.400pt}{3.850pt}}
\put(689.17,375){\rule{0.400pt}{14.100pt}}
\multiput(688.17,375.00)(2.000,40.735){2}{\rule{0.400pt}{7.050pt}}
\multiput(691.61,445.00)(0.447,18.770){3}{\rule{0.108pt}{11.433pt}}
\multiput(690.17,445.00)(3.000,61.270){2}{\rule{0.400pt}{5.717pt}}
\put(694.17,530){\rule{0.400pt}{16.300pt}}
\multiput(693.17,530.00)(2.000,47.169){2}{\rule{0.400pt}{8.150pt}}
\put(696.17,611){\rule{0.400pt}{11.700pt}}
\multiput(695.17,611.00)(2.000,33.716){2}{\rule{0.400pt}{5.850pt}}
\put(698.17,669){\rule{0.400pt}{3.700pt}}
\multiput(697.17,669.00)(2.000,10.320){2}{\rule{0.400pt}{1.850pt}}
\put(700.17,659){\rule{0.400pt}{5.700pt}}
\multiput(699.17,675.17)(2.000,-16.169){2}{\rule{0.400pt}{2.850pt}}
\put(702.17,590){\rule{0.400pt}{13.900pt}}
\multiput(701.17,630.15)(2.000,-40.150){2}{\rule{0.400pt}{6.950pt}}
\put(704.17,495){\rule{0.400pt}{19.100pt}}
\multiput(703.17,550.36)(2.000,-55.357){2}{\rule{0.400pt}{9.550pt}}
\put(706.17,395){\rule{0.400pt}{20.100pt}}
\multiput(705.17,453.28)(2.000,-58.281){2}{\rule{0.400pt}{10.050pt}}
\put(708.17,314){\rule{0.400pt}{16.300pt}}
\multiput(707.17,361.17)(2.000,-47.169){2}{\rule{0.400pt}{8.150pt}}
\put(710.17,274){\rule{0.400pt}{8.100pt}}
\multiput(709.17,297.19)(2.000,-23.188){2}{\rule{0.400pt}{4.050pt}}
\put(712.17,274){\rule{0.400pt}{2.300pt}}
\multiput(711.17,274.00)(2.000,6.226){2}{\rule{0.400pt}{1.150pt}}
\put(714.17,285){\rule{0.400pt}{12.700pt}}
\multiput(713.17,285.00)(2.000,36.641){2}{\rule{0.400pt}{6.350pt}}
\put(716.17,348){\rule{0.400pt}{20.300pt}}
\multiput(715.17,348.00)(2.000,58.866){2}{\rule{0.400pt}{10.150pt}}
\put(718.17,449){\rule{0.400pt}{23.300pt}}
\multiput(717.17,449.00)(2.000,67.640){2}{\rule{0.400pt}{11.650pt}}
\put(720.17,565){\rule{0.400pt}{20.900pt}}
\multiput(719.17,565.00)(2.000,60.621){2}{\rule{0.400pt}{10.450pt}}
\multiput(722.61,669.00)(0.447,14.528){3}{\rule{0.108pt}{8.900pt}}
\multiput(721.17,669.00)(3.000,47.528){2}{\rule{0.400pt}{4.450pt}}
\put(725.17,735){\rule{0.400pt}{2.300pt}}
\multiput(724.17,735.00)(2.000,6.226){2}{\rule{0.400pt}{1.150pt}}
\put(727.17,696){\rule{0.400pt}{10.100pt}}
\multiput(726.17,725.04)(2.000,-29.037){2}{\rule{0.400pt}{5.050pt}}
\put(729.17,597){\rule{0.400pt}{19.900pt}}
\multiput(728.17,654.70)(2.000,-57.697){2}{\rule{0.400pt}{9.950pt}}
\put(731.17,469){\rule{0.400pt}{25.700pt}}
\multiput(730.17,543.66)(2.000,-74.658){2}{\rule{0.400pt}{12.850pt}}
\put(733.17,342){\rule{0.400pt}{25.500pt}}
\multiput(732.17,416.07)(2.000,-74.073){2}{\rule{0.400pt}{12.750pt}}
\put(735.17,249){\rule{0.400pt}{18.700pt}}
\multiput(734.17,303.19)(2.000,-54.187){2}{\rule{0.400pt}{9.350pt}}
\put(737.17,212){\rule{0.400pt}{7.500pt}}
\multiput(736.17,233.43)(2.000,-21.433){2}{\rule{0.400pt}{3.750pt}}
\put(739.17,212){\rule{0.400pt}{5.900pt}}
\multiput(738.17,212.00)(2.000,16.754){2}{\rule{0.400pt}{2.950pt}}
\put(741.17,241){\rule{0.400pt}{18.500pt}}
\multiput(740.17,241.00)(2.000,53.602){2}{\rule{0.400pt}{9.250pt}}
\put(743.17,333){\rule{0.400pt}{26.700pt}}
\multiput(742.17,333.00)(2.000,77.583){2}{\rule{0.400pt}{13.350pt}}
\put(745.17,466){\rule{0.400pt}{28.900pt}}
\multiput(744.17,466.00)(2.000,84.017){2}{\rule{0.400pt}{14.450pt}}
\put(747.17,610){\rule{0.400pt}{24.300pt}}
\multiput(746.17,610.00)(2.000,70.564){2}{\rule{0.400pt}{12.150pt}}
\put(749.17,731){\rule{0.400pt}{13.500pt}}
\multiput(748.17,731.00)(2.000,38.980){2}{\rule{0.400pt}{6.750pt}}
\put(751.17,794){\rule{0.400pt}{0.900pt}}
\multiput(750.17,796.13)(2.000,-2.132){2}{\rule{0.400pt}{0.450pt}}
\put(753.17,719){\rule{0.400pt}{15.100pt}}
\multiput(752.17,762.66)(2.000,-43.659){2}{\rule{0.400pt}{7.550pt}}
\multiput(755.61,646.08)(0.447,-29.040){3}{\rule{0.108pt}{17.567pt}}
\multiput(754.17,682.54)(3.000,-94.540){2}{\rule{0.400pt}{8.783pt}}
\put(758.17,432){\rule{0.400pt}{31.300pt}}
\multiput(757.17,523.04)(2.000,-91.035){2}{\rule{0.400pt}{15.650pt}}
\put(760.17,288){\rule{0.400pt}{28.900pt}}
\multiput(759.17,372.02)(2.000,-84.017){2}{\rule{0.400pt}{14.450pt}}
\put(762.17,190){\rule{0.400pt}{19.700pt}}
\multiput(761.17,247.11)(2.000,-57.112){2}{\rule{0.400pt}{9.850pt}}
\put(764.17,163){\rule{0.400pt}{5.500pt}}
\multiput(763.17,178.58)(2.000,-15.584){2}{\rule{0.400pt}{2.750pt}}
\put(766.17,163){\rule{0.400pt}{10.500pt}}
\multiput(765.17,163.00)(2.000,30.207){2}{\rule{0.400pt}{5.250pt}}
\put(768.17,215){\rule{0.400pt}{24.100pt}}
\multiput(767.17,215.00)(2.000,69.979){2}{\rule{0.400pt}{12.050pt}}
\put(770.17,335){\rule{0.400pt}{32.100pt}}
\multiput(769.17,335.00)(2.000,93.375){2}{\rule{0.400pt}{16.050pt}}
\put(772.17,495){\rule{0.400pt}{32.500pt}}
\multiput(771.17,495.00)(2.000,94.545){2}{\rule{0.400pt}{16.250pt}}
\put(774.17,657){\rule{0.400pt}{25.500pt}}
\multiput(773.17,657.00)(2.000,74.073){2}{\rule{0.400pt}{12.750pt}}
\put(776.17,784){\rule{0.400pt}{11.900pt}}
\multiput(775.17,784.00)(2.000,34.301){2}{\rule{0.400pt}{5.950pt}}
\put(778.17,820){\rule{0.400pt}{4.700pt}}
\multiput(777.17,833.24)(2.000,-13.245){2}{\rule{0.400pt}{2.350pt}}
\put(780.17,719){\rule{0.400pt}{20.300pt}}
\multiput(779.17,777.87)(2.000,-58.866){2}{\rule{0.400pt}{10.150pt}}
\put(782.17,564){\rule{0.400pt}{31.100pt}}
\multiput(781.17,654.45)(2.000,-90.450){2}{\rule{0.400pt}{15.550pt}}
\put(784.17,391){\rule{0.400pt}{34.700pt}}
\multiput(783.17,491.98)(2.000,-100.978){2}{\rule{0.400pt}{17.350pt}}
\put(786.17,242){\rule{0.400pt}{29.900pt}}
\multiput(785.17,328.94)(2.000,-86.941){2}{\rule{0.400pt}{14.950pt}}
\multiput(788.61,191.22)(0.447,-20.109){3}{\rule{0.108pt}{12.233pt}}
\multiput(787.17,216.61)(3.000,-65.609){2}{\rule{0.400pt}{6.117pt}}
\put(791.17,141){\rule{0.400pt}{2.100pt}}
\multiput(790.17,146.64)(2.000,-5.641){2}{\rule{0.400pt}{1.050pt}}
\put(793.17,141){\rule{0.400pt}{14.900pt}}
\multiput(792.17,141.00)(2.000,43.074){2}{\rule{0.400pt}{7.450pt}}
\put(795.17,215){\rule{0.400pt}{28.300pt}}
\multiput(794.17,215.00)(2.000,82.262){2}{\rule{0.400pt}{14.150pt}}
\put(797.17,356){\rule{0.400pt}{35.100pt}}
\multiput(796.17,356.00)(2.000,102.148){2}{\rule{0.400pt}{17.550pt}}
\put(799.17,531){\rule{0.400pt}{33.300pt}}
\multiput(798.17,531.00)(2.000,96.884){2}{\rule{0.400pt}{16.650pt}}
\put(801.17,697){\rule{0.400pt}{23.900pt}}
\multiput(800.17,697.00)(2.000,69.394){2}{\rule{0.400pt}{11.950pt}}
\put(803.17,816){\rule{0.400pt}{8.700pt}}
\multiput(802.17,816.00)(2.000,24.943){2}{\rule{0.400pt}{4.350pt}}
\put(805.17,816){\rule{0.400pt}{8.700pt}}
\multiput(804.17,840.94)(2.000,-24.943){2}{\rule{0.400pt}{4.350pt}}
\put(807.17,697){\rule{0.400pt}{23.900pt}}
\multiput(806.17,766.39)(2.000,-69.394){2}{\rule{0.400pt}{11.950pt}}
\put(809.17,531){\rule{0.400pt}{33.300pt}}
\multiput(808.17,627.88)(2.000,-96.884){2}{\rule{0.400pt}{16.650pt}}
\put(811.17,356){\rule{0.400pt}{35.100pt}}
\multiput(810.17,458.15)(2.000,-102.148){2}{\rule{0.400pt}{17.550pt}}
\put(813.17,215){\rule{0.400pt}{28.300pt}}
\multiput(812.17,297.26)(2.000,-82.262){2}{\rule{0.400pt}{14.150pt}}
\put(815.17,141){\rule{0.400pt}{14.900pt}}
\multiput(814.17,184.07)(2.000,-43.074){2}{\rule{0.400pt}{7.450pt}}
\put(817.17,141){\rule{0.400pt}{2.100pt}}
\multiput(816.17,141.00)(2.000,5.641){2}{\rule{0.400pt}{1.050pt}}
\multiput(819.61,151.00)(0.447,20.109){3}{\rule{0.108pt}{12.233pt}}
\multiput(818.17,151.00)(3.000,65.609){2}{\rule{0.400pt}{6.117pt}}
\put(822.17,242){\rule{0.400pt}{29.900pt}}
\multiput(821.17,242.00)(2.000,86.941){2}{\rule{0.400pt}{14.950pt}}
\put(824.17,391){\rule{0.400pt}{34.700pt}}
\multiput(823.17,391.00)(2.000,100.978){2}{\rule{0.400pt}{17.350pt}}
\put(826.17,564){\rule{0.400pt}{31.100pt}}
\multiput(825.17,564.00)(2.000,90.450){2}{\rule{0.400pt}{15.550pt}}
\put(828.17,719){\rule{0.400pt}{20.300pt}}
\multiput(827.17,719.00)(2.000,58.866){2}{\rule{0.400pt}{10.150pt}}
\put(830.17,820){\rule{0.400pt}{4.700pt}}
\multiput(829.17,820.00)(2.000,13.245){2}{\rule{0.400pt}{2.350pt}}
\put(832.17,784){\rule{0.400pt}{11.900pt}}
\multiput(831.17,818.30)(2.000,-34.301){2}{\rule{0.400pt}{5.950pt}}
\put(834.17,657){\rule{0.400pt}{25.500pt}}
\multiput(833.17,731.07)(2.000,-74.073){2}{\rule{0.400pt}{12.750pt}}
\put(836.17,495){\rule{0.400pt}{32.500pt}}
\multiput(835.17,589.54)(2.000,-94.545){2}{\rule{0.400pt}{16.250pt}}
\put(838.17,335){\rule{0.400pt}{32.100pt}}
\multiput(837.17,428.37)(2.000,-93.375){2}{\rule{0.400pt}{16.050pt}}
\put(840.17,215){\rule{0.400pt}{24.100pt}}
\multiput(839.17,284.98)(2.000,-69.979){2}{\rule{0.400pt}{12.050pt}}
\put(842.17,163){\rule{0.400pt}{10.500pt}}
\multiput(841.17,193.21)(2.000,-30.207){2}{\rule{0.400pt}{5.250pt}}
\put(844.17,163){\rule{0.400pt}{5.500pt}}
\multiput(843.17,163.00)(2.000,15.584){2}{\rule{0.400pt}{2.750pt}}
\put(846.17,190){\rule{0.400pt}{19.700pt}}
\multiput(845.17,190.00)(2.000,57.112){2}{\rule{0.400pt}{9.850pt}}
\put(848.17,288){\rule{0.400pt}{28.900pt}}
\multiput(847.17,288.00)(2.000,84.017){2}{\rule{0.400pt}{14.450pt}}
\put(850.17,432){\rule{0.400pt}{31.300pt}}
\multiput(849.17,432.00)(2.000,91.035){2}{\rule{0.400pt}{15.650pt}}
\multiput(852.61,588.00)(0.447,29.040){3}{\rule{0.108pt}{17.567pt}}
\multiput(851.17,588.00)(3.000,94.540){2}{\rule{0.400pt}{8.783pt}}
\put(855.17,719){\rule{0.400pt}{15.100pt}}
\multiput(854.17,719.00)(2.000,43.659){2}{\rule{0.400pt}{7.550pt}}
\put(857.17,794){\rule{0.400pt}{0.900pt}}
\multiput(856.17,794.00)(2.000,2.132){2}{\rule{0.400pt}{0.450pt}}
\put(859.17,731){\rule{0.400pt}{13.500pt}}
\multiput(858.17,769.98)(2.000,-38.980){2}{\rule{0.400pt}{6.750pt}}
\put(861.17,610){\rule{0.400pt}{24.300pt}}
\multiput(860.17,680.56)(2.000,-70.564){2}{\rule{0.400pt}{12.150pt}}
\put(863.17,466){\rule{0.400pt}{28.900pt}}
\multiput(862.17,550.02)(2.000,-84.017){2}{\rule{0.400pt}{14.450pt}}
\put(865.17,333){\rule{0.400pt}{26.700pt}}
\multiput(864.17,410.58)(2.000,-77.583){2}{\rule{0.400pt}{13.350pt}}
\put(867.17,241){\rule{0.400pt}{18.500pt}}
\multiput(866.17,294.60)(2.000,-53.602){2}{\rule{0.400pt}{9.250pt}}
\put(869.17,212){\rule{0.400pt}{5.900pt}}
\multiput(868.17,228.75)(2.000,-16.754){2}{\rule{0.400pt}{2.950pt}}
\put(871.17,212){\rule{0.400pt}{7.500pt}}
\multiput(870.17,212.00)(2.000,21.433){2}{\rule{0.400pt}{3.750pt}}
\put(873.17,249){\rule{0.400pt}{18.700pt}}
\multiput(872.17,249.00)(2.000,54.187){2}{\rule{0.400pt}{9.350pt}}
\put(875.17,342){\rule{0.400pt}{25.500pt}}
\multiput(874.17,342.00)(2.000,74.073){2}{\rule{0.400pt}{12.750pt}}
\put(877.17,469){\rule{0.400pt}{25.700pt}}
\multiput(876.17,469.00)(2.000,74.658){2}{\rule{0.400pt}{12.850pt}}
\put(879.17,597){\rule{0.400pt}{19.900pt}}
\multiput(878.17,597.00)(2.000,57.697){2}{\rule{0.400pt}{9.950pt}}
\put(881.17,696){\rule{0.400pt}{10.100pt}}
\multiput(880.17,696.00)(2.000,29.037){2}{\rule{0.400pt}{5.050pt}}
\put(883.17,735){\rule{0.400pt}{2.300pt}}
\multiput(882.17,741.23)(2.000,-6.226){2}{\rule{0.400pt}{1.150pt}}
\multiput(885.61,698.06)(0.447,-14.528){3}{\rule{0.108pt}{8.900pt}}
\multiput(884.17,716.53)(3.000,-47.528){2}{\rule{0.400pt}{4.450pt}}
\put(888.17,565){\rule{0.400pt}{20.900pt}}
\multiput(887.17,625.62)(2.000,-60.621){2}{\rule{0.400pt}{10.450pt}}
\put(890.17,449){\rule{0.400pt}{23.300pt}}
\multiput(889.17,516.64)(2.000,-67.640){2}{\rule{0.400pt}{11.650pt}}
\put(892.17,348){\rule{0.400pt}{20.300pt}}
\multiput(891.17,406.87)(2.000,-58.866){2}{\rule{0.400pt}{10.150pt}}
\put(894.17,285){\rule{0.400pt}{12.700pt}}
\multiput(893.17,321.64)(2.000,-36.641){2}{\rule{0.400pt}{6.350pt}}
\put(896.17,274){\rule{0.400pt}{2.300pt}}
\multiput(895.17,280.23)(2.000,-6.226){2}{\rule{0.400pt}{1.150pt}}
\put(898.17,274){\rule{0.400pt}{8.100pt}}
\multiput(897.17,274.00)(2.000,23.188){2}{\rule{0.400pt}{4.050pt}}
\put(900.17,314){\rule{0.400pt}{16.300pt}}
\multiput(899.17,314.00)(2.000,47.169){2}{\rule{0.400pt}{8.150pt}}
\put(902.17,395){\rule{0.400pt}{20.100pt}}
\multiput(901.17,395.00)(2.000,58.281){2}{\rule{0.400pt}{10.050pt}}
\put(904.17,495){\rule{0.400pt}{19.100pt}}
\multiput(903.17,495.00)(2.000,55.357){2}{\rule{0.400pt}{9.550pt}}
\put(906.17,590){\rule{0.400pt}{13.900pt}}
\multiput(905.17,590.00)(2.000,40.150){2}{\rule{0.400pt}{6.950pt}}
\put(908.17,659){\rule{0.400pt}{5.700pt}}
\multiput(907.17,659.00)(2.000,16.169){2}{\rule{0.400pt}{2.850pt}}
\put(910.17,669){\rule{0.400pt}{3.700pt}}
\multiput(909.17,679.32)(2.000,-10.320){2}{\rule{0.400pt}{1.850pt}}
\put(912.17,611){\rule{0.400pt}{11.700pt}}
\multiput(911.17,644.72)(2.000,-33.716){2}{\rule{0.400pt}{5.850pt}}
\put(914.17,530){\rule{0.400pt}{16.300pt}}
\multiput(913.17,577.17)(2.000,-47.169){2}{\rule{0.400pt}{8.150pt}}
\multiput(916.61,482.54)(0.447,-18.770){3}{\rule{0.108pt}{11.433pt}}
\multiput(915.17,506.27)(3.000,-61.270){2}{\rule{0.400pt}{5.717pt}}
\put(919.17,375){\rule{0.400pt}{14.100pt}}
\multiput(918.17,415.73)(2.000,-40.735){2}{\rule{0.400pt}{7.050pt}}
\put(921.17,337){\rule{0.400pt}{7.700pt}}
\multiput(920.17,359.02)(2.000,-22.018){2}{\rule{0.400pt}{3.850pt}}
\put(923,336.67){\rule{0.482pt}{0.400pt}}
\multiput(923.00,336.17)(1.000,1.000){2}{\rule{0.241pt}{0.400pt}}
\put(925.17,338){\rule{0.400pt}{7.500pt}}
\multiput(924.17,338.00)(2.000,21.433){2}{\rule{0.400pt}{3.750pt}}
\put(927.17,375){\rule{0.400pt}{12.500pt}}
\multiput(926.17,375.00)(2.000,36.056){2}{\rule{0.400pt}{6.250pt}}
\put(929.17,437){\rule{0.400pt}{14.500pt}}
\multiput(928.17,437.00)(2.000,41.905){2}{\rule{0.400pt}{7.250pt}}
\put(931.17,509){\rule{0.400pt}{13.100pt}}
\multiput(930.17,509.00)(2.000,37.810){2}{\rule{0.400pt}{6.550pt}}
\put(933.17,574){\rule{0.400pt}{8.700pt}}
\multiput(932.17,574.00)(2.000,24.943){2}{\rule{0.400pt}{4.350pt}}
\put(935.17,617){\rule{0.400pt}{2.500pt}}
\multiput(934.17,617.00)(2.000,6.811){2}{\rule{0.400pt}{1.250pt}}
\put(937.17,609){\rule{0.400pt}{4.100pt}}
\multiput(936.17,620.49)(2.000,-11.490){2}{\rule{0.400pt}{2.050pt}}
\put(939.17,565){\rule{0.400pt}{8.900pt}}
\multiput(938.17,590.53)(2.000,-25.528){2}{\rule{0.400pt}{4.450pt}}
\put(941.17,507){\rule{0.400pt}{11.700pt}}
\multiput(940.17,540.72)(2.000,-33.716){2}{\rule{0.400pt}{5.850pt}}
\put(943.17,450){\rule{0.400pt}{11.500pt}}
\multiput(942.17,483.13)(2.000,-33.131){2}{\rule{0.400pt}{5.750pt}}
\put(945.17,407){\rule{0.400pt}{8.700pt}}
\multiput(944.17,431.94)(2.000,-24.943){2}{\rule{0.400pt}{4.350pt}}
\put(947.17,387){\rule{0.400pt}{4.100pt}}
\multiput(946.17,398.49)(2.000,-11.490){2}{\rule{0.400pt}{2.050pt}}
\multiput(949.61,387.00)(0.447,1.132){3}{\rule{0.108pt}{0.900pt}}
\multiput(948.17,387.00)(3.000,4.132){2}{\rule{0.400pt}{0.450pt}}
\put(952.17,393){\rule{0.400pt}{5.900pt}}
\multiput(951.17,393.00)(2.000,16.754){2}{\rule{0.400pt}{2.950pt}}
\put(954.17,422){\rule{0.400pt}{8.900pt}}
\multiput(953.17,422.00)(2.000,25.528){2}{\rule{0.400pt}{4.450pt}}
\put(956.17,466){\rule{0.400pt}{9.700pt}}
\multiput(955.17,466.00)(2.000,27.867){2}{\rule{0.400pt}{4.850pt}}
\put(958.17,514){\rule{0.400pt}{7.900pt}}
\multiput(957.17,514.00)(2.000,22.603){2}{\rule{0.400pt}{3.950pt}}
\put(960.17,553){\rule{0.400pt}{4.900pt}}
\multiput(959.17,553.00)(2.000,13.830){2}{\rule{0.400pt}{2.450pt}}
\put(962.17,577){\rule{0.400pt}{0.700pt}}
\multiput(961.17,577.00)(2.000,1.547){2}{\rule{0.400pt}{0.350pt}}
\put(964.17,564){\rule{0.400pt}{3.300pt}}
\multiput(963.17,573.15)(2.000,-9.151){2}{\rule{0.400pt}{1.650pt}}
\put(966.17,532){\rule{0.400pt}{6.500pt}}
\multiput(965.17,550.51)(2.000,-18.509){2}{\rule{0.400pt}{3.250pt}}
\put(968.17,495){\rule{0.400pt}{7.500pt}}
\multiput(967.17,516.43)(2.000,-21.433){2}{\rule{0.400pt}{3.750pt}}
\put(970.17,460){\rule{0.400pt}{7.100pt}}
\multiput(969.17,480.26)(2.000,-20.264){2}{\rule{0.400pt}{3.550pt}}
\put(972.17,436){\rule{0.400pt}{4.900pt}}
\multiput(971.17,449.83)(2.000,-13.830){2}{\rule{0.400pt}{2.450pt}}
\put(974.17,428){\rule{0.400pt}{1.700pt}}
\multiput(973.17,432.47)(2.000,-4.472){2}{\rule{0.400pt}{0.850pt}}
\put(976.17,428){\rule{0.400pt}{1.500pt}}
\multiput(975.17,428.00)(2.000,3.887){2}{\rule{0.400pt}{0.750pt}}
\put(978.17,435){\rule{0.400pt}{4.100pt}}
\multiput(977.17,435.00)(2.000,11.490){2}{\rule{0.400pt}{2.050pt}}
\put(980.17,455){\rule{0.400pt}{5.700pt}}
\multiput(979.17,455.00)(2.000,16.169){2}{\rule{0.400pt}{2.850pt}}
\multiput(982.61,483.00)(0.447,6.267){3}{\rule{0.108pt}{3.967pt}}
\multiput(981.17,483.00)(3.000,20.767){2}{\rule{0.400pt}{1.983pt}}
\put(985.17,512){\rule{0.400pt}{4.500pt}}
\multiput(984.17,512.00)(2.000,12.660){2}{\rule{0.400pt}{2.250pt}}
\put(987.17,534){\rule{0.400pt}{2.300pt}}
\multiput(986.17,534.00)(2.000,6.226){2}{\rule{0.400pt}{1.150pt}}
\put(989,543.67){\rule{0.482pt}{0.400pt}}
\multiput(989.00,544.17)(1.000,-1.000){2}{\rule{0.241pt}{0.400pt}}
\put(991.17,532){\rule{0.400pt}{2.500pt}}
\multiput(990.17,538.81)(2.000,-6.811){2}{\rule{0.400pt}{1.250pt}}
\put(993.17,513){\rule{0.400pt}{3.900pt}}
\multiput(992.17,523.91)(2.000,-10.905){2}{\rule{0.400pt}{1.950pt}}
\put(995.17,491){\rule{0.400pt}{4.500pt}}
\multiput(994.17,503.66)(2.000,-12.660){2}{\rule{0.400pt}{2.250pt}}
\put(997.17,472){\rule{0.400pt}{3.900pt}}
\multiput(996.17,482.91)(2.000,-10.905){2}{\rule{0.400pt}{1.950pt}}
\put(999.17,460){\rule{0.400pt}{2.500pt}}
\multiput(998.17,466.81)(2.000,-6.811){2}{\rule{0.400pt}{1.250pt}}
\put(1001.17,457){\rule{0.400pt}{0.700pt}}
\multiput(1000.17,458.55)(2.000,-1.547){2}{\rule{0.400pt}{0.350pt}}
\put(1003.17,457){\rule{0.400pt}{1.300pt}}
\multiput(1002.17,457.00)(2.000,3.302){2}{\rule{0.400pt}{0.650pt}}
\put(1005.17,463){\rule{0.400pt}{2.700pt}}
\multiput(1004.17,463.00)(2.000,7.396){2}{\rule{0.400pt}{1.350pt}}
\put(1007.17,476){\rule{0.400pt}{3.300pt}}
\multiput(1006.17,476.00)(2.000,9.151){2}{\rule{0.400pt}{1.650pt}}
\put(1009.17,492){\rule{0.400pt}{3.100pt}}
\multiput(1008.17,492.00)(2.000,8.566){2}{\rule{0.400pt}{1.550pt}}
\put(1011.17,507){\rule{0.400pt}{2.300pt}}
\multiput(1010.17,507.00)(2.000,6.226){2}{\rule{0.400pt}{1.150pt}}
\multiput(1013.61,518.00)(0.447,0.909){3}{\rule{0.108pt}{0.767pt}}
\multiput(1012.17,518.00)(3.000,3.409){2}{\rule{0.400pt}{0.383pt}}
\put(1016,521.17){\rule{0.482pt}{0.400pt}}
\multiput(1016.00,522.17)(1.000,-2.000){2}{\rule{0.241pt}{0.400pt}}
\put(1018.17,513){\rule{0.400pt}{1.700pt}}
\multiput(1017.17,517.47)(2.000,-4.472){2}{\rule{0.400pt}{0.850pt}}
\put(1020.17,502){\rule{0.400pt}{2.300pt}}
\multiput(1019.17,508.23)(2.000,-6.226){2}{\rule{0.400pt}{1.150pt}}
\put(1022.17,490){\rule{0.400pt}{2.500pt}}
\multiput(1021.17,496.81)(2.000,-6.811){2}{\rule{0.400pt}{1.250pt}}
\put(1024.17,481){\rule{0.400pt}{1.900pt}}
\multiput(1023.17,486.06)(2.000,-5.056){2}{\rule{0.400pt}{0.950pt}}
\put(1026.17,476){\rule{0.400pt}{1.100pt}}
\multiput(1025.17,478.72)(2.000,-2.717){2}{\rule{0.400pt}{0.550pt}}
\put(1028,474.67){\rule{0.482pt}{0.400pt}}
\multiput(1028.00,475.17)(1.000,-1.000){2}{\rule{0.241pt}{0.400pt}}
\put(1030.17,475){\rule{0.400pt}{0.900pt}}
\multiput(1029.17,475.00)(2.000,2.132){2}{\rule{0.400pt}{0.450pt}}
\put(1032.17,479){\rule{0.400pt}{1.700pt}}
\multiput(1031.17,479.00)(2.000,4.472){2}{\rule{0.400pt}{0.850pt}}
\put(1034.17,487){\rule{0.400pt}{1.700pt}}
\multiput(1033.17,487.00)(2.000,4.472){2}{\rule{0.400pt}{0.850pt}}
\put(1036.17,495){\rule{0.400pt}{1.700pt}}
\multiput(1035.17,495.00)(2.000,4.472){2}{\rule{0.400pt}{0.850pt}}
\put(1038.17,503){\rule{0.400pt}{1.100pt}}
\multiput(1037.17,503.00)(2.000,2.717){2}{\rule{0.400pt}{0.550pt}}
\put(1040,507.67){\rule{0.482pt}{0.400pt}}
\multiput(1040.00,507.17)(1.000,1.000){2}{\rule{0.241pt}{0.400pt}}
\put(1042,507.67){\rule{0.482pt}{0.400pt}}
\multiput(1042.00,508.17)(1.000,-1.000){2}{\rule{0.241pt}{0.400pt}}
\put(1044.17,503){\rule{0.400pt}{1.100pt}}
\multiput(1043.17,505.72)(2.000,-2.717){2}{\rule{0.400pt}{0.550pt}}
\multiput(1046.61,499.26)(0.447,-1.132){3}{\rule{0.108pt}{0.900pt}}
\multiput(1045.17,501.13)(3.000,-4.132){2}{\rule{0.400pt}{0.450pt}}
\put(1049.17,492){\rule{0.400pt}{1.100pt}}
\multiput(1048.17,494.72)(2.000,-2.717){2}{\rule{0.400pt}{0.550pt}}
\put(1051.17,487){\rule{0.400pt}{1.100pt}}
\multiput(1050.17,489.72)(2.000,-2.717){2}{\rule{0.400pt}{0.550pt}}
\put(1053,485.17){\rule{0.482pt}{0.400pt}}
\multiput(1053.00,486.17)(1.000,-2.000){2}{\rule{0.241pt}{0.400pt}}
\put(1055,484.67){\rule{0.482pt}{0.400pt}}
\multiput(1055.00,484.17)(1.000,1.000){2}{\rule{0.241pt}{0.400pt}}
\put(1057,486.17){\rule{0.482pt}{0.400pt}}
\multiput(1057.00,485.17)(1.000,2.000){2}{\rule{0.241pt}{0.400pt}}
\put(1059.17,488){\rule{0.400pt}{0.900pt}}
\multiput(1058.17,488.00)(2.000,2.132){2}{\rule{0.400pt}{0.450pt}}
\put(1061.17,492){\rule{0.400pt}{0.900pt}}
\multiput(1060.17,492.00)(2.000,2.132){2}{\rule{0.400pt}{0.450pt}}
\put(1063.17,496){\rule{0.400pt}{0.700pt}}
\multiput(1062.17,496.00)(2.000,1.547){2}{\rule{0.400pt}{0.350pt}}
\put(1065,499.17){\rule{0.482pt}{0.400pt}}
\multiput(1065.00,498.17)(1.000,2.000){2}{\rule{0.241pt}{0.400pt}}
\put(1067,500.67){\rule{0.482pt}{0.400pt}}
\multiput(1067.00,500.17)(1.000,1.000){2}{\rule{0.241pt}{0.400pt}}
\put(1069,500.67){\rule{0.482pt}{0.400pt}}
\multiput(1069.00,501.17)(1.000,-1.000){2}{\rule{0.241pt}{0.400pt}}
\put(1071.17,498){\rule{0.400pt}{0.700pt}}
\multiput(1070.17,499.55)(2.000,-1.547){2}{\rule{0.400pt}{0.350pt}}
\put(1073,496.17){\rule{0.482pt}{0.400pt}}
\multiput(1073.00,497.17)(1.000,-2.000){2}{\rule{0.241pt}{0.400pt}}
\put(1075.17,493){\rule{0.400pt}{0.700pt}}
\multiput(1074.17,494.55)(2.000,-1.547){2}{\rule{0.400pt}{0.350pt}}
\put(1077,491.17){\rule{0.482pt}{0.400pt}}
\multiput(1077.00,492.17)(1.000,-2.000){2}{\rule{0.241pt}{0.400pt}}
\put(526.0,491.0){\rule[-0.200pt]{1.204pt}{0.400pt}}
\put(1084,490.67){\rule{0.482pt}{0.400pt}}
\multiput(1084.00,490.17)(1.000,1.000){2}{\rule{0.241pt}{0.400pt}}
\put(1086,492.17){\rule{0.482pt}{0.400pt}}
\multiput(1086.00,491.17)(1.000,2.000){2}{\rule{0.241pt}{0.400pt}}
\put(1088,494.17){\rule{0.482pt}{0.400pt}}
\multiput(1088.00,493.17)(1.000,2.000){2}{\rule{0.241pt}{0.400pt}}
\put(1090,495.67){\rule{0.482pt}{0.400pt}}
\multiput(1090.00,495.17)(1.000,1.000){2}{\rule{0.241pt}{0.400pt}}
\put(1092,496.67){\rule{0.482pt}{0.400pt}}
\multiput(1092.00,496.17)(1.000,1.000){2}{\rule{0.241pt}{0.400pt}}
\put(1079.0,491.0){\rule[-0.200pt]{1.204pt}{0.400pt}}
\put(1096,496.67){\rule{0.482pt}{0.400pt}}
\multiput(1096.00,497.17)(1.000,-1.000){2}{\rule{0.241pt}{0.400pt}}
\put(1098,495.67){\rule{0.482pt}{0.400pt}}
\multiput(1098.00,496.17)(1.000,-1.000){2}{\rule{0.241pt}{0.400pt}}
\put(1100,494.67){\rule{0.482pt}{0.400pt}}
\multiput(1100.00,495.17)(1.000,-1.000){2}{\rule{0.241pt}{0.400pt}}
\put(1102,493.67){\rule{0.482pt}{0.400pt}}
\multiput(1102.00,494.17)(1.000,-1.000){2}{\rule{0.241pt}{0.400pt}}
\put(1104,492.67){\rule{0.482pt}{0.400pt}}
\multiput(1104.00,493.17)(1.000,-1.000){2}{\rule{0.241pt}{0.400pt}}
\put(1094.0,498.0){\rule[-0.200pt]{0.482pt}{0.400pt}}
\put(1110,492.67){\rule{0.723pt}{0.400pt}}
\multiput(1110.00,492.17)(1.500,1.000){2}{\rule{0.361pt}{0.400pt}}
\put(1113,493.67){\rule{0.482pt}{0.400pt}}
\multiput(1113.00,493.17)(1.000,1.000){2}{\rule{0.241pt}{0.400pt}}
\put(1106.0,493.0){\rule[-0.200pt]{0.964pt}{0.400pt}}
\put(1117,494.67){\rule{0.482pt}{0.400pt}}
\multiput(1117.00,494.17)(1.000,1.000){2}{\rule{0.241pt}{0.400pt}}
\put(1115.0,495.0){\rule[-0.200pt]{0.482pt}{0.400pt}}
\put(1125,494.67){\rule{0.482pt}{0.400pt}}
\multiput(1125.00,495.17)(1.000,-1.000){2}{\rule{0.241pt}{0.400pt}}
\put(1119.0,496.0){\rule[-0.200pt]{1.445pt}{0.400pt}}
\put(1131,493.67){\rule{0.482pt}{0.400pt}}
\multiput(1131.00,494.17)(1.000,-1.000){2}{\rule{0.241pt}{0.400pt}}
\put(1127.0,495.0){\rule[-0.200pt]{0.964pt}{0.400pt}}
\put(1137,493.67){\rule{0.482pt}{0.400pt}}
\multiput(1137.00,493.17)(1.000,1.000){2}{\rule{0.241pt}{0.400pt}}
\put(1133.0,494.0){\rule[-0.200pt]{0.964pt}{0.400pt}}
\put(1139.0,495.0){\rule[-0.200pt]{46.735pt}{0.400pt}}
\put(171.0,131.0){\rule[-0.200pt]{0.400pt}{175.375pt}}
\put(171.0,131.0){\rule[-0.200pt]{305.461pt}{0.400pt}}
\put(1439.0,131.0){\rule[-0.200pt]{0.400pt}{175.375pt}}
\put(171.0,859.0){\rule[-0.200pt]{305.461pt}{0.400pt}}
\end{picture}

	\caption{Approximation of $f(x)=\cos{(80 \pi x)} \exp{(-64 x^2)},$ with $N=3$ and 256 points on $\mathcal{G}^J$. 
	All coefficients are kept.}
\end{figure}
\begin{figure}
	\center
	% GNUPLOT: LaTeX picture
\setlength{\unitlength}{0.240900pt}
\ifx\plotpoint\undefined\newsavebox{\plotpoint}\fi
\sbox{\plotpoint}{\rule[-0.200pt]{0.400pt}{0.400pt}}%
\begin{picture}(1500,900)(0,0)
\sbox{\plotpoint}{\rule[-0.200pt]{0.400pt}{0.400pt}}%
\put(171.0,131.0){\rule[-0.200pt]{4.818pt}{0.400pt}}
\put(151,131){\makebox(0,0)[r]{$-1$}}
\put(1419.0,131.0){\rule[-0.200pt]{4.818pt}{0.400pt}}
\put(171.0,204.0){\rule[-0.200pt]{4.818pt}{0.400pt}}
\put(151,204){\makebox(0,0)[r]{$-0.8$}}
\put(1419.0,204.0){\rule[-0.200pt]{4.818pt}{0.400pt}}
\put(171.0,277.0){\rule[-0.200pt]{4.818pt}{0.400pt}}
\put(151,277){\makebox(0,0)[r]{$-0.6$}}
\put(1419.0,277.0){\rule[-0.200pt]{4.818pt}{0.400pt}}
\put(171.0,349.0){\rule[-0.200pt]{4.818pt}{0.400pt}}
\put(151,349){\makebox(0,0)[r]{$-0.4$}}
\put(1419.0,349.0){\rule[-0.200pt]{4.818pt}{0.400pt}}
\put(171.0,422.0){\rule[-0.200pt]{4.818pt}{0.400pt}}
\put(151,422){\makebox(0,0)[r]{$-0.2$}}
\put(1419.0,422.0){\rule[-0.200pt]{4.818pt}{0.400pt}}
\put(171.0,495.0){\rule[-0.200pt]{4.818pt}{0.400pt}}
\put(151,495){\makebox(0,0)[r]{$0$}}
\put(1419.0,495.0){\rule[-0.200pt]{4.818pt}{0.400pt}}
\put(171.0,568.0){\rule[-0.200pt]{4.818pt}{0.400pt}}
\put(151,568){\makebox(0,0)[r]{$0.2$}}
\put(1419.0,568.0){\rule[-0.200pt]{4.818pt}{0.400pt}}
\put(171.0,641.0){\rule[-0.200pt]{4.818pt}{0.400pt}}
\put(151,641){\makebox(0,0)[r]{$0.4$}}
\put(1419.0,641.0){\rule[-0.200pt]{4.818pt}{0.400pt}}
\put(171.0,713.0){\rule[-0.200pt]{4.818pt}{0.400pt}}
\put(151,713){\makebox(0,0)[r]{$0.6$}}
\put(1419.0,713.0){\rule[-0.200pt]{4.818pt}{0.400pt}}
\put(171.0,786.0){\rule[-0.200pt]{4.818pt}{0.400pt}}
\put(151,786){\makebox(0,0)[r]{$0.8$}}
\put(1419.0,786.0){\rule[-0.200pt]{4.818pt}{0.400pt}}
\put(171.0,859.0){\rule[-0.200pt]{4.818pt}{0.400pt}}
\put(151,859){\makebox(0,0)[r]{$1$}}
\put(1419.0,859.0){\rule[-0.200pt]{4.818pt}{0.400pt}}
\put(171.0,131.0){\rule[-0.200pt]{0.400pt}{4.818pt}}
\put(171,90){\makebox(0,0){$-0.6$}}
\put(171.0,839.0){\rule[-0.200pt]{0.400pt}{4.818pt}}
\put(382.0,131.0){\rule[-0.200pt]{0.400pt}{4.818pt}}
\put(382,90){\makebox(0,0){$-0.4$}}
\put(382.0,839.0){\rule[-0.200pt]{0.400pt}{4.818pt}}
\put(594.0,131.0){\rule[-0.200pt]{0.400pt}{4.818pt}}
\put(594,90){\makebox(0,0){$-0.2$}}
\put(594.0,839.0){\rule[-0.200pt]{0.400pt}{4.818pt}}
\put(805.0,131.0){\rule[-0.200pt]{0.400pt}{4.818pt}}
\put(805,90){\makebox(0,0){$0$}}
\put(805.0,839.0){\rule[-0.200pt]{0.400pt}{4.818pt}}
\put(1016.0,131.0){\rule[-0.200pt]{0.400pt}{4.818pt}}
\put(1016,90){\makebox(0,0){$0.2$}}
\put(1016.0,839.0){\rule[-0.200pt]{0.400pt}{4.818pt}}
\put(1228.0,131.0){\rule[-0.200pt]{0.400pt}{4.818pt}}
\put(1228,90){\makebox(0,0){$0.4$}}
\put(1228.0,839.0){\rule[-0.200pt]{0.400pt}{4.818pt}}
\put(1439.0,131.0){\rule[-0.200pt]{0.400pt}{4.818pt}}
\put(1439,90){\makebox(0,0){$0.6$}}
\put(1439.0,839.0){\rule[-0.200pt]{0.400pt}{4.818pt}}
\put(171.0,131.0){\rule[-0.200pt]{0.400pt}{175.375pt}}
\put(171.0,131.0){\rule[-0.200pt]{305.461pt}{0.400pt}}
\put(1439.0,131.0){\rule[-0.200pt]{0.400pt}{175.375pt}}
\put(171.0,859.0){\rule[-0.200pt]{305.461pt}{0.400pt}}
\put(30,495){\makebox(0,0){$f(x)$}}
\put(805,29){\makebox(0,0){$x$}}
\put(277,495){\usebox{\plotpoint}}
\put(471,493.67){\rule{0.482pt}{0.400pt}}
\multiput(471.00,494.17)(1.000,-1.000){2}{\rule{0.241pt}{0.400pt}}
\put(277.0,495.0){\rule[-0.200pt]{46.735pt}{0.400pt}}
\put(477,493.67){\rule{0.482pt}{0.400pt}}
\multiput(477.00,493.17)(1.000,1.000){2}{\rule{0.241pt}{0.400pt}}
\put(473.0,494.0){\rule[-0.200pt]{0.964pt}{0.400pt}}
\put(483,494.67){\rule{0.482pt}{0.400pt}}
\multiput(483.00,494.17)(1.000,1.000){2}{\rule{0.241pt}{0.400pt}}
\put(479.0,495.0){\rule[-0.200pt]{0.964pt}{0.400pt}}
\put(491,494.67){\rule{0.482pt}{0.400pt}}
\multiput(491.00,495.17)(1.000,-1.000){2}{\rule{0.241pt}{0.400pt}}
\put(485.0,496.0){\rule[-0.200pt]{1.445pt}{0.400pt}}
\put(495,493.67){\rule{0.482pt}{0.400pt}}
\multiput(495.00,494.17)(1.000,-1.000){2}{\rule{0.241pt}{0.400pt}}
\put(497,492.67){\rule{0.723pt}{0.400pt}}
\multiput(497.00,493.17)(1.500,-1.000){2}{\rule{0.361pt}{0.400pt}}
\put(493.0,495.0){\rule[-0.200pt]{0.482pt}{0.400pt}}
\put(504,492.67){\rule{0.482pt}{0.400pt}}
\multiput(504.00,492.17)(1.000,1.000){2}{\rule{0.241pt}{0.400pt}}
\put(506,493.67){\rule{0.482pt}{0.400pt}}
\multiput(506.00,493.17)(1.000,1.000){2}{\rule{0.241pt}{0.400pt}}
\put(508,494.67){\rule{0.482pt}{0.400pt}}
\multiput(508.00,494.17)(1.000,1.000){2}{\rule{0.241pt}{0.400pt}}
\put(510,495.67){\rule{0.482pt}{0.400pt}}
\multiput(510.00,495.17)(1.000,1.000){2}{\rule{0.241pt}{0.400pt}}
\put(512,496.67){\rule{0.482pt}{0.400pt}}
\multiput(512.00,496.17)(1.000,1.000){2}{\rule{0.241pt}{0.400pt}}
\put(500.0,493.0){\rule[-0.200pt]{0.964pt}{0.400pt}}
\put(516,496.67){\rule{0.482pt}{0.400pt}}
\multiput(516.00,497.17)(1.000,-1.000){2}{\rule{0.241pt}{0.400pt}}
\put(518,495.67){\rule{0.482pt}{0.400pt}}
\multiput(518.00,496.17)(1.000,-1.000){2}{\rule{0.241pt}{0.400pt}}
\put(520,494.17){\rule{0.482pt}{0.400pt}}
\multiput(520.00,495.17)(1.000,-2.000){2}{\rule{0.241pt}{0.400pt}}
\put(522,492.17){\rule{0.482pt}{0.400pt}}
\multiput(522.00,493.17)(1.000,-2.000){2}{\rule{0.241pt}{0.400pt}}
\put(524,490.67){\rule{0.482pt}{0.400pt}}
\multiput(524.00,491.17)(1.000,-1.000){2}{\rule{0.241pt}{0.400pt}}
\put(514.0,498.0){\rule[-0.200pt]{0.482pt}{0.400pt}}
\put(531,491.17){\rule{0.482pt}{0.400pt}}
\multiput(531.00,490.17)(1.000,2.000){2}{\rule{0.241pt}{0.400pt}}
\put(533.17,493){\rule{0.400pt}{0.700pt}}
\multiput(532.17,493.00)(2.000,1.547){2}{\rule{0.400pt}{0.350pt}}
\put(535,496.17){\rule{0.482pt}{0.400pt}}
\multiput(535.00,495.17)(1.000,2.000){2}{\rule{0.241pt}{0.400pt}}
\put(537.17,498){\rule{0.400pt}{0.700pt}}
\multiput(536.17,498.00)(2.000,1.547){2}{\rule{0.400pt}{0.350pt}}
\put(539,500.67){\rule{0.482pt}{0.400pt}}
\multiput(539.00,500.17)(1.000,1.000){2}{\rule{0.241pt}{0.400pt}}
\put(541,500.67){\rule{0.482pt}{0.400pt}}
\multiput(541.00,501.17)(1.000,-1.000){2}{\rule{0.241pt}{0.400pt}}
\put(543,499.17){\rule{0.482pt}{0.400pt}}
\multiput(543.00,500.17)(1.000,-2.000){2}{\rule{0.241pt}{0.400pt}}
\put(545.17,496){\rule{0.400pt}{0.700pt}}
\multiput(544.17,497.55)(2.000,-1.547){2}{\rule{0.400pt}{0.350pt}}
\put(547.17,492){\rule{0.400pt}{0.900pt}}
\multiput(546.17,494.13)(2.000,-2.132){2}{\rule{0.400pt}{0.450pt}}
\put(549.17,488){\rule{0.400pt}{0.900pt}}
\multiput(548.17,490.13)(2.000,-2.132){2}{\rule{0.400pt}{0.450pt}}
\put(551,486.17){\rule{0.482pt}{0.400pt}}
\multiput(551.00,487.17)(1.000,-2.000){2}{\rule{0.241pt}{0.400pt}}
\put(553,484.67){\rule{0.482pt}{0.400pt}}
\multiput(553.00,485.17)(1.000,-1.000){2}{\rule{0.241pt}{0.400pt}}
\put(555,485.17){\rule{0.482pt}{0.400pt}}
\multiput(555.00,484.17)(1.000,2.000){2}{\rule{0.241pt}{0.400pt}}
\put(557.17,487){\rule{0.400pt}{1.100pt}}
\multiput(556.17,487.00)(2.000,2.717){2}{\rule{0.400pt}{0.550pt}}
\put(559.17,492){\rule{0.400pt}{1.100pt}}
\multiput(558.17,492.00)(2.000,2.717){2}{\rule{0.400pt}{0.550pt}}
\multiput(561.61,497.00)(0.447,1.132){3}{\rule{0.108pt}{0.900pt}}
\multiput(560.17,497.00)(3.000,4.132){2}{\rule{0.400pt}{0.450pt}}
\put(564.17,503){\rule{0.400pt}{1.100pt}}
\multiput(563.17,503.00)(2.000,2.717){2}{\rule{0.400pt}{0.550pt}}
\put(566,507.67){\rule{0.482pt}{0.400pt}}
\multiput(566.00,507.17)(1.000,1.000){2}{\rule{0.241pt}{0.400pt}}
\put(568,507.67){\rule{0.482pt}{0.400pt}}
\multiput(568.00,508.17)(1.000,-1.000){2}{\rule{0.241pt}{0.400pt}}
\put(570.17,503){\rule{0.400pt}{1.100pt}}
\multiput(569.17,505.72)(2.000,-2.717){2}{\rule{0.400pt}{0.550pt}}
\put(572.17,495){\rule{0.400pt}{1.700pt}}
\multiput(571.17,499.47)(2.000,-4.472){2}{\rule{0.400pt}{0.850pt}}
\put(574.17,487){\rule{0.400pt}{1.700pt}}
\multiput(573.17,491.47)(2.000,-4.472){2}{\rule{0.400pt}{0.850pt}}
\put(576.17,479){\rule{0.400pt}{1.700pt}}
\multiput(575.17,483.47)(2.000,-4.472){2}{\rule{0.400pt}{0.850pt}}
\put(578.17,475){\rule{0.400pt}{0.900pt}}
\multiput(577.17,477.13)(2.000,-2.132){2}{\rule{0.400pt}{0.450pt}}
\put(580,474.67){\rule{0.482pt}{0.400pt}}
\multiput(580.00,474.17)(1.000,1.000){2}{\rule{0.241pt}{0.400pt}}
\put(582.17,476){\rule{0.400pt}{1.100pt}}
\multiput(581.17,476.00)(2.000,2.717){2}{\rule{0.400pt}{0.550pt}}
\put(584.17,481){\rule{0.400pt}{1.900pt}}
\multiput(583.17,481.00)(2.000,5.056){2}{\rule{0.400pt}{0.950pt}}
\put(586.17,490){\rule{0.400pt}{2.500pt}}
\multiput(585.17,490.00)(2.000,6.811){2}{\rule{0.400pt}{1.250pt}}
\put(588.17,502){\rule{0.400pt}{2.300pt}}
\multiput(587.17,502.00)(2.000,6.226){2}{\rule{0.400pt}{1.150pt}}
\put(590.17,513){\rule{0.400pt}{1.700pt}}
\multiput(589.17,513.00)(2.000,4.472){2}{\rule{0.400pt}{0.850pt}}
\put(592,521.17){\rule{0.482pt}{0.400pt}}
\multiput(592.00,520.17)(1.000,2.000){2}{\rule{0.241pt}{0.400pt}}
\multiput(594.61,519.82)(0.447,-0.909){3}{\rule{0.108pt}{0.767pt}}
\multiput(593.17,521.41)(3.000,-3.409){2}{\rule{0.400pt}{0.383pt}}
\put(597.17,507){\rule{0.400pt}{2.300pt}}
\multiput(596.17,513.23)(2.000,-6.226){2}{\rule{0.400pt}{1.150pt}}
\put(599.17,492){\rule{0.400pt}{3.100pt}}
\multiput(598.17,500.57)(2.000,-8.566){2}{\rule{0.400pt}{1.550pt}}
\put(601.17,476){\rule{0.400pt}{3.300pt}}
\multiput(600.17,485.15)(2.000,-9.151){2}{\rule{0.400pt}{1.650pt}}
\put(603.17,463){\rule{0.400pt}{2.700pt}}
\multiput(602.17,470.40)(2.000,-7.396){2}{\rule{0.400pt}{1.350pt}}
\put(605.17,457){\rule{0.400pt}{1.300pt}}
\multiput(604.17,460.30)(2.000,-3.302){2}{\rule{0.400pt}{0.650pt}}
\put(607.17,457){\rule{0.400pt}{0.700pt}}
\multiput(606.17,457.00)(2.000,1.547){2}{\rule{0.400pt}{0.350pt}}
\put(609.17,460){\rule{0.400pt}{2.500pt}}
\multiput(608.17,460.00)(2.000,6.811){2}{\rule{0.400pt}{1.250pt}}
\put(611.17,472){\rule{0.400pt}{3.900pt}}
\multiput(610.17,472.00)(2.000,10.905){2}{\rule{0.400pt}{1.950pt}}
\put(613.17,491){\rule{0.400pt}{4.500pt}}
\multiput(612.17,491.00)(2.000,12.660){2}{\rule{0.400pt}{2.250pt}}
\put(615.17,513){\rule{0.400pt}{3.900pt}}
\multiput(614.17,513.00)(2.000,10.905){2}{\rule{0.400pt}{1.950pt}}
\put(617.17,532){\rule{0.400pt}{2.500pt}}
\multiput(616.17,532.00)(2.000,6.811){2}{\rule{0.400pt}{1.250pt}}
\put(619,543.67){\rule{0.482pt}{0.400pt}}
\multiput(619.00,543.17)(1.000,1.000){2}{\rule{0.241pt}{0.400pt}}
\put(621.17,534){\rule{0.400pt}{2.300pt}}
\multiput(620.17,540.23)(2.000,-6.226){2}{\rule{0.400pt}{1.150pt}}
\put(623.17,512){\rule{0.400pt}{4.500pt}}
\multiput(622.17,524.66)(2.000,-12.660){2}{\rule{0.400pt}{2.250pt}}
\multiput(625.61,495.53)(0.447,-6.267){3}{\rule{0.108pt}{3.967pt}}
\multiput(624.17,503.77)(3.000,-20.767){2}{\rule{0.400pt}{1.983pt}}
\put(628.17,455){\rule{0.400pt}{5.700pt}}
\multiput(627.17,471.17)(2.000,-16.169){2}{\rule{0.400pt}{2.850pt}}
\put(630.17,435){\rule{0.400pt}{4.100pt}}
\multiput(629.17,446.49)(2.000,-11.490){2}{\rule{0.400pt}{2.050pt}}
\put(632.17,428){\rule{0.400pt}{1.500pt}}
\multiput(631.17,431.89)(2.000,-3.887){2}{\rule{0.400pt}{0.750pt}}
\put(634.17,428){\rule{0.400pt}{1.700pt}}
\multiput(633.17,428.00)(2.000,4.472){2}{\rule{0.400pt}{0.850pt}}
\put(636.17,436){\rule{0.400pt}{4.900pt}}
\multiput(635.17,436.00)(2.000,13.830){2}{\rule{0.400pt}{2.450pt}}
\put(638.17,460){\rule{0.400pt}{7.100pt}}
\multiput(637.17,460.00)(2.000,20.264){2}{\rule{0.400pt}{3.550pt}}
\put(640.17,495){\rule{0.400pt}{7.500pt}}
\multiput(639.17,495.00)(2.000,21.433){2}{\rule{0.400pt}{3.750pt}}
\put(642.17,532){\rule{0.400pt}{6.500pt}}
\multiput(641.17,532.00)(2.000,18.509){2}{\rule{0.400pt}{3.250pt}}
\put(644.17,564){\rule{0.400pt}{3.300pt}}
\multiput(643.17,564.00)(2.000,9.151){2}{\rule{0.400pt}{1.650pt}}
\put(646.17,577){\rule{0.400pt}{0.700pt}}
\multiput(645.17,578.55)(2.000,-1.547){2}{\rule{0.400pt}{0.350pt}}
\put(648.17,553){\rule{0.400pt}{4.900pt}}
\multiput(647.17,566.83)(2.000,-13.830){2}{\rule{0.400pt}{2.450pt}}
\put(650.17,514){\rule{0.400pt}{7.900pt}}
\multiput(649.17,536.60)(2.000,-22.603){2}{\rule{0.400pt}{3.950pt}}
\put(652.17,466){\rule{0.400pt}{9.700pt}}
\multiput(651.17,493.87)(2.000,-27.867){2}{\rule{0.400pt}{4.850pt}}
\put(654.17,422){\rule{0.400pt}{8.900pt}}
\multiput(653.17,447.53)(2.000,-25.528){2}{\rule{0.400pt}{4.450pt}}
\put(656.17,393){\rule{0.400pt}{5.900pt}}
\multiput(655.17,409.75)(2.000,-16.754){2}{\rule{0.400pt}{2.950pt}}
\multiput(658.61,389.26)(0.447,-1.132){3}{\rule{0.108pt}{0.900pt}}
\multiput(657.17,391.13)(3.000,-4.132){2}{\rule{0.400pt}{0.450pt}}
\put(661.17,387){\rule{0.400pt}{4.100pt}}
\multiput(660.17,387.00)(2.000,11.490){2}{\rule{0.400pt}{2.050pt}}
\put(663.17,407){\rule{0.400pt}{8.700pt}}
\multiput(662.17,407.00)(2.000,24.943){2}{\rule{0.400pt}{4.350pt}}
\put(665.17,450){\rule{0.400pt}{11.500pt}}
\multiput(664.17,450.00)(2.000,33.131){2}{\rule{0.400pt}{5.750pt}}
\put(667.17,507){\rule{0.400pt}{11.700pt}}
\multiput(666.17,507.00)(2.000,33.716){2}{\rule{0.400pt}{5.850pt}}
\put(669.17,565){\rule{0.400pt}{8.900pt}}
\multiput(668.17,565.00)(2.000,25.528){2}{\rule{0.400pt}{4.450pt}}
\put(671.17,609){\rule{0.400pt}{4.100pt}}
\multiput(670.17,609.00)(2.000,11.490){2}{\rule{0.400pt}{2.050pt}}
\put(673.17,617){\rule{0.400pt}{2.500pt}}
\multiput(672.17,623.81)(2.000,-6.811){2}{\rule{0.400pt}{1.250pt}}
\put(675.17,574){\rule{0.400pt}{8.700pt}}
\multiput(674.17,598.94)(2.000,-24.943){2}{\rule{0.400pt}{4.350pt}}
\put(677.17,509){\rule{0.400pt}{13.100pt}}
\multiput(676.17,546.81)(2.000,-37.810){2}{\rule{0.400pt}{6.550pt}}
\put(679.17,437){\rule{0.400pt}{14.500pt}}
\multiput(678.17,478.90)(2.000,-41.905){2}{\rule{0.400pt}{7.250pt}}
\put(681.17,375){\rule{0.400pt}{12.500pt}}
\multiput(680.17,411.06)(2.000,-36.056){2}{\rule{0.400pt}{6.250pt}}
\put(683.17,338){\rule{0.400pt}{7.500pt}}
\multiput(682.17,359.43)(2.000,-21.433){2}{\rule{0.400pt}{3.750pt}}
\put(685,336.67){\rule{0.482pt}{0.400pt}}
\multiput(685.00,337.17)(1.000,-1.000){2}{\rule{0.241pt}{0.400pt}}
\put(687.17,337){\rule{0.400pt}{7.700pt}}
\multiput(686.17,337.00)(2.000,22.018){2}{\rule{0.400pt}{3.850pt}}
\put(689.17,375){\rule{0.400pt}{14.100pt}}
\multiput(688.17,375.00)(2.000,40.735){2}{\rule{0.400pt}{7.050pt}}
\multiput(691.61,445.00)(0.447,18.770){3}{\rule{0.108pt}{11.433pt}}
\multiput(690.17,445.00)(3.000,61.270){2}{\rule{0.400pt}{5.717pt}}
\put(694.17,530){\rule{0.400pt}{16.300pt}}
\multiput(693.17,530.00)(2.000,47.169){2}{\rule{0.400pt}{8.150pt}}
\put(696.17,611){\rule{0.400pt}{11.700pt}}
\multiput(695.17,611.00)(2.000,33.716){2}{\rule{0.400pt}{5.850pt}}
\put(698.17,669){\rule{0.400pt}{3.700pt}}
\multiput(697.17,669.00)(2.000,10.320){2}{\rule{0.400pt}{1.850pt}}
\put(700.17,659){\rule{0.400pt}{5.700pt}}
\multiput(699.17,675.17)(2.000,-16.169){2}{\rule{0.400pt}{2.850pt}}
\put(702.17,590){\rule{0.400pt}{13.900pt}}
\multiput(701.17,630.15)(2.000,-40.150){2}{\rule{0.400pt}{6.950pt}}
\put(704.17,495){\rule{0.400pt}{19.100pt}}
\multiput(703.17,550.36)(2.000,-55.357){2}{\rule{0.400pt}{9.550pt}}
\put(706.17,395){\rule{0.400pt}{20.100pt}}
\multiput(705.17,453.28)(2.000,-58.281){2}{\rule{0.400pt}{10.050pt}}
\put(708.17,314){\rule{0.400pt}{16.300pt}}
\multiput(707.17,361.17)(2.000,-47.169){2}{\rule{0.400pt}{8.150pt}}
\put(710.17,274){\rule{0.400pt}{8.100pt}}
\multiput(709.17,297.19)(2.000,-23.188){2}{\rule{0.400pt}{4.050pt}}
\put(712.17,274){\rule{0.400pt}{2.300pt}}
\multiput(711.17,274.00)(2.000,6.226){2}{\rule{0.400pt}{1.150pt}}
\put(714.17,285){\rule{0.400pt}{12.700pt}}
\multiput(713.17,285.00)(2.000,36.641){2}{\rule{0.400pt}{6.350pt}}
\put(716.17,348){\rule{0.400pt}{20.300pt}}
\multiput(715.17,348.00)(2.000,58.866){2}{\rule{0.400pt}{10.150pt}}
\put(718.17,449){\rule{0.400pt}{23.300pt}}
\multiput(717.17,449.00)(2.000,67.640){2}{\rule{0.400pt}{11.650pt}}
\put(720.17,565){\rule{0.400pt}{20.900pt}}
\multiput(719.17,565.00)(2.000,60.621){2}{\rule{0.400pt}{10.450pt}}
\multiput(722.61,669.00)(0.447,14.528){3}{\rule{0.108pt}{8.900pt}}
\multiput(721.17,669.00)(3.000,47.528){2}{\rule{0.400pt}{4.450pt}}
\put(725.17,735){\rule{0.400pt}{2.300pt}}
\multiput(724.17,735.00)(2.000,6.226){2}{\rule{0.400pt}{1.150pt}}
\put(727.17,696){\rule{0.400pt}{10.100pt}}
\multiput(726.17,725.04)(2.000,-29.037){2}{\rule{0.400pt}{5.050pt}}
\put(729.17,597){\rule{0.400pt}{19.900pt}}
\multiput(728.17,654.70)(2.000,-57.697){2}{\rule{0.400pt}{9.950pt}}
\put(731.17,469){\rule{0.400pt}{25.700pt}}
\multiput(730.17,543.66)(2.000,-74.658){2}{\rule{0.400pt}{12.850pt}}
\put(733.17,342){\rule{0.400pt}{25.500pt}}
\multiput(732.17,416.07)(2.000,-74.073){2}{\rule{0.400pt}{12.750pt}}
\put(735.17,249){\rule{0.400pt}{18.700pt}}
\multiput(734.17,303.19)(2.000,-54.187){2}{\rule{0.400pt}{9.350pt}}
\put(737.17,212){\rule{0.400pt}{7.500pt}}
\multiput(736.17,233.43)(2.000,-21.433){2}{\rule{0.400pt}{3.750pt}}
\put(739.17,212){\rule{0.400pt}{5.900pt}}
\multiput(738.17,212.00)(2.000,16.754){2}{\rule{0.400pt}{2.950pt}}
\put(741.17,241){\rule{0.400pt}{18.500pt}}
\multiput(740.17,241.00)(2.000,53.602){2}{\rule{0.400pt}{9.250pt}}
\put(743.17,333){\rule{0.400pt}{26.700pt}}
\multiput(742.17,333.00)(2.000,77.583){2}{\rule{0.400pt}{13.350pt}}
\put(745.17,466){\rule{0.400pt}{28.900pt}}
\multiput(744.17,466.00)(2.000,84.017){2}{\rule{0.400pt}{14.450pt}}
\put(747.17,610){\rule{0.400pt}{24.300pt}}
\multiput(746.17,610.00)(2.000,70.564){2}{\rule{0.400pt}{12.150pt}}
\put(749.17,731){\rule{0.400pt}{13.500pt}}
\multiput(748.17,731.00)(2.000,38.980){2}{\rule{0.400pt}{6.750pt}}
\put(751.17,794){\rule{0.400pt}{0.900pt}}
\multiput(750.17,796.13)(2.000,-2.132){2}{\rule{0.400pt}{0.450pt}}
\put(753.17,719){\rule{0.400pt}{15.100pt}}
\multiput(752.17,762.66)(2.000,-43.659){2}{\rule{0.400pt}{7.550pt}}
\multiput(755.61,646.08)(0.447,-29.040){3}{\rule{0.108pt}{17.567pt}}
\multiput(754.17,682.54)(3.000,-94.540){2}{\rule{0.400pt}{8.783pt}}
\put(758.17,432){\rule{0.400pt}{31.300pt}}
\multiput(757.17,523.04)(2.000,-91.035){2}{\rule{0.400pt}{15.650pt}}
\put(760.17,288){\rule{0.400pt}{28.900pt}}
\multiput(759.17,372.02)(2.000,-84.017){2}{\rule{0.400pt}{14.450pt}}
\put(762.17,190){\rule{0.400pt}{19.700pt}}
\multiput(761.17,247.11)(2.000,-57.112){2}{\rule{0.400pt}{9.850pt}}
\put(764.17,163){\rule{0.400pt}{5.500pt}}
\multiput(763.17,178.58)(2.000,-15.584){2}{\rule{0.400pt}{2.750pt}}
\put(766.17,163){\rule{0.400pt}{10.500pt}}
\multiput(765.17,163.00)(2.000,30.207){2}{\rule{0.400pt}{5.250pt}}
\put(768.17,215){\rule{0.400pt}{24.100pt}}
\multiput(767.17,215.00)(2.000,69.979){2}{\rule{0.400pt}{12.050pt}}
\put(770.17,335){\rule{0.400pt}{32.100pt}}
\multiput(769.17,335.00)(2.000,93.375){2}{\rule{0.400pt}{16.050pt}}
\put(772.17,495){\rule{0.400pt}{32.500pt}}
\multiput(771.17,495.00)(2.000,94.545){2}{\rule{0.400pt}{16.250pt}}
\put(774.17,657){\rule{0.400pt}{25.500pt}}
\multiput(773.17,657.00)(2.000,74.073){2}{\rule{0.400pt}{12.750pt}}
\put(776.17,784){\rule{0.400pt}{11.900pt}}
\multiput(775.17,784.00)(2.000,34.301){2}{\rule{0.400pt}{5.950pt}}
\put(778.17,820){\rule{0.400pt}{4.700pt}}
\multiput(777.17,833.24)(2.000,-13.245){2}{\rule{0.400pt}{2.350pt}}
\put(780.17,719){\rule{0.400pt}{20.300pt}}
\multiput(779.17,777.87)(2.000,-58.866){2}{\rule{0.400pt}{10.150pt}}
\put(782.17,564){\rule{0.400pt}{31.100pt}}
\multiput(781.17,654.45)(2.000,-90.450){2}{\rule{0.400pt}{15.550pt}}
\put(784.17,391){\rule{0.400pt}{34.700pt}}
\multiput(783.17,491.98)(2.000,-100.978){2}{\rule{0.400pt}{17.350pt}}
\put(786.17,242){\rule{0.400pt}{29.900pt}}
\multiput(785.17,328.94)(2.000,-86.941){2}{\rule{0.400pt}{14.950pt}}
\multiput(788.61,191.22)(0.447,-20.109){3}{\rule{0.108pt}{12.233pt}}
\multiput(787.17,216.61)(3.000,-65.609){2}{\rule{0.400pt}{6.117pt}}
\put(791.17,141){\rule{0.400pt}{2.100pt}}
\multiput(790.17,146.64)(2.000,-5.641){2}{\rule{0.400pt}{1.050pt}}
\put(793.17,141){\rule{0.400pt}{14.900pt}}
\multiput(792.17,141.00)(2.000,43.074){2}{\rule{0.400pt}{7.450pt}}
\put(795.17,215){\rule{0.400pt}{28.300pt}}
\multiput(794.17,215.00)(2.000,82.262){2}{\rule{0.400pt}{14.150pt}}
\put(797.17,356){\rule{0.400pt}{35.100pt}}
\multiput(796.17,356.00)(2.000,102.148){2}{\rule{0.400pt}{17.550pt}}
\put(799.17,531){\rule{0.400pt}{33.300pt}}
\multiput(798.17,531.00)(2.000,96.884){2}{\rule{0.400pt}{16.650pt}}
\put(801.17,697){\rule{0.400pt}{23.900pt}}
\multiput(800.17,697.00)(2.000,69.394){2}{\rule{0.400pt}{11.950pt}}
\put(803.17,816){\rule{0.400pt}{8.700pt}}
\multiput(802.17,816.00)(2.000,24.943){2}{\rule{0.400pt}{4.350pt}}
\put(805.17,816){\rule{0.400pt}{8.700pt}}
\multiput(804.17,840.94)(2.000,-24.943){2}{\rule{0.400pt}{4.350pt}}
\put(807.17,697){\rule{0.400pt}{23.900pt}}
\multiput(806.17,766.39)(2.000,-69.394){2}{\rule{0.400pt}{11.950pt}}
\put(809.17,531){\rule{0.400pt}{33.300pt}}
\multiput(808.17,627.88)(2.000,-96.884){2}{\rule{0.400pt}{16.650pt}}
\put(811.17,356){\rule{0.400pt}{35.100pt}}
\multiput(810.17,458.15)(2.000,-102.148){2}{\rule{0.400pt}{17.550pt}}
\put(813.17,215){\rule{0.400pt}{28.300pt}}
\multiput(812.17,297.26)(2.000,-82.262){2}{\rule{0.400pt}{14.150pt}}
\put(815.17,141){\rule{0.400pt}{14.900pt}}
\multiput(814.17,184.07)(2.000,-43.074){2}{\rule{0.400pt}{7.450pt}}
\put(817.17,141){\rule{0.400pt}{2.100pt}}
\multiput(816.17,141.00)(2.000,5.641){2}{\rule{0.400pt}{1.050pt}}
\multiput(819.61,151.00)(0.447,20.109){3}{\rule{0.108pt}{12.233pt}}
\multiput(818.17,151.00)(3.000,65.609){2}{\rule{0.400pt}{6.117pt}}
\put(822.17,242){\rule{0.400pt}{29.900pt}}
\multiput(821.17,242.00)(2.000,86.941){2}{\rule{0.400pt}{14.950pt}}
\put(824.17,391){\rule{0.400pt}{34.700pt}}
\multiput(823.17,391.00)(2.000,100.978){2}{\rule{0.400pt}{17.350pt}}
\put(826.17,564){\rule{0.400pt}{31.100pt}}
\multiput(825.17,564.00)(2.000,90.450){2}{\rule{0.400pt}{15.550pt}}
\put(828.17,719){\rule{0.400pt}{20.300pt}}
\multiput(827.17,719.00)(2.000,58.866){2}{\rule{0.400pt}{10.150pt}}
\put(830.17,820){\rule{0.400pt}{4.700pt}}
\multiput(829.17,820.00)(2.000,13.245){2}{\rule{0.400pt}{2.350pt}}
\put(832.17,784){\rule{0.400pt}{11.900pt}}
\multiput(831.17,818.30)(2.000,-34.301){2}{\rule{0.400pt}{5.950pt}}
\put(834.17,657){\rule{0.400pt}{25.500pt}}
\multiput(833.17,731.07)(2.000,-74.073){2}{\rule{0.400pt}{12.750pt}}
\put(836.17,495){\rule{0.400pt}{32.500pt}}
\multiput(835.17,589.54)(2.000,-94.545){2}{\rule{0.400pt}{16.250pt}}
\put(838.17,335){\rule{0.400pt}{32.100pt}}
\multiput(837.17,428.37)(2.000,-93.375){2}{\rule{0.400pt}{16.050pt}}
\put(840.17,215){\rule{0.400pt}{24.100pt}}
\multiput(839.17,284.98)(2.000,-69.979){2}{\rule{0.400pt}{12.050pt}}
\put(842.17,163){\rule{0.400pt}{10.500pt}}
\multiput(841.17,193.21)(2.000,-30.207){2}{\rule{0.400pt}{5.250pt}}
\put(844.17,163){\rule{0.400pt}{5.500pt}}
\multiput(843.17,163.00)(2.000,15.584){2}{\rule{0.400pt}{2.750pt}}
\put(846.17,190){\rule{0.400pt}{19.700pt}}
\multiput(845.17,190.00)(2.000,57.112){2}{\rule{0.400pt}{9.850pt}}
\put(848.17,288){\rule{0.400pt}{28.900pt}}
\multiput(847.17,288.00)(2.000,84.017){2}{\rule{0.400pt}{14.450pt}}
\put(850.17,432){\rule{0.400pt}{31.300pt}}
\multiput(849.17,432.00)(2.000,91.035){2}{\rule{0.400pt}{15.650pt}}
\multiput(852.61,588.00)(0.447,29.040){3}{\rule{0.108pt}{17.567pt}}
\multiput(851.17,588.00)(3.000,94.540){2}{\rule{0.400pt}{8.783pt}}
\put(855.17,719){\rule{0.400pt}{15.100pt}}
\multiput(854.17,719.00)(2.000,43.659){2}{\rule{0.400pt}{7.550pt}}
\put(857.17,794){\rule{0.400pt}{0.900pt}}
\multiput(856.17,794.00)(2.000,2.132){2}{\rule{0.400pt}{0.450pt}}
\put(859.17,731){\rule{0.400pt}{13.500pt}}
\multiput(858.17,769.98)(2.000,-38.980){2}{\rule{0.400pt}{6.750pt}}
\put(861.17,610){\rule{0.400pt}{24.300pt}}
\multiput(860.17,680.56)(2.000,-70.564){2}{\rule{0.400pt}{12.150pt}}
\put(863.17,466){\rule{0.400pt}{28.900pt}}
\multiput(862.17,550.02)(2.000,-84.017){2}{\rule{0.400pt}{14.450pt}}
\put(865.17,333){\rule{0.400pt}{26.700pt}}
\multiput(864.17,410.58)(2.000,-77.583){2}{\rule{0.400pt}{13.350pt}}
\put(867.17,241){\rule{0.400pt}{18.500pt}}
\multiput(866.17,294.60)(2.000,-53.602){2}{\rule{0.400pt}{9.250pt}}
\put(869.17,212){\rule{0.400pt}{5.900pt}}
\multiput(868.17,228.75)(2.000,-16.754){2}{\rule{0.400pt}{2.950pt}}
\put(871.17,212){\rule{0.400pt}{7.500pt}}
\multiput(870.17,212.00)(2.000,21.433){2}{\rule{0.400pt}{3.750pt}}
\put(873.17,249){\rule{0.400pt}{18.700pt}}
\multiput(872.17,249.00)(2.000,54.187){2}{\rule{0.400pt}{9.350pt}}
\put(875.17,342){\rule{0.400pt}{25.500pt}}
\multiput(874.17,342.00)(2.000,74.073){2}{\rule{0.400pt}{12.750pt}}
\put(877.17,469){\rule{0.400pt}{25.700pt}}
\multiput(876.17,469.00)(2.000,74.658){2}{\rule{0.400pt}{12.850pt}}
\put(879.17,597){\rule{0.400pt}{19.900pt}}
\multiput(878.17,597.00)(2.000,57.697){2}{\rule{0.400pt}{9.950pt}}
\put(881.17,696){\rule{0.400pt}{10.100pt}}
\multiput(880.17,696.00)(2.000,29.037){2}{\rule{0.400pt}{5.050pt}}
\put(883.17,735){\rule{0.400pt}{2.300pt}}
\multiput(882.17,741.23)(2.000,-6.226){2}{\rule{0.400pt}{1.150pt}}
\multiput(885.61,698.06)(0.447,-14.528){3}{\rule{0.108pt}{8.900pt}}
\multiput(884.17,716.53)(3.000,-47.528){2}{\rule{0.400pt}{4.450pt}}
\put(888.17,565){\rule{0.400pt}{20.900pt}}
\multiput(887.17,625.62)(2.000,-60.621){2}{\rule{0.400pt}{10.450pt}}
\put(890.17,449){\rule{0.400pt}{23.300pt}}
\multiput(889.17,516.64)(2.000,-67.640){2}{\rule{0.400pt}{11.650pt}}
\put(892.17,348){\rule{0.400pt}{20.300pt}}
\multiput(891.17,406.87)(2.000,-58.866){2}{\rule{0.400pt}{10.150pt}}
\put(894.17,285){\rule{0.400pt}{12.700pt}}
\multiput(893.17,321.64)(2.000,-36.641){2}{\rule{0.400pt}{6.350pt}}
\put(896.17,274){\rule{0.400pt}{2.300pt}}
\multiput(895.17,280.23)(2.000,-6.226){2}{\rule{0.400pt}{1.150pt}}
\put(898.17,274){\rule{0.400pt}{8.100pt}}
\multiput(897.17,274.00)(2.000,23.188){2}{\rule{0.400pt}{4.050pt}}
\put(900.17,314){\rule{0.400pt}{16.300pt}}
\multiput(899.17,314.00)(2.000,47.169){2}{\rule{0.400pt}{8.150pt}}
\put(902.17,395){\rule{0.400pt}{20.100pt}}
\multiput(901.17,395.00)(2.000,58.281){2}{\rule{0.400pt}{10.050pt}}
\put(904.17,495){\rule{0.400pt}{19.100pt}}
\multiput(903.17,495.00)(2.000,55.357){2}{\rule{0.400pt}{9.550pt}}
\put(906.17,590){\rule{0.400pt}{13.900pt}}
\multiput(905.17,590.00)(2.000,40.150){2}{\rule{0.400pt}{6.950pt}}
\put(908.17,659){\rule{0.400pt}{5.700pt}}
\multiput(907.17,659.00)(2.000,16.169){2}{\rule{0.400pt}{2.850pt}}
\put(910.17,669){\rule{0.400pt}{3.700pt}}
\multiput(909.17,679.32)(2.000,-10.320){2}{\rule{0.400pt}{1.850pt}}
\put(912.17,611){\rule{0.400pt}{11.700pt}}
\multiput(911.17,644.72)(2.000,-33.716){2}{\rule{0.400pt}{5.850pt}}
\put(914.17,530){\rule{0.400pt}{16.300pt}}
\multiput(913.17,577.17)(2.000,-47.169){2}{\rule{0.400pt}{8.150pt}}
\multiput(916.61,482.54)(0.447,-18.770){3}{\rule{0.108pt}{11.433pt}}
\multiput(915.17,506.27)(3.000,-61.270){2}{\rule{0.400pt}{5.717pt}}
\put(919.17,375){\rule{0.400pt}{14.100pt}}
\multiput(918.17,415.73)(2.000,-40.735){2}{\rule{0.400pt}{7.050pt}}
\put(921.17,337){\rule{0.400pt}{7.700pt}}
\multiput(920.17,359.02)(2.000,-22.018){2}{\rule{0.400pt}{3.850pt}}
\put(923,336.67){\rule{0.482pt}{0.400pt}}
\multiput(923.00,336.17)(1.000,1.000){2}{\rule{0.241pt}{0.400pt}}
\put(925.17,338){\rule{0.400pt}{7.500pt}}
\multiput(924.17,338.00)(2.000,21.433){2}{\rule{0.400pt}{3.750pt}}
\put(927.17,375){\rule{0.400pt}{12.500pt}}
\multiput(926.17,375.00)(2.000,36.056){2}{\rule{0.400pt}{6.250pt}}
\put(929.17,437){\rule{0.400pt}{14.500pt}}
\multiput(928.17,437.00)(2.000,41.905){2}{\rule{0.400pt}{7.250pt}}
\put(931.17,509){\rule{0.400pt}{13.100pt}}
\multiput(930.17,509.00)(2.000,37.810){2}{\rule{0.400pt}{6.550pt}}
\put(933.17,574){\rule{0.400pt}{8.700pt}}
\multiput(932.17,574.00)(2.000,24.943){2}{\rule{0.400pt}{4.350pt}}
\put(935.17,617){\rule{0.400pt}{2.500pt}}
\multiput(934.17,617.00)(2.000,6.811){2}{\rule{0.400pt}{1.250pt}}
\put(937.17,609){\rule{0.400pt}{4.100pt}}
\multiput(936.17,620.49)(2.000,-11.490){2}{\rule{0.400pt}{2.050pt}}
\put(939.17,565){\rule{0.400pt}{8.900pt}}
\multiput(938.17,590.53)(2.000,-25.528){2}{\rule{0.400pt}{4.450pt}}
\put(941.17,507){\rule{0.400pt}{11.700pt}}
\multiput(940.17,540.72)(2.000,-33.716){2}{\rule{0.400pt}{5.850pt}}
\put(943.17,450){\rule{0.400pt}{11.500pt}}
\multiput(942.17,483.13)(2.000,-33.131){2}{\rule{0.400pt}{5.750pt}}
\put(945.17,407){\rule{0.400pt}{8.700pt}}
\multiput(944.17,431.94)(2.000,-24.943){2}{\rule{0.400pt}{4.350pt}}
\put(947.17,387){\rule{0.400pt}{4.100pt}}
\multiput(946.17,398.49)(2.000,-11.490){2}{\rule{0.400pt}{2.050pt}}
\multiput(949.61,387.00)(0.447,1.132){3}{\rule{0.108pt}{0.900pt}}
\multiput(948.17,387.00)(3.000,4.132){2}{\rule{0.400pt}{0.450pt}}
\put(952.17,393){\rule{0.400pt}{5.900pt}}
\multiput(951.17,393.00)(2.000,16.754){2}{\rule{0.400pt}{2.950pt}}
\put(954.17,422){\rule{0.400pt}{8.900pt}}
\multiput(953.17,422.00)(2.000,25.528){2}{\rule{0.400pt}{4.450pt}}
\put(956.17,466){\rule{0.400pt}{9.700pt}}
\multiput(955.17,466.00)(2.000,27.867){2}{\rule{0.400pt}{4.850pt}}
\put(958.17,514){\rule{0.400pt}{7.900pt}}
\multiput(957.17,514.00)(2.000,22.603){2}{\rule{0.400pt}{3.950pt}}
\put(960.17,553){\rule{0.400pt}{4.900pt}}
\multiput(959.17,553.00)(2.000,13.830){2}{\rule{0.400pt}{2.450pt}}
\put(962.17,577){\rule{0.400pt}{0.700pt}}
\multiput(961.17,577.00)(2.000,1.547){2}{\rule{0.400pt}{0.350pt}}
\put(964.17,564){\rule{0.400pt}{3.300pt}}
\multiput(963.17,573.15)(2.000,-9.151){2}{\rule{0.400pt}{1.650pt}}
\put(966.17,532){\rule{0.400pt}{6.500pt}}
\multiput(965.17,550.51)(2.000,-18.509){2}{\rule{0.400pt}{3.250pt}}
\put(968.17,495){\rule{0.400pt}{7.500pt}}
\multiput(967.17,516.43)(2.000,-21.433){2}{\rule{0.400pt}{3.750pt}}
\put(970.17,460){\rule{0.400pt}{7.100pt}}
\multiput(969.17,480.26)(2.000,-20.264){2}{\rule{0.400pt}{3.550pt}}
\put(972.17,436){\rule{0.400pt}{4.900pt}}
\multiput(971.17,449.83)(2.000,-13.830){2}{\rule{0.400pt}{2.450pt}}
\put(974.17,428){\rule{0.400pt}{1.700pt}}
\multiput(973.17,432.47)(2.000,-4.472){2}{\rule{0.400pt}{0.850pt}}
\put(976.17,428){\rule{0.400pt}{1.500pt}}
\multiput(975.17,428.00)(2.000,3.887){2}{\rule{0.400pt}{0.750pt}}
\put(978.17,435){\rule{0.400pt}{4.100pt}}
\multiput(977.17,435.00)(2.000,11.490){2}{\rule{0.400pt}{2.050pt}}
\put(980.17,455){\rule{0.400pt}{5.700pt}}
\multiput(979.17,455.00)(2.000,16.169){2}{\rule{0.400pt}{2.850pt}}
\multiput(982.61,483.00)(0.447,6.267){3}{\rule{0.108pt}{3.967pt}}
\multiput(981.17,483.00)(3.000,20.767){2}{\rule{0.400pt}{1.983pt}}
\put(985.17,512){\rule{0.400pt}{4.500pt}}
\multiput(984.17,512.00)(2.000,12.660){2}{\rule{0.400pt}{2.250pt}}
\put(987.17,534){\rule{0.400pt}{2.300pt}}
\multiput(986.17,534.00)(2.000,6.226){2}{\rule{0.400pt}{1.150pt}}
\put(989,543.67){\rule{0.482pt}{0.400pt}}
\multiput(989.00,544.17)(1.000,-1.000){2}{\rule{0.241pt}{0.400pt}}
\put(991.17,532){\rule{0.400pt}{2.500pt}}
\multiput(990.17,538.81)(2.000,-6.811){2}{\rule{0.400pt}{1.250pt}}
\put(993.17,513){\rule{0.400pt}{3.900pt}}
\multiput(992.17,523.91)(2.000,-10.905){2}{\rule{0.400pt}{1.950pt}}
\put(995.17,491){\rule{0.400pt}{4.500pt}}
\multiput(994.17,503.66)(2.000,-12.660){2}{\rule{0.400pt}{2.250pt}}
\put(997.17,472){\rule{0.400pt}{3.900pt}}
\multiput(996.17,482.91)(2.000,-10.905){2}{\rule{0.400pt}{1.950pt}}
\put(999.17,460){\rule{0.400pt}{2.500pt}}
\multiput(998.17,466.81)(2.000,-6.811){2}{\rule{0.400pt}{1.250pt}}
\put(1001.17,457){\rule{0.400pt}{0.700pt}}
\multiput(1000.17,458.55)(2.000,-1.547){2}{\rule{0.400pt}{0.350pt}}
\put(1003.17,457){\rule{0.400pt}{1.300pt}}
\multiput(1002.17,457.00)(2.000,3.302){2}{\rule{0.400pt}{0.650pt}}
\put(1005.17,463){\rule{0.400pt}{2.700pt}}
\multiput(1004.17,463.00)(2.000,7.396){2}{\rule{0.400pt}{1.350pt}}
\put(1007.17,476){\rule{0.400pt}{3.300pt}}
\multiput(1006.17,476.00)(2.000,9.151){2}{\rule{0.400pt}{1.650pt}}
\put(1009.17,492){\rule{0.400pt}{3.100pt}}
\multiput(1008.17,492.00)(2.000,8.566){2}{\rule{0.400pt}{1.550pt}}
\put(1011.17,507){\rule{0.400pt}{2.300pt}}
\multiput(1010.17,507.00)(2.000,6.226){2}{\rule{0.400pt}{1.150pt}}
\multiput(1013.61,518.00)(0.447,0.909){3}{\rule{0.108pt}{0.767pt}}
\multiput(1012.17,518.00)(3.000,3.409){2}{\rule{0.400pt}{0.383pt}}
\put(1016,521.17){\rule{0.482pt}{0.400pt}}
\multiput(1016.00,522.17)(1.000,-2.000){2}{\rule{0.241pt}{0.400pt}}
\put(1018.17,513){\rule{0.400pt}{1.700pt}}
\multiput(1017.17,517.47)(2.000,-4.472){2}{\rule{0.400pt}{0.850pt}}
\put(1020.17,502){\rule{0.400pt}{2.300pt}}
\multiput(1019.17,508.23)(2.000,-6.226){2}{\rule{0.400pt}{1.150pt}}
\put(1022.17,490){\rule{0.400pt}{2.500pt}}
\multiput(1021.17,496.81)(2.000,-6.811){2}{\rule{0.400pt}{1.250pt}}
\put(1024.17,481){\rule{0.400pt}{1.900pt}}
\multiput(1023.17,486.06)(2.000,-5.056){2}{\rule{0.400pt}{0.950pt}}
\put(1026.17,476){\rule{0.400pt}{1.100pt}}
\multiput(1025.17,478.72)(2.000,-2.717){2}{\rule{0.400pt}{0.550pt}}
\put(1028,474.67){\rule{0.482pt}{0.400pt}}
\multiput(1028.00,475.17)(1.000,-1.000){2}{\rule{0.241pt}{0.400pt}}
\put(1030.17,475){\rule{0.400pt}{0.900pt}}
\multiput(1029.17,475.00)(2.000,2.132){2}{\rule{0.400pt}{0.450pt}}
\put(1032.17,479){\rule{0.400pt}{1.700pt}}
\multiput(1031.17,479.00)(2.000,4.472){2}{\rule{0.400pt}{0.850pt}}
\put(1034.17,487){\rule{0.400pt}{1.700pt}}
\multiput(1033.17,487.00)(2.000,4.472){2}{\rule{0.400pt}{0.850pt}}
\put(1036.17,495){\rule{0.400pt}{1.700pt}}
\multiput(1035.17,495.00)(2.000,4.472){2}{\rule{0.400pt}{0.850pt}}
\put(1038.17,503){\rule{0.400pt}{1.100pt}}
\multiput(1037.17,503.00)(2.000,2.717){2}{\rule{0.400pt}{0.550pt}}
\put(1040,507.67){\rule{0.482pt}{0.400pt}}
\multiput(1040.00,507.17)(1.000,1.000){2}{\rule{0.241pt}{0.400pt}}
\put(1042,507.67){\rule{0.482pt}{0.400pt}}
\multiput(1042.00,508.17)(1.000,-1.000){2}{\rule{0.241pt}{0.400pt}}
\put(1044.17,503){\rule{0.400pt}{1.100pt}}
\multiput(1043.17,505.72)(2.000,-2.717){2}{\rule{0.400pt}{0.550pt}}
\multiput(1046.61,499.26)(0.447,-1.132){3}{\rule{0.108pt}{0.900pt}}
\multiput(1045.17,501.13)(3.000,-4.132){2}{\rule{0.400pt}{0.450pt}}
\put(1049.17,492){\rule{0.400pt}{1.100pt}}
\multiput(1048.17,494.72)(2.000,-2.717){2}{\rule{0.400pt}{0.550pt}}
\put(1051.17,487){\rule{0.400pt}{1.100pt}}
\multiput(1050.17,489.72)(2.000,-2.717){2}{\rule{0.400pt}{0.550pt}}
\put(1053,485.17){\rule{0.482pt}{0.400pt}}
\multiput(1053.00,486.17)(1.000,-2.000){2}{\rule{0.241pt}{0.400pt}}
\put(1055,484.67){\rule{0.482pt}{0.400pt}}
\multiput(1055.00,484.17)(1.000,1.000){2}{\rule{0.241pt}{0.400pt}}
\put(1057,486.17){\rule{0.482pt}{0.400pt}}
\multiput(1057.00,485.17)(1.000,2.000){2}{\rule{0.241pt}{0.400pt}}
\put(1059.17,488){\rule{0.400pt}{0.900pt}}
\multiput(1058.17,488.00)(2.000,2.132){2}{\rule{0.400pt}{0.450pt}}
\put(1061.17,492){\rule{0.400pt}{0.900pt}}
\multiput(1060.17,492.00)(2.000,2.132){2}{\rule{0.400pt}{0.450pt}}
\put(1063.17,496){\rule{0.400pt}{0.700pt}}
\multiput(1062.17,496.00)(2.000,1.547){2}{\rule{0.400pt}{0.350pt}}
\put(1065,499.17){\rule{0.482pt}{0.400pt}}
\multiput(1065.00,498.17)(1.000,2.000){2}{\rule{0.241pt}{0.400pt}}
\put(1067,500.67){\rule{0.482pt}{0.400pt}}
\multiput(1067.00,500.17)(1.000,1.000){2}{\rule{0.241pt}{0.400pt}}
\put(1069,500.67){\rule{0.482pt}{0.400pt}}
\multiput(1069.00,501.17)(1.000,-1.000){2}{\rule{0.241pt}{0.400pt}}
\put(1071.17,498){\rule{0.400pt}{0.700pt}}
\multiput(1070.17,499.55)(2.000,-1.547){2}{\rule{0.400pt}{0.350pt}}
\put(1073,496.17){\rule{0.482pt}{0.400pt}}
\multiput(1073.00,497.17)(1.000,-2.000){2}{\rule{0.241pt}{0.400pt}}
\put(1075.17,493){\rule{0.400pt}{0.700pt}}
\multiput(1074.17,494.55)(2.000,-1.547){2}{\rule{0.400pt}{0.350pt}}
\put(1077,491.17){\rule{0.482pt}{0.400pt}}
\multiput(1077.00,492.17)(1.000,-2.000){2}{\rule{0.241pt}{0.400pt}}
\put(526.0,491.0){\rule[-0.200pt]{1.204pt}{0.400pt}}
\put(1084,490.67){\rule{0.482pt}{0.400pt}}
\multiput(1084.00,490.17)(1.000,1.000){2}{\rule{0.241pt}{0.400pt}}
\put(1086,492.17){\rule{0.482pt}{0.400pt}}
\multiput(1086.00,491.17)(1.000,2.000){2}{\rule{0.241pt}{0.400pt}}
\put(1088,494.17){\rule{0.482pt}{0.400pt}}
\multiput(1088.00,493.17)(1.000,2.000){2}{\rule{0.241pt}{0.400pt}}
\put(1090,495.67){\rule{0.482pt}{0.400pt}}
\multiput(1090.00,495.17)(1.000,1.000){2}{\rule{0.241pt}{0.400pt}}
\put(1092,496.67){\rule{0.482pt}{0.400pt}}
\multiput(1092.00,496.17)(1.000,1.000){2}{\rule{0.241pt}{0.400pt}}
\put(1079.0,491.0){\rule[-0.200pt]{1.204pt}{0.400pt}}
\put(1096,496.67){\rule{0.482pt}{0.400pt}}
\multiput(1096.00,497.17)(1.000,-1.000){2}{\rule{0.241pt}{0.400pt}}
\put(1098,495.67){\rule{0.482pt}{0.400pt}}
\multiput(1098.00,496.17)(1.000,-1.000){2}{\rule{0.241pt}{0.400pt}}
\put(1100,494.67){\rule{0.482pt}{0.400pt}}
\multiput(1100.00,495.17)(1.000,-1.000){2}{\rule{0.241pt}{0.400pt}}
\put(1102,493.67){\rule{0.482pt}{0.400pt}}
\multiput(1102.00,494.17)(1.000,-1.000){2}{\rule{0.241pt}{0.400pt}}
\put(1104,492.67){\rule{0.482pt}{0.400pt}}
\multiput(1104.00,493.17)(1.000,-1.000){2}{\rule{0.241pt}{0.400pt}}
\put(1094.0,498.0){\rule[-0.200pt]{0.482pt}{0.400pt}}
\put(1110,492.67){\rule{0.723pt}{0.400pt}}
\multiput(1110.00,492.17)(1.500,1.000){2}{\rule{0.361pt}{0.400pt}}
\put(1113,493.67){\rule{0.482pt}{0.400pt}}
\multiput(1113.00,493.17)(1.000,1.000){2}{\rule{0.241pt}{0.400pt}}
\put(1106.0,493.0){\rule[-0.200pt]{0.964pt}{0.400pt}}
\put(1117,494.67){\rule{0.482pt}{0.400pt}}
\multiput(1117.00,494.17)(1.000,1.000){2}{\rule{0.241pt}{0.400pt}}
\put(1115.0,495.0){\rule[-0.200pt]{0.482pt}{0.400pt}}
\put(1125,494.67){\rule{0.482pt}{0.400pt}}
\multiput(1125.00,495.17)(1.000,-1.000){2}{\rule{0.241pt}{0.400pt}}
\put(1119.0,496.0){\rule[-0.200pt]{1.445pt}{0.400pt}}
\put(1131,493.67){\rule{0.482pt}{0.400pt}}
\multiput(1131.00,494.17)(1.000,-1.000){2}{\rule{0.241pt}{0.400pt}}
\put(1127.0,495.0){\rule[-0.200pt]{0.964pt}{0.400pt}}
\put(1137,493.67){\rule{0.482pt}{0.400pt}}
\multiput(1137.00,493.17)(1.000,1.000){2}{\rule{0.241pt}{0.400pt}}
\put(1133.0,494.0){\rule[-0.200pt]{0.964pt}{0.400pt}}
\put(1139.0,495.0){\rule[-0.200pt]{46.735pt}{0.400pt}}
\put(171.0,131.0){\rule[-0.200pt]{0.400pt}{175.375pt}}
\put(171.0,131.0){\rule[-0.200pt]{305.461pt}{0.400pt}}
\put(1439.0,131.0){\rule[-0.200pt]{0.400pt}{175.375pt}}
\put(171.0,859.0){\rule[-0.200pt]{305.461pt}{0.400pt}}
\end{picture}

	\caption{Same function with a tolerance of $\epsilon = 10^{-5}$.} 
\end{figure}
\begin{figure}
	\center
	% GNUPLOT: LaTeX picture
\setlength{\unitlength}{0.240900pt}
\ifx\plotpoint\undefined\newsavebox{\plotpoint}\fi
\sbox{\plotpoint}{\rule[-0.200pt]{0.400pt}{0.400pt}}%
\begin{picture}(1500,900)(0,0)
\sbox{\plotpoint}{\rule[-0.200pt]{0.400pt}{0.400pt}}%
\put(171.0,131.0){\rule[-0.200pt]{4.818pt}{0.400pt}}
\put(151,131){\makebox(0,0)[r]{$-1$}}
\put(1419.0,131.0){\rule[-0.200pt]{4.818pt}{0.400pt}}
\put(171.0,204.0){\rule[-0.200pt]{4.818pt}{0.400pt}}
\put(151,204){\makebox(0,0)[r]{$-0.8$}}
\put(1419.0,204.0){\rule[-0.200pt]{4.818pt}{0.400pt}}
\put(171.0,277.0){\rule[-0.200pt]{4.818pt}{0.400pt}}
\put(151,277){\makebox(0,0)[r]{$-0.6$}}
\put(1419.0,277.0){\rule[-0.200pt]{4.818pt}{0.400pt}}
\put(171.0,349.0){\rule[-0.200pt]{4.818pt}{0.400pt}}
\put(151,349){\makebox(0,0)[r]{$-0.4$}}
\put(1419.0,349.0){\rule[-0.200pt]{4.818pt}{0.400pt}}
\put(171.0,422.0){\rule[-0.200pt]{4.818pt}{0.400pt}}
\put(151,422){\makebox(0,0)[r]{$-0.2$}}
\put(1419.0,422.0){\rule[-0.200pt]{4.818pt}{0.400pt}}
\put(171.0,495.0){\rule[-0.200pt]{4.818pt}{0.400pt}}
\put(151,495){\makebox(0,0)[r]{$0$}}
\put(1419.0,495.0){\rule[-0.200pt]{4.818pt}{0.400pt}}
\put(171.0,568.0){\rule[-0.200pt]{4.818pt}{0.400pt}}
\put(151,568){\makebox(0,0)[r]{$0.2$}}
\put(1419.0,568.0){\rule[-0.200pt]{4.818pt}{0.400pt}}
\put(171.0,641.0){\rule[-0.200pt]{4.818pt}{0.400pt}}
\put(151,641){\makebox(0,0)[r]{$0.4$}}
\put(1419.0,641.0){\rule[-0.200pt]{4.818pt}{0.400pt}}
\put(171.0,713.0){\rule[-0.200pt]{4.818pt}{0.400pt}}
\put(151,713){\makebox(0,0)[r]{$0.6$}}
\put(1419.0,713.0){\rule[-0.200pt]{4.818pt}{0.400pt}}
\put(171.0,786.0){\rule[-0.200pt]{4.818pt}{0.400pt}}
\put(151,786){\makebox(0,0)[r]{$0.8$}}
\put(1419.0,786.0){\rule[-0.200pt]{4.818pt}{0.400pt}}
\put(171.0,859.0){\rule[-0.200pt]{4.818pt}{0.400pt}}
\put(151,859){\makebox(0,0)[r]{$1$}}
\put(1419.0,859.0){\rule[-0.200pt]{4.818pt}{0.400pt}}
\put(171.0,131.0){\rule[-0.200pt]{0.400pt}{4.818pt}}
\put(171,90){\makebox(0,0){$-0.6$}}
\put(171.0,839.0){\rule[-0.200pt]{0.400pt}{4.818pt}}
\put(382.0,131.0){\rule[-0.200pt]{0.400pt}{4.818pt}}
\put(382,90){\makebox(0,0){$-0.4$}}
\put(382.0,839.0){\rule[-0.200pt]{0.400pt}{4.818pt}}
\put(594.0,131.0){\rule[-0.200pt]{0.400pt}{4.818pt}}
\put(594,90){\makebox(0,0){$-0.2$}}
\put(594.0,839.0){\rule[-0.200pt]{0.400pt}{4.818pt}}
\put(805.0,131.0){\rule[-0.200pt]{0.400pt}{4.818pt}}
\put(805,90){\makebox(0,0){$0$}}
\put(805.0,839.0){\rule[-0.200pt]{0.400pt}{4.818pt}}
\put(1016.0,131.0){\rule[-0.200pt]{0.400pt}{4.818pt}}
\put(1016,90){\makebox(0,0){$0.2$}}
\put(1016.0,839.0){\rule[-0.200pt]{0.400pt}{4.818pt}}
\put(1228.0,131.0){\rule[-0.200pt]{0.400pt}{4.818pt}}
\put(1228,90){\makebox(0,0){$0.4$}}
\put(1228.0,839.0){\rule[-0.200pt]{0.400pt}{4.818pt}}
\put(1439.0,131.0){\rule[-0.200pt]{0.400pt}{4.818pt}}
\put(1439,90){\makebox(0,0){$0.6$}}
\put(1439.0,839.0){\rule[-0.200pt]{0.400pt}{4.818pt}}
\put(171.0,131.0){\rule[-0.200pt]{0.400pt}{175.375pt}}
\put(171.0,131.0){\rule[-0.200pt]{305.461pt}{0.400pt}}
\put(1439.0,131.0){\rule[-0.200pt]{0.400pt}{175.375pt}}
\put(171.0,859.0){\rule[-0.200pt]{305.461pt}{0.400pt}}
\put(30,495){\makebox(0,0){$f(x)$}}
\put(805,29){\makebox(0,0){$x$}}
\put(277,495){\usebox{\plotpoint}}
\put(460,493.67){\rule{0.482pt}{0.400pt}}
\multiput(460.00,494.17)(1.000,-1.000){2}{\rule{0.241pt}{0.400pt}}
\put(277.0,495.0){\rule[-0.200pt]{44.085pt}{0.400pt}}
\put(477,493.67){\rule{0.482pt}{0.400pt}}
\multiput(477.00,493.17)(1.000,1.000){2}{\rule{0.241pt}{0.400pt}}
\put(462.0,494.0){\rule[-0.200pt]{3.613pt}{0.400pt}}
\put(485,494.67){\rule{0.482pt}{0.400pt}}
\multiput(485.00,494.17)(1.000,1.000){2}{\rule{0.241pt}{0.400pt}}
\put(479.0,495.0){\rule[-0.200pt]{1.445pt}{0.400pt}}
\put(491,494.67){\rule{0.482pt}{0.400pt}}
\multiput(491.00,495.17)(1.000,-1.000){2}{\rule{0.241pt}{0.400pt}}
\put(487.0,496.0){\rule[-0.200pt]{0.964pt}{0.400pt}}
\put(495,493.67){\rule{0.482pt}{0.400pt}}
\multiput(495.00,494.17)(1.000,-1.000){2}{\rule{0.241pt}{0.400pt}}
\put(497,492.67){\rule{0.723pt}{0.400pt}}
\multiput(497.00,493.17)(1.500,-1.000){2}{\rule{0.361pt}{0.400pt}}
\put(493.0,495.0){\rule[-0.200pt]{0.482pt}{0.400pt}}
\put(502,492.67){\rule{0.482pt}{0.400pt}}
\multiput(502.00,492.17)(1.000,1.000){2}{\rule{0.241pt}{0.400pt}}
\put(500.0,493.0){\rule[-0.200pt]{0.482pt}{0.400pt}}
\put(506,493.67){\rule{0.482pt}{0.400pt}}
\multiput(506.00,493.17)(1.000,1.000){2}{\rule{0.241pt}{0.400pt}}
\put(508,494.67){\rule{0.482pt}{0.400pt}}
\multiput(508.00,494.17)(1.000,1.000){2}{\rule{0.241pt}{0.400pt}}
\put(510,495.67){\rule{0.482pt}{0.400pt}}
\multiput(510.00,495.17)(1.000,1.000){2}{\rule{0.241pt}{0.400pt}}
\put(512,496.67){\rule{0.482pt}{0.400pt}}
\multiput(512.00,496.17)(1.000,1.000){2}{\rule{0.241pt}{0.400pt}}
\put(504.0,494.0){\rule[-0.200pt]{0.482pt}{0.400pt}}
\put(516,496.67){\rule{0.482pt}{0.400pt}}
\multiput(516.00,497.17)(1.000,-1.000){2}{\rule{0.241pt}{0.400pt}}
\put(518,495.17){\rule{0.482pt}{0.400pt}}
\multiput(518.00,496.17)(1.000,-2.000){2}{\rule{0.241pt}{0.400pt}}
\put(520,493.67){\rule{0.482pt}{0.400pt}}
\multiput(520.00,494.17)(1.000,-1.000){2}{\rule{0.241pt}{0.400pt}}
\put(522,492.17){\rule{0.482pt}{0.400pt}}
\multiput(522.00,493.17)(1.000,-2.000){2}{\rule{0.241pt}{0.400pt}}
\put(524,490.67){\rule{0.482pt}{0.400pt}}
\multiput(524.00,491.17)(1.000,-1.000){2}{\rule{0.241pt}{0.400pt}}
\put(514.0,498.0){\rule[-0.200pt]{0.482pt}{0.400pt}}
\put(528,490.67){\rule{0.723pt}{0.400pt}}
\multiput(528.00,490.17)(1.500,1.000){2}{\rule{0.361pt}{0.400pt}}
\put(531,491.67){\rule{0.482pt}{0.400pt}}
\multiput(531.00,491.17)(1.000,1.000){2}{\rule{0.241pt}{0.400pt}}
\put(533,493.17){\rule{0.482pt}{0.400pt}}
\multiput(533.00,492.17)(1.000,2.000){2}{\rule{0.241pt}{0.400pt}}
\put(535.17,495){\rule{0.400pt}{0.700pt}}
\multiput(534.17,495.00)(2.000,1.547){2}{\rule{0.400pt}{0.350pt}}
\put(537.17,498){\rule{0.400pt}{0.700pt}}
\multiput(536.17,498.00)(2.000,1.547){2}{\rule{0.400pt}{0.350pt}}
\put(539,500.67){\rule{0.482pt}{0.400pt}}
\multiput(539.00,500.17)(1.000,1.000){2}{\rule{0.241pt}{0.400pt}}
\put(541,500.67){\rule{0.482pt}{0.400pt}}
\multiput(541.00,501.17)(1.000,-1.000){2}{\rule{0.241pt}{0.400pt}}
\put(543,499.17){\rule{0.482pt}{0.400pt}}
\multiput(543.00,500.17)(1.000,-2.000){2}{\rule{0.241pt}{0.400pt}}
\put(545.17,496){\rule{0.400pt}{0.700pt}}
\multiput(544.17,497.55)(2.000,-1.547){2}{\rule{0.400pt}{0.350pt}}
\put(547.17,492){\rule{0.400pt}{0.900pt}}
\multiput(546.17,494.13)(2.000,-2.132){2}{\rule{0.400pt}{0.450pt}}
\put(549.17,488){\rule{0.400pt}{0.900pt}}
\multiput(548.17,490.13)(2.000,-2.132){2}{\rule{0.400pt}{0.450pt}}
\put(551,486.17){\rule{0.482pt}{0.400pt}}
\multiput(551.00,487.17)(1.000,-2.000){2}{\rule{0.241pt}{0.400pt}}
\put(553,484.67){\rule{0.482pt}{0.400pt}}
\multiput(553.00,485.17)(1.000,-1.000){2}{\rule{0.241pt}{0.400pt}}
\put(555,485.17){\rule{0.482pt}{0.400pt}}
\multiput(555.00,484.17)(1.000,2.000){2}{\rule{0.241pt}{0.400pt}}
\put(557.17,487){\rule{0.400pt}{1.100pt}}
\multiput(556.17,487.00)(2.000,2.717){2}{\rule{0.400pt}{0.550pt}}
\put(559.17,492){\rule{0.400pt}{1.300pt}}
\multiput(558.17,492.00)(2.000,3.302){2}{\rule{0.400pt}{0.650pt}}
\multiput(561.61,498.00)(0.447,1.132){3}{\rule{0.108pt}{0.900pt}}
\multiput(560.17,498.00)(3.000,4.132){2}{\rule{0.400pt}{0.450pt}}
\put(564.17,504){\rule{0.400pt}{0.900pt}}
\multiput(563.17,504.00)(2.000,2.132){2}{\rule{0.400pt}{0.450pt}}
\put(566,507.67){\rule{0.482pt}{0.400pt}}
\multiput(566.00,507.17)(1.000,1.000){2}{\rule{0.241pt}{0.400pt}}
\put(568,507.67){\rule{0.482pt}{0.400pt}}
\multiput(568.00,508.17)(1.000,-1.000){2}{\rule{0.241pt}{0.400pt}}
\put(570.17,503){\rule{0.400pt}{1.100pt}}
\multiput(569.17,505.72)(2.000,-2.717){2}{\rule{0.400pt}{0.550pt}}
\put(572.17,495){\rule{0.400pt}{1.700pt}}
\multiput(571.17,499.47)(2.000,-4.472){2}{\rule{0.400pt}{0.850pt}}
\put(574.17,487){\rule{0.400pt}{1.700pt}}
\multiput(573.17,491.47)(2.000,-4.472){2}{\rule{0.400pt}{0.850pt}}
\put(576.17,479){\rule{0.400pt}{1.700pt}}
\multiput(575.17,483.47)(2.000,-4.472){2}{\rule{0.400pt}{0.850pt}}
\put(578.17,475){\rule{0.400pt}{0.900pt}}
\multiput(577.17,477.13)(2.000,-2.132){2}{\rule{0.400pt}{0.450pt}}
\put(580,474.67){\rule{0.482pt}{0.400pt}}
\multiput(580.00,474.17)(1.000,1.000){2}{\rule{0.241pt}{0.400pt}}
\put(582.17,476){\rule{0.400pt}{1.100pt}}
\multiput(581.17,476.00)(2.000,2.717){2}{\rule{0.400pt}{0.550pt}}
\put(584.17,481){\rule{0.400pt}{1.900pt}}
\multiput(583.17,481.00)(2.000,5.056){2}{\rule{0.400pt}{0.950pt}}
\put(586.17,490){\rule{0.400pt}{2.500pt}}
\multiput(585.17,490.00)(2.000,6.811){2}{\rule{0.400pt}{1.250pt}}
\put(588.17,502){\rule{0.400pt}{2.300pt}}
\multiput(587.17,502.00)(2.000,6.226){2}{\rule{0.400pt}{1.150pt}}
\put(590.17,513){\rule{0.400pt}{1.700pt}}
\multiput(589.17,513.00)(2.000,4.472){2}{\rule{0.400pt}{0.850pt}}
\put(592,521.17){\rule{0.482pt}{0.400pt}}
\multiput(592.00,520.17)(1.000,2.000){2}{\rule{0.241pt}{0.400pt}}
\multiput(594.61,519.82)(0.447,-0.909){3}{\rule{0.108pt}{0.767pt}}
\multiput(593.17,521.41)(3.000,-3.409){2}{\rule{0.400pt}{0.383pt}}
\put(597.17,507){\rule{0.400pt}{2.300pt}}
\multiput(596.17,513.23)(2.000,-6.226){2}{\rule{0.400pt}{1.150pt}}
\put(599.17,492){\rule{0.400pt}{3.100pt}}
\multiput(598.17,500.57)(2.000,-8.566){2}{\rule{0.400pt}{1.550pt}}
\put(601.17,476){\rule{0.400pt}{3.300pt}}
\multiput(600.17,485.15)(2.000,-9.151){2}{\rule{0.400pt}{1.650pt}}
\put(603.17,463){\rule{0.400pt}{2.700pt}}
\multiput(602.17,470.40)(2.000,-7.396){2}{\rule{0.400pt}{1.350pt}}
\put(605.17,457){\rule{0.400pt}{1.300pt}}
\multiput(604.17,460.30)(2.000,-3.302){2}{\rule{0.400pt}{0.650pt}}
\put(607.17,457){\rule{0.400pt}{0.700pt}}
\multiput(606.17,457.00)(2.000,1.547){2}{\rule{0.400pt}{0.350pt}}
\put(609.17,460){\rule{0.400pt}{2.500pt}}
\multiput(608.17,460.00)(2.000,6.811){2}{\rule{0.400pt}{1.250pt}}
\put(611.17,472){\rule{0.400pt}{3.900pt}}
\multiput(610.17,472.00)(2.000,10.905){2}{\rule{0.400pt}{1.950pt}}
\put(613.17,491){\rule{0.400pt}{4.500pt}}
\multiput(612.17,491.00)(2.000,12.660){2}{\rule{0.400pt}{2.250pt}}
\put(615.17,513){\rule{0.400pt}{3.900pt}}
\multiput(614.17,513.00)(2.000,10.905){2}{\rule{0.400pt}{1.950pt}}
\put(617.17,532){\rule{0.400pt}{2.500pt}}
\multiput(616.17,532.00)(2.000,6.811){2}{\rule{0.400pt}{1.250pt}}
\put(619,543.67){\rule{0.482pt}{0.400pt}}
\multiput(619.00,543.17)(1.000,1.000){2}{\rule{0.241pt}{0.400pt}}
\put(621.17,534){\rule{0.400pt}{2.300pt}}
\multiput(620.17,540.23)(2.000,-6.226){2}{\rule{0.400pt}{1.150pt}}
\put(623.17,511){\rule{0.400pt}{4.700pt}}
\multiput(622.17,524.24)(2.000,-13.245){2}{\rule{0.400pt}{2.350pt}}
\multiput(625.61,495.09)(0.447,-6.044){3}{\rule{0.108pt}{3.833pt}}
\multiput(624.17,503.04)(3.000,-20.044){2}{\rule{0.400pt}{1.917pt}}
\put(628.17,455){\rule{0.400pt}{5.700pt}}
\multiput(627.17,471.17)(2.000,-16.169){2}{\rule{0.400pt}{2.850pt}}
\put(630.17,435){\rule{0.400pt}{4.100pt}}
\multiput(629.17,446.49)(2.000,-11.490){2}{\rule{0.400pt}{2.050pt}}
\put(632.17,428){\rule{0.400pt}{1.500pt}}
\multiput(631.17,431.89)(2.000,-3.887){2}{\rule{0.400pt}{0.750pt}}
\put(634.17,428){\rule{0.400pt}{1.700pt}}
\multiput(633.17,428.00)(2.000,4.472){2}{\rule{0.400pt}{0.850pt}}
\put(636.17,436){\rule{0.400pt}{5.100pt}}
\multiput(635.17,436.00)(2.000,14.415){2}{\rule{0.400pt}{2.550pt}}
\put(638.17,461){\rule{0.400pt}{6.900pt}}
\multiput(637.17,461.00)(2.000,19.679){2}{\rule{0.400pt}{3.450pt}}
\put(640.17,495){\rule{0.400pt}{7.500pt}}
\multiput(639.17,495.00)(2.000,21.433){2}{\rule{0.400pt}{3.750pt}}
\put(642.17,532){\rule{0.400pt}{6.500pt}}
\multiput(641.17,532.00)(2.000,18.509){2}{\rule{0.400pt}{3.250pt}}
\put(644.17,564){\rule{0.400pt}{3.300pt}}
\multiput(643.17,564.00)(2.000,9.151){2}{\rule{0.400pt}{1.650pt}}
\put(646.17,577){\rule{0.400pt}{0.700pt}}
\multiput(645.17,578.55)(2.000,-1.547){2}{\rule{0.400pt}{0.350pt}}
\put(648.17,553){\rule{0.400pt}{4.900pt}}
\multiput(647.17,566.83)(2.000,-13.830){2}{\rule{0.400pt}{2.450pt}}
\put(650.17,514){\rule{0.400pt}{7.900pt}}
\multiput(649.17,536.60)(2.000,-22.603){2}{\rule{0.400pt}{3.950pt}}
\put(652.17,466){\rule{0.400pt}{9.700pt}}
\multiput(651.17,493.87)(2.000,-27.867){2}{\rule{0.400pt}{4.850pt}}
\put(654.17,422){\rule{0.400pt}{8.900pt}}
\multiput(653.17,447.53)(2.000,-25.528){2}{\rule{0.400pt}{4.450pt}}
\put(656.17,394){\rule{0.400pt}{5.700pt}}
\multiput(655.17,410.17)(2.000,-16.169){2}{\rule{0.400pt}{2.850pt}}
\multiput(658.61,389.71)(0.447,-1.355){3}{\rule{0.108pt}{1.033pt}}
\multiput(657.17,391.86)(3.000,-4.855){2}{\rule{0.400pt}{0.517pt}}
\put(661.17,387){\rule{0.400pt}{4.100pt}}
\multiput(660.17,387.00)(2.000,11.490){2}{\rule{0.400pt}{2.050pt}}
\put(663.17,407){\rule{0.400pt}{8.700pt}}
\multiput(662.17,407.00)(2.000,24.943){2}{\rule{0.400pt}{4.350pt}}
\put(665.17,450){\rule{0.400pt}{11.500pt}}
\multiput(664.17,450.00)(2.000,33.131){2}{\rule{0.400pt}{5.750pt}}
\put(667.17,507){\rule{0.400pt}{11.700pt}}
\multiput(666.17,507.00)(2.000,33.716){2}{\rule{0.400pt}{5.850pt}}
\put(669.17,565){\rule{0.400pt}{8.900pt}}
\multiput(668.17,565.00)(2.000,25.528){2}{\rule{0.400pt}{4.450pt}}
\put(671.17,609){\rule{0.400pt}{4.100pt}}
\multiput(670.17,609.00)(2.000,11.490){2}{\rule{0.400pt}{2.050pt}}
\put(673.17,616){\rule{0.400pt}{2.700pt}}
\multiput(672.17,623.40)(2.000,-7.396){2}{\rule{0.400pt}{1.350pt}}
\put(675.17,574){\rule{0.400pt}{8.500pt}}
\multiput(674.17,598.36)(2.000,-24.358){2}{\rule{0.400pt}{4.250pt}}
\put(677.17,509){\rule{0.400pt}{13.100pt}}
\multiput(676.17,546.81)(2.000,-37.810){2}{\rule{0.400pt}{6.550pt}}
\put(679.17,437){\rule{0.400pt}{14.500pt}}
\multiput(678.17,478.90)(2.000,-41.905){2}{\rule{0.400pt}{7.250pt}}
\put(681.17,375){\rule{0.400pt}{12.500pt}}
\multiput(680.17,411.06)(2.000,-36.056){2}{\rule{0.400pt}{6.250pt}}
\put(683.17,338){\rule{0.400pt}{7.500pt}}
\multiput(682.17,359.43)(2.000,-21.433){2}{\rule{0.400pt}{3.750pt}}
\put(526.0,491.0){\rule[-0.200pt]{0.482pt}{0.400pt}}
\put(687.17,338){\rule{0.400pt}{7.500pt}}
\multiput(686.17,338.00)(2.000,21.433){2}{\rule{0.400pt}{3.750pt}}
\put(689.17,375){\rule{0.400pt}{14.100pt}}
\multiput(688.17,375.00)(2.000,40.735){2}{\rule{0.400pt}{7.050pt}}
\multiput(691.61,445.00)(0.447,18.770){3}{\rule{0.108pt}{11.433pt}}
\multiput(690.17,445.00)(3.000,61.270){2}{\rule{0.400pt}{5.717pt}}
\put(694.17,530){\rule{0.400pt}{16.300pt}}
\multiput(693.17,530.00)(2.000,47.169){2}{\rule{0.400pt}{8.150pt}}
\put(696.17,611){\rule{0.400pt}{11.700pt}}
\multiput(695.17,611.00)(2.000,33.716){2}{\rule{0.400pt}{5.850pt}}
\put(698.17,669){\rule{0.400pt}{3.700pt}}
\multiput(697.17,669.00)(2.000,10.320){2}{\rule{0.400pt}{1.850pt}}
\put(700.17,659){\rule{0.400pt}{5.700pt}}
\multiput(699.17,675.17)(2.000,-16.169){2}{\rule{0.400pt}{2.850pt}}
\put(702.17,590){\rule{0.400pt}{13.900pt}}
\multiput(701.17,630.15)(2.000,-40.150){2}{\rule{0.400pt}{6.950pt}}
\put(704.17,495){\rule{0.400pt}{19.100pt}}
\multiput(703.17,550.36)(2.000,-55.357){2}{\rule{0.400pt}{9.550pt}}
\put(706.17,395){\rule{0.400pt}{20.100pt}}
\multiput(705.17,453.28)(2.000,-58.281){2}{\rule{0.400pt}{10.050pt}}
\put(708.17,314){\rule{0.400pt}{16.300pt}}
\multiput(707.17,361.17)(2.000,-47.169){2}{\rule{0.400pt}{8.150pt}}
\put(710.17,274){\rule{0.400pt}{8.100pt}}
\multiput(709.17,297.19)(2.000,-23.188){2}{\rule{0.400pt}{4.050pt}}
\put(712.17,274){\rule{0.400pt}{2.300pt}}
\multiput(711.17,274.00)(2.000,6.226){2}{\rule{0.400pt}{1.150pt}}
\put(714.17,285){\rule{0.400pt}{12.900pt}}
\multiput(713.17,285.00)(2.000,37.225){2}{\rule{0.400pt}{6.450pt}}
\put(716.17,349){\rule{0.400pt}{20.100pt}}
\multiput(715.17,349.00)(2.000,58.281){2}{\rule{0.400pt}{10.050pt}}
\put(718.17,449){\rule{0.400pt}{23.300pt}}
\multiput(717.17,449.00)(2.000,67.640){2}{\rule{0.400pt}{11.650pt}}
\put(720.17,565){\rule{0.400pt}{20.900pt}}
\multiput(719.17,565.00)(2.000,60.621){2}{\rule{0.400pt}{10.450pt}}
\multiput(722.61,669.00)(0.447,14.528){3}{\rule{0.108pt}{8.900pt}}
\multiput(721.17,669.00)(3.000,47.528){2}{\rule{0.400pt}{4.450pt}}
\put(725.17,735){\rule{0.400pt}{2.300pt}}
\multiput(724.17,735.00)(2.000,6.226){2}{\rule{0.400pt}{1.150pt}}
\put(727.17,696){\rule{0.400pt}{10.100pt}}
\multiput(726.17,725.04)(2.000,-29.037){2}{\rule{0.400pt}{5.050pt}}
\put(729.17,597){\rule{0.400pt}{19.900pt}}
\multiput(728.17,654.70)(2.000,-57.697){2}{\rule{0.400pt}{9.950pt}}
\put(731.17,468){\rule{0.400pt}{25.900pt}}
\multiput(730.17,543.24)(2.000,-75.243){2}{\rule{0.400pt}{12.950pt}}
\put(733.17,342){\rule{0.400pt}{25.300pt}}
\multiput(732.17,415.49)(2.000,-73.489){2}{\rule{0.400pt}{12.650pt}}
\put(735.17,249){\rule{0.400pt}{18.700pt}}
\multiput(734.17,303.19)(2.000,-54.187){2}{\rule{0.400pt}{9.350pt}}
\put(737.17,212){\rule{0.400pt}{7.500pt}}
\multiput(736.17,233.43)(2.000,-21.433){2}{\rule{0.400pt}{3.750pt}}
\put(739.17,212){\rule{0.400pt}{5.900pt}}
\multiput(738.17,212.00)(2.000,16.754){2}{\rule{0.400pt}{2.950pt}}
\put(741.17,241){\rule{0.400pt}{18.500pt}}
\multiput(740.17,241.00)(2.000,53.602){2}{\rule{0.400pt}{9.250pt}}
\put(743.17,333){\rule{0.400pt}{26.700pt}}
\multiput(742.17,333.00)(2.000,77.583){2}{\rule{0.400pt}{13.350pt}}
\put(745.17,466){\rule{0.400pt}{28.900pt}}
\multiput(744.17,466.00)(2.000,84.017){2}{\rule{0.400pt}{14.450pt}}
\put(747.17,610){\rule{0.400pt}{24.300pt}}
\multiput(746.17,610.00)(2.000,70.564){2}{\rule{0.400pt}{12.150pt}}
\put(749.17,731){\rule{0.400pt}{13.500pt}}
\multiput(748.17,731.00)(2.000,38.980){2}{\rule{0.400pt}{6.750pt}}
\put(751.17,794){\rule{0.400pt}{0.900pt}}
\multiput(750.17,796.13)(2.000,-2.132){2}{\rule{0.400pt}{0.450pt}}
\put(753.17,719){\rule{0.400pt}{15.100pt}}
\multiput(752.17,762.66)(2.000,-43.659){2}{\rule{0.400pt}{7.550pt}}
\multiput(755.61,645.53)(0.447,-29.263){3}{\rule{0.108pt}{17.700pt}}
\multiput(754.17,682.26)(3.000,-95.263){2}{\rule{0.400pt}{8.850pt}}
\put(758.17,432){\rule{0.400pt}{31.100pt}}
\multiput(757.17,522.45)(2.000,-90.450){2}{\rule{0.400pt}{15.550pt}}
\put(760.17,288){\rule{0.400pt}{28.900pt}}
\multiput(759.17,372.02)(2.000,-84.017){2}{\rule{0.400pt}{14.450pt}}
\put(762.17,190){\rule{0.400pt}{19.700pt}}
\multiput(761.17,247.11)(2.000,-57.112){2}{\rule{0.400pt}{9.850pt}}
\put(764.17,163){\rule{0.400pt}{5.500pt}}
\multiput(763.17,178.58)(2.000,-15.584){2}{\rule{0.400pt}{2.750pt}}
\put(766.17,163){\rule{0.400pt}{10.500pt}}
\multiput(765.17,163.00)(2.000,30.207){2}{\rule{0.400pt}{5.250pt}}
\put(768.17,215){\rule{0.400pt}{24.300pt}}
\multiput(767.17,215.00)(2.000,70.564){2}{\rule{0.400pt}{12.150pt}}
\put(770.17,336){\rule{0.400pt}{31.900pt}}
\multiput(769.17,336.00)(2.000,92.790){2}{\rule{0.400pt}{15.950pt}}
\put(772.17,495){\rule{0.400pt}{32.500pt}}
\multiput(771.17,495.00)(2.000,94.545){2}{\rule{0.400pt}{16.250pt}}
\put(774.17,657){\rule{0.400pt}{25.500pt}}
\multiput(773.17,657.00)(2.000,74.073){2}{\rule{0.400pt}{12.750pt}}
\put(776.17,784){\rule{0.400pt}{11.900pt}}
\multiput(775.17,784.00)(2.000,34.301){2}{\rule{0.400pt}{5.950pt}}
\put(778.17,820){\rule{0.400pt}{4.700pt}}
\multiput(777.17,833.24)(2.000,-13.245){2}{\rule{0.400pt}{2.350pt}}
\put(780.17,719){\rule{0.400pt}{20.300pt}}
\multiput(779.17,777.87)(2.000,-58.866){2}{\rule{0.400pt}{10.150pt}}
\put(782.17,564){\rule{0.400pt}{31.100pt}}
\multiput(781.17,654.45)(2.000,-90.450){2}{\rule{0.400pt}{15.550pt}}
\put(784.17,392){\rule{0.400pt}{34.500pt}}
\multiput(783.17,492.39)(2.000,-100.394){2}{\rule{0.400pt}{17.250pt}}
\put(786.17,242){\rule{0.400pt}{30.100pt}}
\multiput(785.17,329.53)(2.000,-87.526){2}{\rule{0.400pt}{15.050pt}}
\multiput(788.61,191.22)(0.447,-20.109){3}{\rule{0.108pt}{12.233pt}}
\multiput(787.17,216.61)(3.000,-65.609){2}{\rule{0.400pt}{6.117pt}}
\put(791.17,141){\rule{0.400pt}{2.100pt}}
\multiput(790.17,146.64)(2.000,-5.641){2}{\rule{0.400pt}{1.050pt}}
\put(793.17,141){\rule{0.400pt}{14.900pt}}
\multiput(792.17,141.00)(2.000,43.074){2}{\rule{0.400pt}{7.450pt}}
\put(795.17,215){\rule{0.400pt}{28.300pt}}
\multiput(794.17,215.00)(2.000,82.262){2}{\rule{0.400pt}{14.150pt}}
\put(797.17,356){\rule{0.400pt}{34.900pt}}
\multiput(796.17,356.00)(2.000,101.563){2}{\rule{0.400pt}{17.450pt}}
\put(799.17,530){\rule{0.400pt}{33.500pt}}
\multiput(798.17,530.00)(2.000,97.469){2}{\rule{0.400pt}{16.750pt}}
\put(801.17,697){\rule{0.400pt}{23.900pt}}
\multiput(800.17,697.00)(2.000,69.394){2}{\rule{0.400pt}{11.950pt}}
\put(803.17,816){\rule{0.400pt}{8.700pt}}
\multiput(802.17,816.00)(2.000,24.943){2}{\rule{0.400pt}{4.350pt}}
\put(805.17,816){\rule{0.400pt}{8.700pt}}
\multiput(804.17,840.94)(2.000,-24.943){2}{\rule{0.400pt}{4.350pt}}
\put(807.17,697){\rule{0.400pt}{23.900pt}}
\multiput(806.17,766.39)(2.000,-69.394){2}{\rule{0.400pt}{11.950pt}}
\put(809.17,530){\rule{0.400pt}{33.500pt}}
\multiput(808.17,627.47)(2.000,-97.469){2}{\rule{0.400pt}{16.750pt}}
\put(811.17,356){\rule{0.400pt}{34.900pt}}
\multiput(810.17,457.56)(2.000,-101.563){2}{\rule{0.400pt}{17.450pt}}
\put(813.17,215){\rule{0.400pt}{28.300pt}}
\multiput(812.17,297.26)(2.000,-82.262){2}{\rule{0.400pt}{14.150pt}}
\put(815.17,141){\rule{0.400pt}{14.900pt}}
\multiput(814.17,184.07)(2.000,-43.074){2}{\rule{0.400pt}{7.450pt}}
\put(817.17,141){\rule{0.400pt}{2.100pt}}
\multiput(816.17,141.00)(2.000,5.641){2}{\rule{0.400pt}{1.050pt}}
\multiput(819.61,151.00)(0.447,20.109){3}{\rule{0.108pt}{12.233pt}}
\multiput(818.17,151.00)(3.000,65.609){2}{\rule{0.400pt}{6.117pt}}
\put(822.17,242){\rule{0.400pt}{30.100pt}}
\multiput(821.17,242.00)(2.000,87.526){2}{\rule{0.400pt}{15.050pt}}
\put(824.17,392){\rule{0.400pt}{34.500pt}}
\multiput(823.17,392.00)(2.000,100.394){2}{\rule{0.400pt}{17.250pt}}
\put(826.17,564){\rule{0.400pt}{31.100pt}}
\multiput(825.17,564.00)(2.000,90.450){2}{\rule{0.400pt}{15.550pt}}
\put(828.17,719){\rule{0.400pt}{20.300pt}}
\multiput(827.17,719.00)(2.000,58.866){2}{\rule{0.400pt}{10.150pt}}
\put(830.17,820){\rule{0.400pt}{4.700pt}}
\multiput(829.17,820.00)(2.000,13.245){2}{\rule{0.400pt}{2.350pt}}
\put(832.17,784){\rule{0.400pt}{11.900pt}}
\multiput(831.17,818.30)(2.000,-34.301){2}{\rule{0.400pt}{5.950pt}}
\put(834.17,657){\rule{0.400pt}{25.500pt}}
\multiput(833.17,731.07)(2.000,-74.073){2}{\rule{0.400pt}{12.750pt}}
\put(836.17,495){\rule{0.400pt}{32.500pt}}
\multiput(835.17,589.54)(2.000,-94.545){2}{\rule{0.400pt}{16.250pt}}
\put(838.17,336){\rule{0.400pt}{31.900pt}}
\multiput(837.17,428.79)(2.000,-92.790){2}{\rule{0.400pt}{15.950pt}}
\put(840.17,215){\rule{0.400pt}{24.300pt}}
\multiput(839.17,285.56)(2.000,-70.564){2}{\rule{0.400pt}{12.150pt}}
\put(842.17,163){\rule{0.400pt}{10.500pt}}
\multiput(841.17,193.21)(2.000,-30.207){2}{\rule{0.400pt}{5.250pt}}
\put(844.17,163){\rule{0.400pt}{5.500pt}}
\multiput(843.17,163.00)(2.000,15.584){2}{\rule{0.400pt}{2.750pt}}
\put(846.17,190){\rule{0.400pt}{19.700pt}}
\multiput(845.17,190.00)(2.000,57.112){2}{\rule{0.400pt}{9.850pt}}
\put(848.17,288){\rule{0.400pt}{28.900pt}}
\multiput(847.17,288.00)(2.000,84.017){2}{\rule{0.400pt}{14.450pt}}
\put(850.17,432){\rule{0.400pt}{31.100pt}}
\multiput(849.17,432.00)(2.000,90.450){2}{\rule{0.400pt}{15.550pt}}
\multiput(852.61,587.00)(0.447,29.263){3}{\rule{0.108pt}{17.700pt}}
\multiput(851.17,587.00)(3.000,95.263){2}{\rule{0.400pt}{8.850pt}}
\put(855.17,719){\rule{0.400pt}{15.100pt}}
\multiput(854.17,719.00)(2.000,43.659){2}{\rule{0.400pt}{7.550pt}}
\put(857.17,794){\rule{0.400pt}{0.900pt}}
\multiput(856.17,794.00)(2.000,2.132){2}{\rule{0.400pt}{0.450pt}}
\put(859.17,731){\rule{0.400pt}{13.500pt}}
\multiput(858.17,769.98)(2.000,-38.980){2}{\rule{0.400pt}{6.750pt}}
\put(861.17,610){\rule{0.400pt}{24.300pt}}
\multiput(860.17,680.56)(2.000,-70.564){2}{\rule{0.400pt}{12.150pt}}
\put(863.17,466){\rule{0.400pt}{28.900pt}}
\multiput(862.17,550.02)(2.000,-84.017){2}{\rule{0.400pt}{14.450pt}}
\put(865.17,333){\rule{0.400pt}{26.700pt}}
\multiput(864.17,410.58)(2.000,-77.583){2}{\rule{0.400pt}{13.350pt}}
\put(867.17,241){\rule{0.400pt}{18.500pt}}
\multiput(866.17,294.60)(2.000,-53.602){2}{\rule{0.400pt}{9.250pt}}
\put(869.17,212){\rule{0.400pt}{5.900pt}}
\multiput(868.17,228.75)(2.000,-16.754){2}{\rule{0.400pt}{2.950pt}}
\put(871.17,212){\rule{0.400pt}{7.500pt}}
\multiput(870.17,212.00)(2.000,21.433){2}{\rule{0.400pt}{3.750pt}}
\put(873.17,249){\rule{0.400pt}{18.700pt}}
\multiput(872.17,249.00)(2.000,54.187){2}{\rule{0.400pt}{9.350pt}}
\put(875.17,342){\rule{0.400pt}{25.300pt}}
\multiput(874.17,342.00)(2.000,73.489){2}{\rule{0.400pt}{12.650pt}}
\put(877.17,468){\rule{0.400pt}{25.900pt}}
\multiput(876.17,468.00)(2.000,75.243){2}{\rule{0.400pt}{12.950pt}}
\put(879.17,597){\rule{0.400pt}{19.900pt}}
\multiput(878.17,597.00)(2.000,57.697){2}{\rule{0.400pt}{9.950pt}}
\put(881.17,696){\rule{0.400pt}{10.100pt}}
\multiput(880.17,696.00)(2.000,29.037){2}{\rule{0.400pt}{5.050pt}}
\put(883.17,735){\rule{0.400pt}{2.300pt}}
\multiput(882.17,741.23)(2.000,-6.226){2}{\rule{0.400pt}{1.150pt}}
\multiput(885.61,698.06)(0.447,-14.528){3}{\rule{0.108pt}{8.900pt}}
\multiput(884.17,716.53)(3.000,-47.528){2}{\rule{0.400pt}{4.450pt}}
\put(888.17,565){\rule{0.400pt}{20.900pt}}
\multiput(887.17,625.62)(2.000,-60.621){2}{\rule{0.400pt}{10.450pt}}
\put(890.17,449){\rule{0.400pt}{23.300pt}}
\multiput(889.17,516.64)(2.000,-67.640){2}{\rule{0.400pt}{11.650pt}}
\put(892.17,349){\rule{0.400pt}{20.100pt}}
\multiput(891.17,407.28)(2.000,-58.281){2}{\rule{0.400pt}{10.050pt}}
\put(894.17,285){\rule{0.400pt}{12.900pt}}
\multiput(893.17,322.23)(2.000,-37.225){2}{\rule{0.400pt}{6.450pt}}
\put(896.17,274){\rule{0.400pt}{2.300pt}}
\multiput(895.17,280.23)(2.000,-6.226){2}{\rule{0.400pt}{1.150pt}}
\put(898.17,274){\rule{0.400pt}{8.100pt}}
\multiput(897.17,274.00)(2.000,23.188){2}{\rule{0.400pt}{4.050pt}}
\put(900.17,314){\rule{0.400pt}{16.300pt}}
\multiput(899.17,314.00)(2.000,47.169){2}{\rule{0.400pt}{8.150pt}}
\put(902.17,395){\rule{0.400pt}{20.100pt}}
\multiput(901.17,395.00)(2.000,58.281){2}{\rule{0.400pt}{10.050pt}}
\put(904.17,495){\rule{0.400pt}{19.100pt}}
\multiput(903.17,495.00)(2.000,55.357){2}{\rule{0.400pt}{9.550pt}}
\put(906.17,590){\rule{0.400pt}{13.900pt}}
\multiput(905.17,590.00)(2.000,40.150){2}{\rule{0.400pt}{6.950pt}}
\put(908.17,659){\rule{0.400pt}{5.700pt}}
\multiput(907.17,659.00)(2.000,16.169){2}{\rule{0.400pt}{2.850pt}}
\put(910.17,669){\rule{0.400pt}{3.700pt}}
\multiput(909.17,679.32)(2.000,-10.320){2}{\rule{0.400pt}{1.850pt}}
\put(912.17,611){\rule{0.400pt}{11.700pt}}
\multiput(911.17,644.72)(2.000,-33.716){2}{\rule{0.400pt}{5.850pt}}
\put(914.17,530){\rule{0.400pt}{16.300pt}}
\multiput(913.17,577.17)(2.000,-47.169){2}{\rule{0.400pt}{8.150pt}}
\multiput(916.61,482.54)(0.447,-18.770){3}{\rule{0.108pt}{11.433pt}}
\multiput(915.17,506.27)(3.000,-61.270){2}{\rule{0.400pt}{5.717pt}}
\put(919.17,375){\rule{0.400pt}{14.100pt}}
\multiput(918.17,415.73)(2.000,-40.735){2}{\rule{0.400pt}{7.050pt}}
\put(921.17,338){\rule{0.400pt}{7.500pt}}
\multiput(920.17,359.43)(2.000,-21.433){2}{\rule{0.400pt}{3.750pt}}
\put(685.0,338.0){\rule[-0.200pt]{0.482pt}{0.400pt}}
\put(925.17,338){\rule{0.400pt}{7.500pt}}
\multiput(924.17,338.00)(2.000,21.433){2}{\rule{0.400pt}{3.750pt}}
\put(927.17,375){\rule{0.400pt}{12.500pt}}
\multiput(926.17,375.00)(2.000,36.056){2}{\rule{0.400pt}{6.250pt}}
\put(929.17,437){\rule{0.400pt}{14.500pt}}
\multiput(928.17,437.00)(2.000,41.905){2}{\rule{0.400pt}{7.250pt}}
\put(931.17,509){\rule{0.400pt}{13.100pt}}
\multiput(930.17,509.00)(2.000,37.810){2}{\rule{0.400pt}{6.550pt}}
\put(933.17,574){\rule{0.400pt}{8.500pt}}
\multiput(932.17,574.00)(2.000,24.358){2}{\rule{0.400pt}{4.250pt}}
\put(935.17,616){\rule{0.400pt}{2.700pt}}
\multiput(934.17,616.00)(2.000,7.396){2}{\rule{0.400pt}{1.350pt}}
\put(937.17,609){\rule{0.400pt}{4.100pt}}
\multiput(936.17,620.49)(2.000,-11.490){2}{\rule{0.400pt}{2.050pt}}
\put(939.17,565){\rule{0.400pt}{8.900pt}}
\multiput(938.17,590.53)(2.000,-25.528){2}{\rule{0.400pt}{4.450pt}}
\put(941.17,507){\rule{0.400pt}{11.700pt}}
\multiput(940.17,540.72)(2.000,-33.716){2}{\rule{0.400pt}{5.850pt}}
\put(943.17,450){\rule{0.400pt}{11.500pt}}
\multiput(942.17,483.13)(2.000,-33.131){2}{\rule{0.400pt}{5.750pt}}
\put(945.17,407){\rule{0.400pt}{8.700pt}}
\multiput(944.17,431.94)(2.000,-24.943){2}{\rule{0.400pt}{4.350pt}}
\put(947.17,387){\rule{0.400pt}{4.100pt}}
\multiput(946.17,398.49)(2.000,-11.490){2}{\rule{0.400pt}{2.050pt}}
\multiput(949.61,387.00)(0.447,1.355){3}{\rule{0.108pt}{1.033pt}}
\multiput(948.17,387.00)(3.000,4.855){2}{\rule{0.400pt}{0.517pt}}
\put(952.17,394){\rule{0.400pt}{5.700pt}}
\multiput(951.17,394.00)(2.000,16.169){2}{\rule{0.400pt}{2.850pt}}
\put(954.17,422){\rule{0.400pt}{8.900pt}}
\multiput(953.17,422.00)(2.000,25.528){2}{\rule{0.400pt}{4.450pt}}
\put(956.17,466){\rule{0.400pt}{9.700pt}}
\multiput(955.17,466.00)(2.000,27.867){2}{\rule{0.400pt}{4.850pt}}
\put(958.17,514){\rule{0.400pt}{7.900pt}}
\multiput(957.17,514.00)(2.000,22.603){2}{\rule{0.400pt}{3.950pt}}
\put(960.17,553){\rule{0.400pt}{4.900pt}}
\multiput(959.17,553.00)(2.000,13.830){2}{\rule{0.400pt}{2.450pt}}
\put(962.17,577){\rule{0.400pt}{0.700pt}}
\multiput(961.17,577.00)(2.000,1.547){2}{\rule{0.400pt}{0.350pt}}
\put(964.17,564){\rule{0.400pt}{3.300pt}}
\multiput(963.17,573.15)(2.000,-9.151){2}{\rule{0.400pt}{1.650pt}}
\put(966.17,532){\rule{0.400pt}{6.500pt}}
\multiput(965.17,550.51)(2.000,-18.509){2}{\rule{0.400pt}{3.250pt}}
\put(968.17,495){\rule{0.400pt}{7.500pt}}
\multiput(967.17,516.43)(2.000,-21.433){2}{\rule{0.400pt}{3.750pt}}
\put(970.17,461){\rule{0.400pt}{6.900pt}}
\multiput(969.17,480.68)(2.000,-19.679){2}{\rule{0.400pt}{3.450pt}}
\put(972.17,436){\rule{0.400pt}{5.100pt}}
\multiput(971.17,450.41)(2.000,-14.415){2}{\rule{0.400pt}{2.550pt}}
\put(974.17,428){\rule{0.400pt}{1.700pt}}
\multiput(973.17,432.47)(2.000,-4.472){2}{\rule{0.400pt}{0.850pt}}
\put(976.17,428){\rule{0.400pt}{1.500pt}}
\multiput(975.17,428.00)(2.000,3.887){2}{\rule{0.400pt}{0.750pt}}
\put(978.17,435){\rule{0.400pt}{4.100pt}}
\multiput(977.17,435.00)(2.000,11.490){2}{\rule{0.400pt}{2.050pt}}
\put(980.17,455){\rule{0.400pt}{5.700pt}}
\multiput(979.17,455.00)(2.000,16.169){2}{\rule{0.400pt}{2.850pt}}
\multiput(982.61,483.00)(0.447,6.267){3}{\rule{0.108pt}{3.967pt}}
\multiput(981.17,483.00)(3.000,20.767){2}{\rule{0.400pt}{1.983pt}}
\put(985.17,512){\rule{0.400pt}{4.500pt}}
\multiput(984.17,512.00)(2.000,12.660){2}{\rule{0.400pt}{2.250pt}}
\put(987.17,534){\rule{0.400pt}{2.300pt}}
\multiput(986.17,534.00)(2.000,6.226){2}{\rule{0.400pt}{1.150pt}}
\put(989,543.67){\rule{0.482pt}{0.400pt}}
\multiput(989.00,544.17)(1.000,-1.000){2}{\rule{0.241pt}{0.400pt}}
\put(991.17,532){\rule{0.400pt}{2.500pt}}
\multiput(990.17,538.81)(2.000,-6.811){2}{\rule{0.400pt}{1.250pt}}
\put(993.17,513){\rule{0.400pt}{3.900pt}}
\multiput(992.17,523.91)(2.000,-10.905){2}{\rule{0.400pt}{1.950pt}}
\put(995.17,491){\rule{0.400pt}{4.500pt}}
\multiput(994.17,503.66)(2.000,-12.660){2}{\rule{0.400pt}{2.250pt}}
\put(997.17,472){\rule{0.400pt}{3.900pt}}
\multiput(996.17,482.91)(2.000,-10.905){2}{\rule{0.400pt}{1.950pt}}
\put(999.17,460){\rule{0.400pt}{2.500pt}}
\multiput(998.17,466.81)(2.000,-6.811){2}{\rule{0.400pt}{1.250pt}}
\put(1001.17,457){\rule{0.400pt}{0.700pt}}
\multiput(1000.17,458.55)(2.000,-1.547){2}{\rule{0.400pt}{0.350pt}}
\put(1003.17,457){\rule{0.400pt}{1.300pt}}
\multiput(1002.17,457.00)(2.000,3.302){2}{\rule{0.400pt}{0.650pt}}
\put(1005.17,463){\rule{0.400pt}{2.700pt}}
\multiput(1004.17,463.00)(2.000,7.396){2}{\rule{0.400pt}{1.350pt}}
\put(1007.17,476){\rule{0.400pt}{3.300pt}}
\multiput(1006.17,476.00)(2.000,9.151){2}{\rule{0.400pt}{1.650pt}}
\put(1009.17,492){\rule{0.400pt}{3.100pt}}
\multiput(1008.17,492.00)(2.000,8.566){2}{\rule{0.400pt}{1.550pt}}
\put(1011.17,507){\rule{0.400pt}{2.300pt}}
\multiput(1010.17,507.00)(2.000,6.226){2}{\rule{0.400pt}{1.150pt}}
\multiput(1013.61,518.00)(0.447,0.909){3}{\rule{0.108pt}{0.767pt}}
\multiput(1012.17,518.00)(3.000,3.409){2}{\rule{0.400pt}{0.383pt}}
\put(1016,521.17){\rule{0.482pt}{0.400pt}}
\multiput(1016.00,522.17)(1.000,-2.000){2}{\rule{0.241pt}{0.400pt}}
\put(1018.17,513){\rule{0.400pt}{1.700pt}}
\multiput(1017.17,517.47)(2.000,-4.472){2}{\rule{0.400pt}{0.850pt}}
\put(1020.17,502){\rule{0.400pt}{2.300pt}}
\multiput(1019.17,508.23)(2.000,-6.226){2}{\rule{0.400pt}{1.150pt}}
\put(1022.17,490){\rule{0.400pt}{2.500pt}}
\multiput(1021.17,496.81)(2.000,-6.811){2}{\rule{0.400pt}{1.250pt}}
\put(1024.17,481){\rule{0.400pt}{1.900pt}}
\multiput(1023.17,486.06)(2.000,-5.056){2}{\rule{0.400pt}{0.950pt}}
\put(1026.17,475){\rule{0.400pt}{1.300pt}}
\multiput(1025.17,478.30)(2.000,-3.302){2}{\rule{0.400pt}{0.650pt}}
\put(923.0,338.0){\rule[-0.200pt]{0.482pt}{0.400pt}}
\put(1030.17,475){\rule{0.400pt}{0.900pt}}
\multiput(1029.17,475.00)(2.000,2.132){2}{\rule{0.400pt}{0.450pt}}
\put(1032.17,479){\rule{0.400pt}{1.500pt}}
\multiput(1031.17,479.00)(2.000,3.887){2}{\rule{0.400pt}{0.750pt}}
\put(1034.17,486){\rule{0.400pt}{1.900pt}}
\multiput(1033.17,486.00)(2.000,5.056){2}{\rule{0.400pt}{0.950pt}}
\put(1036.17,495){\rule{0.400pt}{1.500pt}}
\multiput(1035.17,495.00)(2.000,3.887){2}{\rule{0.400pt}{0.750pt}}
\put(1038.17,502){\rule{0.400pt}{1.300pt}}
\multiput(1037.17,502.00)(2.000,3.302){2}{\rule{0.400pt}{0.650pt}}
\put(1040,507.67){\rule{0.482pt}{0.400pt}}
\multiput(1040.00,507.17)(1.000,1.000){2}{\rule{0.241pt}{0.400pt}}
\put(1042,507.67){\rule{0.482pt}{0.400pt}}
\multiput(1042.00,508.17)(1.000,-1.000){2}{\rule{0.241pt}{0.400pt}}
\put(1044.17,503){\rule{0.400pt}{1.100pt}}
\multiput(1043.17,505.72)(2.000,-2.717){2}{\rule{0.400pt}{0.550pt}}
\multiput(1046.61,499.82)(0.447,-0.909){3}{\rule{0.108pt}{0.767pt}}
\multiput(1045.17,501.41)(3.000,-3.409){2}{\rule{0.400pt}{0.383pt}}
\put(1049.17,492){\rule{0.400pt}{1.300pt}}
\multiput(1048.17,495.30)(2.000,-3.302){2}{\rule{0.400pt}{0.650pt}}
\put(1051.17,487){\rule{0.400pt}{1.100pt}}
\multiput(1050.17,489.72)(2.000,-2.717){2}{\rule{0.400pt}{0.550pt}}
\put(1053,485.17){\rule{0.482pt}{0.400pt}}
\multiput(1053.00,486.17)(1.000,-2.000){2}{\rule{0.241pt}{0.400pt}}
\put(1055,484.67){\rule{0.482pt}{0.400pt}}
\multiput(1055.00,484.17)(1.000,1.000){2}{\rule{0.241pt}{0.400pt}}
\put(1057,486.17){\rule{0.482pt}{0.400pt}}
\multiput(1057.00,485.17)(1.000,2.000){2}{\rule{0.241pt}{0.400pt}}
\put(1059.17,488){\rule{0.400pt}{0.900pt}}
\multiput(1058.17,488.00)(2.000,2.132){2}{\rule{0.400pt}{0.450pt}}
\put(1061.17,492){\rule{0.400pt}{0.900pt}}
\multiput(1060.17,492.00)(2.000,2.132){2}{\rule{0.400pt}{0.450pt}}
\put(1063.17,496){\rule{0.400pt}{0.700pt}}
\multiput(1062.17,496.00)(2.000,1.547){2}{\rule{0.400pt}{0.350pt}}
\put(1065,499.17){\rule{0.482pt}{0.400pt}}
\multiput(1065.00,498.17)(1.000,2.000){2}{\rule{0.241pt}{0.400pt}}
\put(1067,500.67){\rule{0.482pt}{0.400pt}}
\multiput(1067.00,500.17)(1.000,1.000){2}{\rule{0.241pt}{0.400pt}}
\put(1069,500.67){\rule{0.482pt}{0.400pt}}
\multiput(1069.00,501.17)(1.000,-1.000){2}{\rule{0.241pt}{0.400pt}}
\put(1071.17,498){\rule{0.400pt}{0.700pt}}
\multiput(1070.17,499.55)(2.000,-1.547){2}{\rule{0.400pt}{0.350pt}}
\put(1073,496.17){\rule{0.482pt}{0.400pt}}
\multiput(1073.00,497.17)(1.000,-2.000){2}{\rule{0.241pt}{0.400pt}}
\put(1075.17,493){\rule{0.400pt}{0.700pt}}
\multiput(1074.17,494.55)(2.000,-1.547){2}{\rule{0.400pt}{0.350pt}}
\put(1077,491.67){\rule{0.482pt}{0.400pt}}
\multiput(1077.00,492.17)(1.000,-1.000){2}{\rule{0.241pt}{0.400pt}}
\put(1079,490.67){\rule{0.723pt}{0.400pt}}
\multiput(1079.00,491.17)(1.500,-1.000){2}{\rule{0.361pt}{0.400pt}}
\put(1082,490.67){\rule{0.482pt}{0.400pt}}
\multiput(1082.00,490.17)(1.000,1.000){2}{\rule{0.241pt}{0.400pt}}
\put(1028.0,475.0){\rule[-0.200pt]{0.482pt}{0.400pt}}
\put(1086,492.17){\rule{0.482pt}{0.400pt}}
\multiput(1086.00,491.17)(1.000,2.000){2}{\rule{0.241pt}{0.400pt}}
\put(1088,493.67){\rule{0.482pt}{0.400pt}}
\multiput(1088.00,493.17)(1.000,1.000){2}{\rule{0.241pt}{0.400pt}}
\put(1090,495.17){\rule{0.482pt}{0.400pt}}
\multiput(1090.00,494.17)(1.000,2.000){2}{\rule{0.241pt}{0.400pt}}
\put(1092,496.67){\rule{0.482pt}{0.400pt}}
\multiput(1092.00,496.17)(1.000,1.000){2}{\rule{0.241pt}{0.400pt}}
\put(1084.0,492.0){\rule[-0.200pt]{0.482pt}{0.400pt}}
\put(1096,496.67){\rule{0.482pt}{0.400pt}}
\multiput(1096.00,497.17)(1.000,-1.000){2}{\rule{0.241pt}{0.400pt}}
\put(1098,495.67){\rule{0.482pt}{0.400pt}}
\multiput(1098.00,496.17)(1.000,-1.000){2}{\rule{0.241pt}{0.400pt}}
\put(1100,494.67){\rule{0.482pt}{0.400pt}}
\multiput(1100.00,495.17)(1.000,-1.000){2}{\rule{0.241pt}{0.400pt}}
\put(1094.0,498.0){\rule[-0.200pt]{0.482pt}{0.400pt}}
\put(1104,493.67){\rule{0.482pt}{0.400pt}}
\multiput(1104.00,494.17)(1.000,-1.000){2}{\rule{0.241pt}{0.400pt}}
\put(1102.0,495.0){\rule[-0.200pt]{0.482pt}{0.400pt}}
\put(1113,493.67){\rule{0.482pt}{0.400pt}}
\multiput(1113.00,493.17)(1.000,1.000){2}{\rule{0.241pt}{0.400pt}}
\put(1115,494.67){\rule{0.482pt}{0.400pt}}
\multiput(1115.00,494.17)(1.000,1.000){2}{\rule{0.241pt}{0.400pt}}
\put(1117,495.67){\rule{0.482pt}{0.400pt}}
\multiput(1117.00,495.17)(1.000,1.000){2}{\rule{0.241pt}{0.400pt}}
\put(1106.0,494.0){\rule[-0.200pt]{1.686pt}{0.400pt}}
\put(1121,495.67){\rule{0.482pt}{0.400pt}}
\multiput(1121.00,496.17)(1.000,-1.000){2}{\rule{0.241pt}{0.400pt}}
\put(1119.0,497.0){\rule[-0.200pt]{0.482pt}{0.400pt}}
\put(1129,494.67){\rule{0.482pt}{0.400pt}}
\multiput(1129.00,495.17)(1.000,-1.000){2}{\rule{0.241pt}{0.400pt}}
\put(1123.0,496.0){\rule[-0.200pt]{1.445pt}{0.400pt}}
\put(1131.0,495.0){\rule[-0.200pt]{48.662pt}{0.400pt}}
\put(171.0,131.0){\rule[-0.200pt]{0.400pt}{175.375pt}}
\put(171.0,131.0){\rule[-0.200pt]{305.461pt}{0.400pt}}
\put(1439.0,131.0){\rule[-0.200pt]{0.400pt}{175.375pt}}
\put(171.0,859.0){\rule[-0.200pt]{305.461pt}{0.400pt}}
\end{picture}

	\caption{Again with a coarse tolerance of $\epsilon = 10^{-3}$.} 
\end{figure}

\subsubsection*{Calculation of Spatial Derivatives}
The calculation of spatial derivatives involves analyzing the detail coefficients, as they are a measure of how well the 
given function is approximated by the local interpolant. If a point $x_{k}^{j}$ on level $j$ of the dyadic grid, does 
not have detail coefficients above the threshold at points $x_{2k+1}^{j+1}$ or $x_{2k-1}^{j+1}$, then the given 
function's error is bounded by the thresholding parameter $\epsilon$.  Thus a stencil consisting of the same points
which constructed the underlying polynomial should result in an accurate result at that point, 
within some constant multiple of the tolerance $\epsilon$.

The computational complexity of constructing Lagrange interpolating polynomial coefficients is $\mathcal{O}(N)$. 
Once these terms are known, the weights needed for computing spatial derivatives on the adaptive grid can be computed 
with only slightly more effort.
This is also an $\mathcal{O}(N)$ operation. The analytic formula for the coefficients is given by
\begin{equation}
    \frac{d}{dx} L_{k+l}(x) = L_{k+l}(x) \sum_{ \substack{ i=k-N+1 \\ i\neq k+l } }^{k+N} \frac{1}{x_{k+l}-x_i}.
\end{equation}
However once this coefficient is calculated for each point in the stencil, it must be multiplied by its functional value, 
and all points are then summed again in the same way that the Lagrange polynomial is in the first place. 
Thus the total complexity of computing derivatives this way is $\mathcal{O}(N^2)$, which is not desirable. The order of 
accuracy for computing one spatial derivative this way is $\mathcal{O}((h^j)^{2N-1})$, where $h^j$ is the local grid spacing. 
Note that for a stencil consisting of four points $(N=2)$, this method is not only computationally inefficient, but also 1 order 
of accuracy worse than a comparable four-point centered finite difference scheme, which has $\mathcal{O}((h^j)^{2N})$
accuracy. The only advantage of this method can be seen when the stencil is non-uniform, in which case the finite difference 
coefficients would have to be computed either in an implicit sense, or by interpolation, as done here.

An alternative and perhaps more efficient method in some cases, is to precompute every possible finite-difference stencil on the grid
and store the coefficients in a lookup table \cite{rastigejev}.

\subsubsection*{Time-Stepping}
Advancement in time of the adaptive solution is done as in any method of lines procedure. The forcing term for the differential equation
(in time) is evaluated via appropriate finite differencing as previously explained. This procedure is repeated for as many times as there
are stages in the time integration method (4 for RK4). At each iteration, to maintain stability of the compressed solution,
a buffer zone of points is added. Typically the buffer zone ensures that each point that originally met the threshold is surrounded
by two points of the same resolution level. If these points are missing, it is possible that during time integration, the points
could become relevant (physics moving into that area) but that algorithm would have no way of knowing it.

\subsubsection*{Adaptive Wavelet Collocation Code}
A program was successfully developed in \texttt{C++} to solve numerous one-dimensional partial differential equations with the AWCM.
The code repository can be found at \texttt{https://github.com/blg13/Wavelet/tree/CollocationPoint} 

\subsection*{Multiresolution Finite-Volume}
Most multiresolution methods for PDEs do not explicitly compute wavelets, but rather use prediction operators and the multiresolution
framework for compression of the solution. This method was pioneered by Harten \cite{Harten}. The finite volume method works with cell
averages, $u_{j,k}$, where $j$ denotes the resolution level and $k$ denotes the spatial index. The general idea is to 
represent the cell-averaged data on a fine grid as values on a coarse grid plus the differences at subsequent grid levels. The idea
is identical in nature to the restriction and prolongation actions in the multigrid framework. 

\subsubsection*{Structured \& Regular Rectilinear Grid} 
As previously mentioned, the multiresolution analysis is tied to the structure of the grid. For regular, rectilinear grids
the prediction operators are fixed and can be easily derived in more than one dimension. In one dimension the update (fine
to coarse) operator is given by
\begin{equation}
\overline{u}_{j,k} = \frac{1}{2} ( \overline{u}_{j+1,2k} + \overline{u}_{j+1,2k+1} )
\end{equation}
and the prediction operator (coarse to fine) is given as
\begin{align}
\hat{u}_{j+1,2k} & = \overline{u}_{j,k} + \sum_{i=1}^{m} \gamma_k ( \overline{u}_{j,k+i} + \overline{u}_{j,k-i} ) \\
\hat{u}_{j+1,2k+1} & = \overline{u}_{j,k} - \sum_{i=1}^{m} \gamma_k ( \overline{u}_{j,k+i} + \overline{u}_{j,k-i} ),
\end{align}
where if fifth-order interpolation is used ($m=2$), then the coefficients are $\gamma_1=-\frac{22}{128}$ and $\gamma_2=\frac{3}{128}$.
The detail coefficients computed at every resolution level are given by
\begin{equation}
\overline{d}_{j,k} = \overline{u}_{j,k} - \hat{u}_{j,k}.
\end{equation}
If the prediction operator is stable, then the detail coefficients should be an indicator of regularity, or smoothness of the underlying
function. In this way, spatial adaptivity can be obtained by only accepting cells with detail coefficients of significant value.

\subsubsection*{Irregular Connectivity Meshes}
The need to compress large quantities of three-dimensional image data drove the graphics community over the last three decades to
develop many methods of subdivision for creating level-of-detail models. These meshes have no regular structure, and thus first-generation
multiresolution analysis is not possible.
 
% Print bibliography:
\printbibliography

\end{document}
