\documentclass[11pt]{article}
\usepackage{fullpage}
\usepackage{float}
\usepackage{graphicx}
\usepackage{amssymb}
\usepackage{subcaption}
\usepackage{mwe}
\usepackage{epstopdf}
\usepackage{color,verbatim}
\usepackage{bm}
\usepackage{listings}
\usepackage{mathtools}          %loads amsmath as well
\usepackage{titling}
\DeclareGraphicsRule{.tif}{png}{.png}{`convert #1 `dirname #1`/`basename #1 .tif`.png}
%\usepackage{doublespace}

\begin{document}
\title{Weekly Status Report}
\author{Development of a 1D Adaptive Wavelet Collocation Code \\
Brandon Gusto \\}
\date{Week Beginning 8/28/2017}

\maketitle
%
\section{Summary of method}
The adaptive wavelet collocation method for differential equations has a number of merits which may make it a viable
alternative to traditional finite element, finite volume, or high order finite difference methods for certain problems. 
Some important advantages include the following:
\begin{itemize}
    \item the multiresolution properties of wavelets allow for natural grid adaptation
    \item the existence of a fast $\mathcal{O}(\mathcal{N})$ scheme for computation of wavelet coefficients and 
          spatial derivatives
    \item the ability of second-generation wavelets to adapt to complex geometries
\end{itemize}
\subsection{Dyadic Grid}
The method makes use of a dyadic grid. Each grid level is defined by 
\begin{equation}
    \mathcal{G}^j= \{ x_{k}^{j} \in \Omega : k \in \mathcal{K}^j \}, j \in \mathcal{Z},
\end{equation}
where $\mathcal{K}^j$ is the integer set representing the spatial locations in the grid at level $j$. The grids are 
nested, implying that $\mathcal{G}^{j} \subset \mathcal{G}^{j+1}$. In other words, the points at level $x^{j}$ are 
a perfect subset of the points at level $x^{j+1}$. This can also be demonsrated by the relation that 
$x_{k}^{j}=x_{2k}^{j+1}$.
\subsection{Interpolating Subdivision}
The interpolating subdivision scheme is central to the second-generation wavelet collocation approach. The scheme is used to
approximate values at odd points $x_{2k+1}^{j+1}$ by constructing interpolating polynomials of order $2N-1$ with the points
at grid level $j$. Lagrange polynomials are used, and the method can be used with a uniform grid or with 
nonuniform points such as Chebychev points. The interpolating scheme is 
\begin{equation}
    f^j(x_{2k+1}^{j+1})=\sum_{l=-N+1}^{N} w_{k,l}^{j} f(x_{k+l}^{j}),
\end{equation}
where the coefficients $w_{k,l}^{j}$ are the lagrange polynomials $L_{2N,k+l}(x)$ evaluated at $x=x_{2k+1}^{j+1}$. The 
lagrange polynomial in this notation is given by 
\begin{equation}
    L_{2N,k+l}(x)=\prod_{i=k}^{2N+k+l} \frac{x-x_i}{x_
\end{equation}
For 
a uniform grid, the coefficients should be constant irrespective of the grid level $j$, reducing computational cost. 
\subsection{Wavelet Construction}
The construction of second-generation wavelets makes use of the dyadic grid. A fast interpolating subdivision scheme is 
used to interpolate functional values defined at points on level $j$, to odd points (i.e. $x_{2k+1}^{j+1}$) at the next
higher level of resolution. This scheme is used to construct the scaling and detail wavelet functions.
\section{This week's goals}
\begin{enumerate}
\item To use the scaling and detail wavelet functions, in conjunction with the forward wavelet transform, to approximate some initial function 
    $u(x)$. A function $u(x)$ may be approximated by
    \begin{equation}
        u^J(x)=\sum_{k \in \mathcal{K^0}} c_{k}^{0} \phi_{k}^{0}(x) + \sum_{j=0}^{J-1} \sum_{l \in \mathcal{L}^j}
                d_{l}^{j} \psi_{l}^{j}(x).
    \end{equation}
\item Once a function can be approximated, an algorithm to throw away small detail coefficients can be developed.
\item The grid points can then be altered using the binary tree structure.
\item Develop a code to calculate spatial derivatives as in Vasilyev \& Bowman (2000).
\end{enumerate}

\end{document}
